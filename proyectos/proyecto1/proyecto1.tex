\documentclass[letterpaper,12pt]{article}

% Soporte para los acentos.
\usepackage[utf8]{inputenc}
\usepackage[T1]{fontenc}    
% Idioma español.
\usepackage[spanish,mexico, es-tabla]{babel}
% Soporte de símbolos adicionales (matemáticas)
\usepackage{multirow}
\usepackage{amsmath}		
\usepackage{amssymb}		
\usepackage{amsthm}
\usepackage{amsfonts}
\usepackage{latexsym}
\usepackage{enumerate}
\usepackage{ragged2e}
% Soporte para el código.
\usepackage{listings}
% Eliminamos el título del abstract.
\AtBeginDocument{\renewcommand{\abstractname}{}}
% Definiciones, teoremas y lemas.
\newtheorem{define}{Definición}[]
\newtheorem{lema}{Lema}[]
\newtheorem{prop}{Proposición}[]
\newtheorem{teo}{Teorema}[]
\newtheorem{ejem}{Ejemplo}[]
% Deducción natural.
\usepackage{bussproofs}
% Haskell
\usepackage{color}
\lstset{
  frame=single,
  xleftmargin=2pt,
  belowcaptionskip=\bigskipamount,
  captionpos=b,
  escapeinside={*'}{'*},
  language=haskell,
  tabsize=2,
  emphstyle={\bf},
  commentstyle=\it,
  stringstyle=\mdseries\rmfamily,
  showspaces=false,
  keywordstyle=\bfseries\rmfamily,
  columns=flexible,
  basicstyle=\scriptsize\ttfamily,
  showstringspaces=false,
  morecomment=[l]\%,
}

% Modificamos los márgenes del documento.
\usepackage[lmargin=2cm,rmargin=2cm,top=2cm,bottom=2cm]{geometry}

% Datos del título.
\title{Lógica Computacional \\ 
Proyecto 1: Implementación de la solución de un problema lógico}
\author{Rubí Rojas Tania Michelle \\
        Universidad Nacional Autónoma de México \\
        taniarubi@ciencias.unam.mx \\
        $\#$ de cuenta: 315121719}
\date{\today}

\begin{document}
    \maketitle

    % Abstract: Incluí la bibliografía y el breve propósito del proyecto.
    \begin{abstract}
        La parte teórica del proyecto está basado en las notas de clase del
        curso de Lógica Computacional impartido por la profesora Estefanía 
        Prieto Larios y se nutrió con las sugerencias y correcciones de la 
        misma. El propósito de este proyecto es realizar la implementación 
        computacional de la solución a un clásico problema lógico.
    \end{abstract}

    % Sección 1. Lógica Proposicional.
    \section{Lógica Proposicional}

    % 1.1 Definición.
    \subsection{Definición}
    La lógica proposicional es el sistema lógico más simple. Se encarga del 
    manejo de proposiciones mediante conectivos lógicos. Una proposición es
    un enunciado que puede calificarse de verdadero o falso. 
    % Ejemplos.
    \begin{ejem} 
        Enunciados que son proposiciones.
        \begin{itemize}
            \item Los números pares son divisibles por dos. 
            \item Una ballena no es roja.
            \item He pasado mis vacaciones en Grecia.
        \end{itemize}
    \end{ejem}

    \begin{ejem} 
        Enunciados que no son proposiciones.
        \begin{itemize}
            \item ¡Auxilio, me desmayo!
            \item No sé si vendrán al viaje.
            \item $x + y$
        \end{itemize}
    \end{ejem}

    % 1.2 Sintaxis de la lógica proposicional.
    \subsection{Sintaxis de la lógica proposicional}
    Definimos ahora un lenguaje formal para la lógica proposicional. \\
    El alfabeto consta de:
    \begin{itemize}
        \item Símbolos o variables proposicionales (un número 
        infinito) : $p_{1}, ... , p_{n}, ...$
        \item Constantes lógicas: $\bot, \top$
        \item Conectivos u operadores lógicos: $\neg, \land, \lor 
        \rightarrow, \leftrightarrow$
        \item Símbolos auxiliares: $(,)$
    \end{itemize}

    El conjunto de expresiones o fórmulas atómicas, denotado $ATOM$ consta de:
    \begin{itemize}
        \item Las variables proposicionales: $p_{1}, ..., p_{n}, ...$
        \item Las constantes $\bot, \top$
    \end{itemize}
    
    Las expresiones que formarán nuestro lenguaje $PROP$, llamadas usualmente
    fórmulas, se definen recursivamente como sigue: 
    \begin{itemize}
        \item Si $\varphi \in ATOM$ entonces $\varphi \in PROP$. Es decir,
        toda fórmula atómica es una fórmula.
        \item Si $\varphi \in PROP$ entonces $(\neg \varphi) \in PROP$.
        \item $\varphi, \psi$ entonces $(\varphi \land \psi), 
        (\varphi \lor \psi), (\varphi \rightarrow \psi), 
        (\varphi \leftrightarrow \psi) \in PROP$.
        \item Son todas.
    \end{itemize}

    % Ejemplos.
    \begin{ejem} 
        Representación de enunciados en Lógica Proposicional.
        \begin{itemize}
            \item[i)] El enunciado \textbf{nuestra bandera es blanca y celeste} se
            puede ver en lógica proposicional como 
            \begin{center}
                $p \land q$
            \end{center}
            donde $p =$ nuestra bandera es blanca, y $q =$ nuestra bandera es
            celeste.
            
            \item[ii)] El enunciado \textbf{está nublado por lo que va a llover;
            entonces no saldremos} se puede ver en lógica proposicional como 
            \begin{center}
                $(a \rightarrow b) \rightarrow \neg c$
            \end{center}
            donde $a =$ está nublado, $b =$ va a llover y $c =$ saldremos.
        \end{itemize}
    \end{ejem}

    % 1.3 Semántica de la lógica proposicional.
    \subsection{Semántica de la lógica proposicional}
    % Definición. Valores booleanos.
    \begin{define} 
        El tipo de valores booleanos denotado \textbf{Bool} se define
        como $Bool = \{0, 1 \}$
    \end{define}
    % Definición. Estado de las variables.
    \begin{define}
        Un estado o asignación de las variables (proposicionales) es una 
        función 
        \begin{center}
            $\mathcal{I} : (Var P) \rightarrow Bool$
        \end{center}
        Dadas $n$ variables proposicionales existen $2^{n}$ estados 
        distintos para estas variables.
    \end{define}
    % Definición. Interpretación de las fórmulas.
    \begin{define}
        Dado un estado de las variables 
        $\mathcal{I} : (Var P) \rightarrow Bool$, definamos la interpretación
        de las fórmulas con respecto a $\mathcal{I}$ como la función 
        $\mathcal{I^{\star}} : PROP \rightarrow Bool$ tal que: 
        % Interpretación de cada una de las fórmulas.
        \begin{itemize}
            \item $\mathcal{I^{\star}}(p) = \mathcal{I}(p)$ para 
            $p \in Var P$, es decir, 
            $\mathcal{I^{\star}}|_{Var P} = \mathcal{I}$
            \item $\mathcal{I^{\star}}(\top) = 1$
            \item $\mathcal{I^{\star}}(\bot) = 0$
            \item $\mathcal{I^{\star}}(\neg \varphi) = 1$ sii 
            $\mathcal{I^{\star}}(\varphi) = 0$
            \item $\mathcal{I^{\star}}(\varphi \land \psi) = 1$
            sii $\mathcal{I^{\star}}(\varphi) = \mathcal{I^{\star}}(\psi) = 1$
            \item $\mathcal{I^{\star}}(\varphi \lor \psi) = 0$
            sii $\mathcal{I^{\star}}(\varphi) = 
            \mathcal{I^{\star}}(\psi) = 0$
            \item $\mathcal{I^{\star}}(\varphi \rightarrow \psi) = 0$
            sii $\mathcal{I^{\star}}(\varphi) = 1$ e
            $\mathcal{I^{\star}}(\psi) = 0$
            \item $\mathcal{I^{\star}}(\varphi \leftrightarrow \psi) = 1$
            sii $\mathcal{I^{\star}}(\varphi) = \mathcal{I^{\star}}(\psi)$
        \end{itemize}
    \end{define}

    Notemos que dado un estado de las variables $\mathcal{I}$, la 
    interpretación $\mathcal{I^{\star}}$ generada por $I$ está 
    determinada de manera única, por lo que de ahora en adelante
    escribiremos simplemente $\mathcal{I}$ en lugar de $\mathcal{I^{\star}}$.
    % Ejemplos.
    \begin{ejem} 
        Interpretación de fórmulas con valores arbitrarios.
        \begin{itemize}
            \item[i)] $(s \lor t) \leftrightarrow (s \land t)$ \\
            Si $\mathcal{I}(s) = 1$, $\mathcal{I}(t) = 0$, por la definición 
            de $\mathcal{I}$ tenemos que 
            $\mathcal{I}(s \lor t) = 1$ y $\mathcal{I}(s \land t) = 0$,
            por lo que 
            \begin{center}
                $\mathcal{I}((s \lor t \leftrightarrow (s \land t))) = 0$
            \end{center}
        
            \item[ii)] $(p \land q) \rightarrow \neg (r \land q)$ \\
            Si $\mathcal{I}(p) = 1$, $\mathcal{I}(q) = 1$, $\mathcal{I}(r) = 0$,
            por la definición de $\mathcal{I}$ tenemos que  
            $\mathcal{I}(p \land q) = 1$, $\mathcal{I}(r \land q) = 0$ y
            $\mathcal{I}(\neg (r \land q)) = 1$, por lo que
            \begin{center}
                $\mathcal{I}((p \land q) \rightarrow \neg (r \land q)) = 1$
            \end{center}
        \end{itemize}
    \end{ejem}

    El siguiente lema es de importancia para restringir las interpretaciones
    de interés al analizar una fórmula.

    \newpage
    % --- Primer lema que hay que demostrar ---
    \begin{lema} [\textbf{Coincidencia}]
        Sean $\mathcal{I}_{1}$, $\mathcal{I}_{2} : PROP \rightarrow Bool$ dos
        estados que coinciden en las variables proposicionales de la fórmula 
        $\varphi$, es decir, $\mathcal{I}_{1}(p) = \mathcal{I}_{2}(p)$ para 
        toda $p \in vars(\varphi)$. Entonces 
        $\mathcal{I}_{1}(\varphi) = \mathcal{I}_{2}(\varphi)$
    \end{lema}
    % Demostración.
    \begin{proof}
        Inducción estructural sobre $\varphi$.\\ \\
        % Base de inducción.
        \texttt{Base de inducción.} $\varphi$ es atómica.
        \begin{itemize}
            % Caso cuando phi es top.
            \item $\varphi = \top$. Notemos que 
            $\mathcal{I}_{1}(\top) = \mathcal{I}_{2}(\top) = 1$ siempre.
            % Caso cuando phi es bot.
            \item $\varphi = \bot$. Notemos que 
            $\mathcal{I}_{1}(\bot) = \mathcal{I}_{2}(\bot) = 0$ siempre.
            % Caso cuando phi es una variable proposicional.
            \item $\varphi = VarP$. Como la única fórmula atómica que ocurre
            en $\varphi$ es $\varphi$, tenemos que 
            $\mathcal{I}_{1}(\varphi) = \mathcal{I}_{2}(\varphi)$.
        \end{itemize}

        \justify
        % Hipótesis de inducción.
        \texttt{Hipótesis de inducción.} Supongamos que el argumento es cierto
        para la fórmula $\varphi '$, es decir, que se cumple 
        $\mathcal{I}_{1}(\varphi ') = \mathcal{I}_{2}(\varphi ')$. \\ \\
        % Paso inductivo.
        \texttt{Paso inductivo.} Tenemos $5$ casos:
        \begin{itemize}
            % Caso cuando la phi es la negación de una fórmula.
            \item $\varphi = \neg \varphi '$. Supongamos que 
            $\mathcal{I}_{1} = \mathcal{I}_{2}$ coinciden en las variables
            proposicionales de $\neg \varphi '$; pero las variables de 
            $\neg \varphi$ son las mismas variables de $\varphi$, y luego 
            por hipótesis de inducción tenemos que 
            $\mathcal{I}_{1}(\varphi) = \mathcal{I}_{2}(\varphi)$. Así, 
            \begin{align*}
                \mathcal{I}_{1}(\neg \varphi) = 1 
                &\Leftrightarrow \mathcal{I}_{1}(\varphi) = 0
                && \text{por definición de $\mathcal{I}$} \\
                &\Leftrightarrow \mathcal{I}_{2}(\varphi) = 0
                && \text{por Hipótesis de Inducción} \\
                &\Leftrightarrow \mathcal{I}_{2}(\neg \varphi) = 1
                && \text{por definición de $\mathcal{I}$}
            \end{align*}

            Por lo tanto, 
            $\mathcal{I}_{1}(\varphi) = \mathcal{I}_{2}(\varphi)$.

            % Caso cuando phi es la conjunción de dos fórmulas.
            \item $\varphi = \phi \land \psi$. Entonces tenemos que
            \begin{align*}
                \mathcal{I}_{1}(\phi \land \psi) = 1
                &\Leftrightarrow \mathcal{I}_{1}(\phi) = \mathcal{I}_{1}(\psi)
                = 1
                && \text{por definición de $\mathcal{I}$} \\
                &\Leftrightarrow \mathcal{I}_{2}(\phi) = \mathcal{I}_{2}(\psi)
                = 1
                && \text{por Hipótesis de Inducción} \\
                &\Leftrightarrow \mathcal{I}_{2}(\phi \land \psi) = 1
                && \text{por definición de $\mathcal{I}$}
            \end{align*}

            Por lo tanto, 
            $\mathcal{I}_{1}(\varphi) = \mathcal{I}_{2}(\varphi)$.

            % Caso cuando phi es una disyunción de fórmulas.
            \item $\varphi = \phi \lor \psi$. Entonces tenemos que 
            \begin{align*}
                \mathcal{I}_{1}(\phi \lor \psi) = 0
                &\Leftrightarrow \mathcal{I}_{1}(\phi) = \mathcal{I}_{1}(\psi)
                = 0
                && \text{por definición de $\mathcal{I}$} \\
                &\Leftrightarrow \mathcal{I}_{2}(\phi) = \mathcal{I}_{2}(\psi)
                = 0
                && \text{por Hipótesis de Inducción} \\
                &\Leftrightarrow \mathcal{I}_{2}(\phi \lor \psi) = 0
                && \text{por definición de $\mathcal{I}$} 
            \end{align*}

            Por lo tanto, 
            $\mathcal{I}_{1}(\varphi) = \mathcal{I}_{2}(\varphi)$.

            % Caso en que phi es la implicación de dos fórmulas.
            \item $\varphi = \phi \rightarrow \psi$. Entonces tenemos que 
            \begin{align*}
                \mathcal{I}_{1}(\phi \rightarrow \psi) = 0
                &\Leftrightarrow \mathcal{I}_{1}(\phi) = 1,
                \mathcal{I}_{1}(\psi) = 0
                && \text{por definición de $\mathcal{I}$} \\
                &\Leftrightarrow \mathcal{I}_{2}(\phi) = 1, 
                \mathcal{I}_{2}(\psi) = 0
                && \text{por Hipótesis de Inducción} \\
                &\Leftrightarrow \mathcal{I}_{2}(\phi \rightarrow \psi) = 0
            \end{align*}

            Por lo tanto, 
            $\mathcal{I}_{1}(\varphi) = \mathcal{I}_{2}(\varphi)$.

            % Caso en que phi es la equivalencia de dos fórmulas.
            \item $\varphi = \phi \leftrightarrow \psi$. Entonces tenemos que
            \begin{align*}
                \mathcal{I}_{1}(\phi \leftrightarrow \psi) = 1
                &\Leftrightarrow \mathcal{I}_{1}(\phi) = \mathcal{I}_{1}(\psi)
                && \text{por definición de $\mathcal{I}$} \\
                &\Leftrightarrow \mathcal{I}_{2}(\phi) = \mathcal{I}_{2}(\psi)
                && \text{por Hipótesis de Inducción} \\
                &\Leftrightarrow \mathcal{I}_{2}(\phi \leftrightarrow \psi) = 1
                && \text{por definición de $\mathcal{I}$}
            \end{align*}

            Por lo tanto, 
            $\mathcal{I}_{1}(\varphi) = \mathcal{I}_{2}(\varphi)$.

        \end{itemize}
    \end{proof}

    El lema anterior implica que aún cuando existen una infinidad de estados,
    dada una fórmula $\varphi$ basta considerar únicamente aquellos que 
    difieren en las variables proposicionales de $\varphi$, a saber $2^{n}$
    estados distintos si $\varphi$ tiene $n$ variables proposicionales.

    % 1.3.1 Conceptos semánticos básicos.
    \subsubsection{Conceptos semánticos básicos}
    % Definición de satisfacibilidad, tautología y contradicción.
    \begin{define}
        Sea $\varphi$ una fórmula. Entonces 
        \begin{itemize}
            % Definición de tautología.
            \item Si $\mathcal{I}(\varphi) = 1$ para toda interpretación 
            $\mathcal{I}$ decimos que $\varphi$ es una tautología o una 
            fórmula válida y escribimos $\models \varphi$. 
            % Definición de satisfacibilidad.
            \item Si $\mathcal{I}(\varphi) = 1$ para alguna interpretación
            $\mathcal{I}$ decimos que $\varphi$ es satisfacible o que
            $\mathcal{I}$ es modelo de $\varphi$ y escribimos 
            $\mathcal{I} \models \varphi$.
            % Definición de insatisfacibilidad.
            \item Si $\mathcal{I}(\varphi) = 0$ para alguna interpretación
            $\mathcal{I}$ decimos que $\varphi$ es falsa o insatisfacible
            en $\mathcal{I}$ o que $\mathcal{I}$ no es modelo de $\varphi$
            y escribimos $\mathcal{I} \not \models \varphi$.
            % Definición de contradicción.  
            \item Si $\mathcal{I}(\varphi) = 0$ para toda interpretación
            $\mathcal{I}$ decimos que $\varphi$ es una contradicción o fórmula
            no satisfacible. 
        \end{itemize}
    \end{define}

    \justify
    Similarmente, si $\Gamma$ es un conjunto de fórmulas decimos que:
    \begin{itemize}
        % Definición de conjunto satisfacible.
        \item $\Gamma$ es satisfacible si tiene un modelo, es decir, si existe
        una interpretación $\mathcal{I}$ tal que $\mathcal{I}(\varphi) = 1$
        para toda $\varphi \in \Gamma$.
        % Definición de conjunto insatisfacible.
        \item $\Gamma$ es insatisfacible si no tiene un modelo, es decir, si 
        no existe una interpretación $\mathcal{I}$ tal que 
        $\mathcal{I}(\varphi) = 1$ para toda $\varphi \in \Gamma$.
    \end{itemize}

    % Ejemplos.
    \begin{ejem} 
        Sean $\varphi$ una fórmula y $\Gamma$ un conjunto de fórmulas.
        \begin{itemize}
            % Ejemplo 1. Tautología.
            \item[i)] $\varphi = p \lor \neg p$ \\
            Notemos que si $\mathcal{I}(p) = 1$ entonces 
            $\mathcal{I}(\varphi) = 1$ pero si $\mathcal{I}(p) = 0$ entonces 
            también $\mathcal{I}(\varphi) = 1$. Como en ambos casos obtenemos 
            que $\mathcal{I}(\varphi) = 1$, entonces $\varphi$ es una 
            tautología.
            % Ejemplo 2. Satisfacibilidad.
            \item[ii)] $\Gamma = \{ p \rightarrow q, r \rightarrow s, \neg s\}$
            \newline 
            Si $\mathcal{I}(s) = \mathcal{I}(r) = \mathcal{I}(p) = 0$, entonces 
            $\mathcal{I}(\Gamma) = 1$, por lo que $\Gamma$ es satisfacible en 
            $\mathcal{I}$.
            % Ejemplo 3. Insatisfacibilidad.
            \item[iii)] $\varphi = p \rightarrow (q \lor r)$ \\
            Si $\mathcal{I}(p) = 1$ y $\mathcal{I}(q) = \mathcal{I}(r) = 0$, 
            entonces $\mathcal{I}(\varphi) = 0$, por lo que $\varphi$ es 
            insatisfacible en $\mathcal{I}$
            % Ejemplo 4. Contradicción.
            \item[iv)] $\Gamma = \{ p \rightarrow q, \neg (q \lor s), 
            s \lor p\}$ \\
            Notemos que $\Gamma$ es insatisfacible, pues supóngase que existe 
            una interpretación $\mathcal{I}$ tal que $\mathcal{I}(\Gamma) = 1$. 
            Entonces se tiene que $\mathcal{I}(\neg (q \lor s)) = 1$, por lo 
            que $\mathcal{I}(\neg q) = \mathcal{I}(\neg s) = 1$. Además, como 
            $\mathcal{I}(p \rightarrow q) = 1$, entonces $\mathcal{I}(p) = 0$,
            puesto que el consecuente de la implicación es falso. De esto 
            último se tiene que $\mathcal{I}(s) = 1$, dado que 
            $\mathcal{I}(s \lor p) = 1$. De manera que se tiene 
            $\mathcal{I}(\neg s) = 1 = \mathcal{I}(s)$, lo cual es imposible. 
            Por lo tanto, no puede existir una interpretación $\mathcal{I}$ 
            que satisfaga a $\Gamma$.
        \end{itemize}
    \end{ejem}

    \begin{prop}
        Sea $\Gamma = \{ \varphi_{1},..., \varphi_{n}\}$ un conjunto de 
        fórmulas.
        \begin{itemize}
            \item $\Gamma$ es satisfacible si y sólo si
            $\varphi_{1} \land ... \land \varphi_{n}$ es satisfacible.
            \item $\Gamma$ es insatisfacible si y sólo si 
            $\varphi_{1} \land ... \land \varphi_{n}$ es insatisfacible.
        \end{itemize}
    \end{prop}

    % Sección 2. Consecuencia Lógica.
    \section{Consecuencia Lógica}
    % Definición de consecuencia lógica.
    \begin{define}
        Sean $\Gamma$ un conjunto de fórmulas y $\varphi$ una fórmula. Decimos
        que $\varphi$ es consecuencia lógica de $\Gamma$ si para toda 
        interpretación $\mathcal{I}$ que satisface a $\Gamma$ se tiene que 
        $\mathcal{I}(\varphi) = 1$. Es decir, si se cumple que siempre que 
        $\mathcal{I}$ satisface a $\Gamma$ entonces necesariamente 
        $\mathcal{I}$ satisface a $\varphi$. En tal caso escribimos 
        $\Gamma \models \varphi$.
    \end{define}

    \justify
    Nótese que la relación de consecuencia lógica está dada por una implicación
    de la forma 
    \begin{center}
        $\mathcal{I}(\Gamma) = 1 \Rightarrow \mathcal{I}(\varphi) = 1$
    \end{center}

    \justify
    lo cual informalmente significa que \textit{todo modelo de $\Gamma$ es 
    modelo de $\varphi$}. \\
    Obsérvese la sobrecarga del símbolo $\models$ que previamente utilizamos
    para denotar satisfacibilidad $\mathcal{I} \models \varphi$ y tautologías
    $\models \varphi$.

    % Ejemplos.
    \begin{ejem}
        Considérese el siguiente conjunto 
        $\Gamma = \{ q \rightarrow p, p \leftrightarrow t, t \rightarrow s,
        s \rightarrow r\}$. Muestre que $\Gamma \models q \rightarrow r$. \\
        \textit{Solución:} Sea $\mathcal{I}$ un modelo de $\Gamma$. Tenemos 
        que demostrar que $\mathcal{I}(q \rightarrow r) = 1$. Si 
        $\mathcal{I}(q) = 0$ entonces $\mathcal{I}(q \rightarrow r) = 1$ y 
        terminamos. En otro caso se tiene que $\mathcal{I}(q) = 1$ de donde 
        $\mathcal{I}(p) = 1$ pues $\mathcal{I}(q \rightarrow p) = 1$. Entonces
        se tiene que $\mathcal{I}(t) = 1$, pues $\mathcal{I}$ es modelo de 
        $p \leftrightarrow t$, de donde $\mathcal{I}(s) = 1$ dado que 
        $\mathcal{I}$ también es modelo de $t \leftrightarrow s$. Finalmente,
        como $\mathcal{I}(s \rightarrow r) = 1$ e $\mathcal{I}(s) = 1$, 
        entonces $\mathcal{I}(r) = 1$. Por lo tanto, 
        $\mathcal{I}(q \rightarrow r) = 1$. 
    \end{ejem}
    
    \begin{ejem}
        Considérese el siguiente conjunto 
        $\Gamma = \{ (p \land q), (q \lor r), (\neg s) \}$. Muestre que 
        $ \Gamma \models p \land s$. \\
        \textit{Solución:} Debemos mostrar que $\mathcal{I}(p \land s) = 1$.
        Entonces 
        \begin{align*}
            \mathcal{I}(p \land q) &= 1
            && \text{Premisa (1)} \\
            \mathcal{I}(q \lor r) &= 1
            && \text{Premisa (2)} \\
            \mathcal{I}(\neg s) &= 1 
            && \text{Premisa (3)} \\
            \mathcal{I}(p) &= 1
            && \text{por (1)} \\
            \mathcal{I}(q) &= 1
            && \text{por (1)} \\
            \mathcal{I}(s) &= 0
            && \text{por (3)}
        \end{align*}

        De manera que la interpretación dada por 
        $\mathcal{I}(p) = \mathcal{I}(q) = 1$ e $\mathcal{I}(s) = 0$ e 
        $\mathcal{I}(r) =$ arbitrario, es un contraejemplo al argumento, pues 
        con esta interpretación tenemos que $\mathcal{I}(p \land s) = 0$.
        Por lo tanto, el argumento es falso.
    \end{ejem}

    % Propiedades de la consecuencia lógica.
    \begin{prop}
        La relación de consecuencia lógica cumple las siguientes propiedades:
        \begin{itemize}
            \item Si $\varphi \in \Gamma$ entonces $\Gamma \models \varphi$.
            \item Principio de refutación: $\Gamma \models \varphi$ si y 
            sólo si $\Gamma \cup \{ \neg \varphi\}$ es insatisfacible.
            \item $\Gamma \models \varphi \rightarrow \psi$ si y sólo si 
            $\Gamma \cup \{ \varphi \} \models \psi$.
            \item Insatisfacibilidad implica trivialidad: Si $\Gamma$ es 
            insatisfacible entonces $\Gamma \models \varphi$ para toda 
            $\varphi \in PROP$.
            \item Si $\Gamma \models \bot$ entonces $\Gamma$ es insatisfacible.
            \item $\varphi \equiv \psi$ si y sólo si $\varphi \models \psi$ y 
            $\psi \models \varphi$.
            \item $\models \varphi$ (es decir, si $\varphi$ es tautología)
            si y sólo si $\varnothing \models \varphi$ (es decir, $\varphi$
            es consecuencia lógica del conjunto vacío).
        \end{itemize}
    \end{prop}

    % Sección 3. Deducción Natural.
    \section{Deducción Natural}
    Los sistemas de deducción natural, introducidos por G. Gentzen en $1934$,
    son formalismos deductivos que modelan el razonamiento matemático ordinario
    de manera más fiel que un sistema axiomático o que el método de tableaux.
    \newline
    Un sistema de deducción natural consiste de reglas de inferencia donde las
    hipótesis se encuentran en la parte superior de una línea horizontal y la 
    conclusión en la parte inferior. Las reglas describen las formas para 
    \textbf{introducir} y \textbf{eliminar} cada uno de los conectivos lógicos.
    Las pruebas o derivaciones que se construyen en estos sistemas son 
    mediante la aplicación de dichas reglas en una sucesión adecuada que 
    relaciona conclusiones con premisas de reglas posteriores. De igual forma
    que en el razonamiento ordinario se pueden hacer hipótesis temporales 
    durante la prueba, las cuales se pueden \textbf{descargar} al 
    incorporarlas a la conclusión.

    \begin{define}
        Un contexto es un conjunto finito de fórmulas 
        $\{ \varphi_{1},..., \varphi_{n}\}$. Usualmente denotaremos un 
        contexto con $\Gamma, \Delta$. En lugar de $\Gamma \cup \Delta$ 
        escribimos $\Gamma, \Delta$. Análogamente, $\Gamma, \varphi$ denota al 
        contexto $\Gamma \cup \{ \varphi \}$.
    \end{define}

    \justify
    Así, siempre que un contexto sea de la forma $\Gamma, A$, suponemos que la 
    fórmula $A$ no figura en $\Gamma$.

    % 3.1 Reglas de inferencia.
    \subsection{Reglas de inferencia}
    La relación de derivabilidad o deducibilidad $\Gamma \vdash A$ se define
    recursivamente a partir de la regla de inicio
    \begin{prooftree}
        \AxiomC{}
        \RightLabel{(Hip)}
        \UnaryInfC{$\Gamma, \varphi \vdash \varphi$}
    \end{prooftree}

    dando reglas de introducción y eliminación para cada conectivo que
    queremos esté presente en el sistema:
    \begin{itemize}
        % Regla DN para la implicación.
        \item Implicación:
        \begin{prooftree}
            \AxiomC{$\Gamma, \varphi \vdash \psi$}
            \RightLabel{($\rightarrow$ I)}
            \UnaryInfC{$\Gamma \vdash \varphi \rightarrow \psi$}
            \DisplayProof
            \quad \quad \quad \quad 
            \AxiomC{$\Gamma \vdash \varphi \rightarrow \psi$}
            \AxiomC{$\Gamma \vdash \varphi$}
            \RightLabel{($\rightarrow$ E)}
            \BinaryInfC{$\Gamma \vdash \psi$}
        \end{prooftree}

        % Regla DN para la conjunción.
        \item Conjunción:
        \begin{prooftree}
            \AxiomC{$\Gamma \vdash \varphi$}
            \AxiomC{$\Gamma \vdash \psi$}
            \RightLabel{($\land$ I)}
            \BinaryInfC{$\Gamma \vdash \varphi \land \psi$}
            \DisplayProof
            \quad \quad
            \AxiomC{$\Gamma \vdash \varphi \land \psi$}
            \RightLabel{($\land$ E)}
            \UnaryInfC{$\Gamma \vdash \psi$}
            \DisplayProof
            \quad \quad 
            \AxiomC{$\Gamma \vdash \varphi \land \psi$}
            \RightLabel{($\land$ E)}
            \UnaryInfC{$\Gamma \vdash \varphi$}
        \end{prooftree}

        % Regla DN para la distunción.
        \item Disyunción:
        \begin{prooftree}
            \AxiomC{$\Gamma \vdash \varphi$}
            \RightLabel{($\lor$ I)}
            \UnaryInfC{$\Gamma \vdash \varphi \lor \psi$}
            \DisplayProof
            \quad \quad 
            \AxiomC{$\Gamma \vdash \psi$}
            \RightLabel{($\lor$ I)}
            \UnaryInfC{$\Gamma \vdash \varphi \lor \psi$}
            \DisplayProof
            \quad \quad 
            \AxiomC{$\Gamma \vdash \varphi \lor \psi$}
            \AxiomC{$\Gamma, \varphi \vdash \chi$}
            \AxiomC{$\Gamma, \psi \vdash \chi$}
            \RightLabel{($\lor$ E)}
            \TrinaryInfC{$\Gamma \vdash \chi$}
        \end{prooftree}
    \end{itemize}

    Notemos que mediante estas reglas de inferencia no estamos derivando 
    fórmulas sino expresiones de la forma $\Gamma \vdash A$, conocidas como 
    secuentes. En particular las reglas de inferencia son correctas con
    respecto a la consecuencia lógica, es decir, transforman secuentes válidos
    (respecto a $\models$) en secuentes válidos como lo asegura lo siguiente

    % Consecuencia lógica y DN.
    \begin{prop}
        Sean $\Gamma$ un contexto y $A,B,C$ fórmulas. Se cumple lo siguiente
        \begin{itemize}
            \item Si $\Gamma, A \models B$ entonces 
            $\Gamma \models A \rightarrow B$.
            \item Si $\Gamma \models A$ y $\Gamma \models A \rightarrow B$
            entonces $\Gamma \models B$.
            \item $\Gamma \models A \land B$ si y sólo si $\Gamma \models A$
            y $\Gamma \models B$.
            \item Si $\Gamma \models A$ entonces $\Gamma \models A \lor B$.
            \item Si $\Gamma \models B$ entonces $\Gamma \models A \lor B$.
            \item Si $\Gamma \models A \lor B$, $\Gamma, A \models C$ y 
            $\Gamma, B \models C$ entonces $\Gamma \models C$.
        \end{itemize}
    \end{prop}

    % Definición de derivación del secuente.
    \begin{define}
        Una derivación del secuente $\Gamma \vdash A$ es una sucesión finita
        de secuentes de la forma 
        $\Gamma_{1} \vdash A_{1},..., \Gamma_{n} \vdash A_{n}$ tal que:
        \begin{itemize}
            \item $\Gamma_{i} \vdash A_{1}$ es una instancia de la regla 
            (Hip) ó
            \item $\Gamma_{i} \vdash A_{1}$ es conclusión de alguna regla de 
            inferencia tal que las premisas necesarias figuran antes en la 
            sucesión.
            \item $\Gamma \vdash A$ es el último elemento de la sucesión.
        \end{itemize}
    \end{define}

    % Más definiciones.
    \begin{define}
        Si $\vdash \varphi$ es derivable, es decir, si 
        $\varnothing \vdash \varphi$ es derivable ($\varphi$ es derivable sin 
        hipótesis) entonces decimos que $\varphi$ es un teorema.
    \end{define}

    % 3.2 La Negación.
    \subsection{La Negación}
    Un sistema de deducción natural puede clasificarse, de acuerdo a qué clase
    de negación tenga, como minimal, intuicionista o clásico.

    % 3.2.1 Lógica minimal.
    \subsubsection{Lógica Minimal}
    Se dice que la lógica es minimal si no hay reglas para la negación $\neg$
    ni para lo falso $\bot$. En un sistema minimal, la constante $\bot$ está 
    presente pero no tiene propiedades particulares. En la presencia de $\bot$,
    el símbolo de negación se define como 
    \begin{center}
        $\neg A =_{def} A \rightarrow \bot$
    \end{center}

    \justify
    En cuyo caso hablamos de la negación constructiva, cuyas reglas de
    inferencia son:
    \begin{prooftree}
        \AxiomC{$\Gamma, A \vdash \bot$}\RightLabel{($\neg$ I)}
        \UnaryInfC{$\Gamma \vdash \neg A$}
        \DisplayProof
        \quad \quad
        \AxiomC{$\Gamma \vdash A$}
        \AxiomC{$\Gamma \vdash \neg A$}
        \RightLabel{($\neg$ E)}
        \BinaryInfC{$\Gamma \vdash \bot$}
    \end{prooftree}

    % 3.2.2 Lógica intuicionista.
    \subsubsection{Lógica Intuicionista}
    La lógica intuicionista se obtiene al agregar a la lógica minimal la regla
    de eliminación de lo falso ($\bot$ E) conocida también como 
    ex-falso-quodlibet.
    \begin{prooftree}
        \AxiomC{$\Gamma \vdash \bot$}
        \RightLabel{(EFQ)}
        \UnaryInfC{$\Gamma \vdash \varphi$}
    \end{prooftree}

    Se observa que cualquier fórmula derivada en la lógica minimal sigue
    siendo derivable en la lógica intuicionista.

    % 3.2.3 Lógica clásica.
    \subsubsection{Lógica Clásica}
    Para recuperar a la lógica clásica tenemos que postular alguna de las 
    siguientes reglas:
    \begin{itemize}
        \item Tercer excluido.
        \begin{prooftree}
            \AxiomC{}
            \RightLabel{(TE)}
            \UnaryInfC{$\Gamma \vdash A \lor \neg A$}
        \end{prooftree}

        \item Reducción al absurdo.
        \begin{prooftree}
            \AxiomC{$\Gamma, \neg A \vdash \bot$}
            \RightLabel{(RAA)}
            \UnaryInfC{$\Gamma \vdash A$}
        \end{prooftree}

        \item Eliminación de la doble negación.
        \begin{prooftree}
            \AxiomC{$\Gamma \vdash \neg \neg A$}
            \RightLabel{($\neg \neg$ E)}
            \UnaryInfC{$\Gamma \vdash A$}
        \end{prooftree}
    \end{itemize}

    % 3.3 Ejemplos de derivaciones.
    \subsection{Ejemplos de derivaciones}
    \begin{ejem}
        Demuestra que el siguiente secuente es válido usando deducción 
        natural. 
        \begin{center}
            $\{ p \rightarrow (q \lor r) \} \vdash
            (p \rightarrow q) \lor (p \rightarrow r)$
        \end{center}
        
        \begin{proof}
            Por el principio de refutación basta mostrar que
            \begin{center}
                $\Gamma = \{ p \rightarrow (q \lor r), 
                (p \land \neg q) \land (p \land \neg r) \} \vdash \bot$.
            \end{center}
            
            \justify
            Entonces 
            \begin{align*}
                1. \Gamma &\vdash p \rightarrow (q \lor r)
                && \text{Hip} \\
                2. \Gamma &\vdash (p \land \neg q) \land (p \land \neg r)
                && \text{Hip} \\
                3. \Gamma &\vdash p \land \neg q
                && \text{($\land$ E) 2} \\
                4. \Gamma &\vdash p \land \neg r
                && \text{($\land$ E) 2} \\
                5. \Gamma &\vdash p
                && \text{($\land$ E) 3} \\
                6. \Gamma &\vdash \neg q
                && \text{($\land$ E) 3} \\
                7. \Gamma &\vdash \neg r
                && \text{($\land$ E) 4} \\
                8. \Gamma &\vdash q \lor r
                && \text{($\rightarrow$ E) 1 y 5} \\
                9. \Gamma, q &\vdash q
                && \text{Hip} \\
                10. \Gamma, q &\vdash \bot 
                && \text{($\neg$ E) 6 y 9} \\
                11. \Gamma, r &\vdash r
                && \text{Hip} \\
                12. \Gamma, r &\vdash \bot 
                && \text{($\neg$ E) 7 y 11}
            \end{align*}
            
            \justify
            Así, como en ambos casos obtenemos $\bot$ podemos concluir que 
            $\{ p \rightarrow (q \lor r) \} \vdash 
            (p \rightarrow q) \lor (p \rightarrow r)$. 

        \end{proof}
    \end{ejem}

    \begin{ejem}
        Demuestra que el siguiente secuente es válido usando deducción 
        natural.
        \begin{center}
            $\vdash (p \land q \rightarrow r) \rightarrow p \rightarrow q
            \rightarrow r$
        \end{center}

        \begin{proof}
            Aplicando varias veces el Teorema de Deducción natural basta
            mostrar que 
            \begin{center}
                $\Gamma = \{(p \land q) \rightarrow r, p, q \} \vdash r$
            \end{center}

            \justify
            Entonces 
            \begin{align*}
                1. \Gamma &\vdash (p \land q) \rightarrow r
                && \text{Hip} \\
                2. \Gamma &\vdash p
                && \text{Hip} \\
                3. \Gamma &\vdash q 
                && \text{Hip} \\
                4. \Gamma &\vdash p \land q
                && \text{($\land$ I) 2 y 3} \\
                5. \Gamma &\vdash r
                && \text{($\rightarrow$ E) 1 y 4}
            \end{align*}
            
            \justify
            Por lo tanto, podemos concluir que
            $\vdash (p \land q \rightarrow r) \rightarrow p 
            \rightarrow q \rightarrow r$.
            
        \end{proof}
    \end{ejem}

    \newpage
    \begin{ejem}
        Demuestra que el siguiente secuente es válido usando deducción 
        natural.
        \begin{center}
            $\Gamma = \{p, p \rightarrow q,
            p \rightarrow (q \rightarrow r) \} \vdash r$
        \end{center}

        \begin{proof}
            \begin{align*}
                1. \Gamma &\vdash p
                && \text{Hip} \\
                2. \Gamma &\vdash p \rightarrow q
                && \text{Hip} \\
                3. \Gamma &\vdash p \rightarrow (q \rightarrow r)
                && \text{Hip} \\
                4. \Gamma &\vdash q
                && \text{($\rightarrow$ E) 1 y 2} \\
                5. \Gamma &\vdash q \rightarrow r
                && \text{($\rightarrow$ E) 1 y 3} \\
                6. \Gamma &\vdash r
                && \text{($\rightarrow$ E) 4 y 5} 
            \end{align*}

            \justify
            Por lo tanto, podemos concluir que 
            $\{p, p \rightarrow q, p \rightarrow (q \rightarrow r) \} 
            \vdash r$

        \end{proof}
    \end{ejem}

    % Sección 3.4 Teorema de Gödel.
    \subsection{Teorema de completud y correctud para la lógica clásica}
    El siguiente teorema vincula el mundo de la semántica con el de la 
    sintáxis de manera biunívoca. Estrictamente hablando, se conoce como
    \textbf{completud} a la implicación directa, mientras que al regreso 
    se le conoce como \textbf{correcctud}, ya que asegura que no se 
    pueden deducir cosas falsas.
    % --- Segundo Teorema que tengo que demostrar. ---
    \begin{teo}
        Sean $\Gamma$ un conjunto de fórmulas y $\varphi$ una fórmula.
        \begin{center}
            $\Gamma \models \varphi$ si y sólo si $\Gamma \vdash \varphi$
        \end{center}
    \end{teo}

    \begin{proof}
        Sean $\Gamma$ un conjunto de fórmulas y $\varphi$ una fórmula.
        \begin{itemize}
            \item[$\Rightarrow )$] Supongamos que $\Gamma \models \varphi$.
            Luego, para toda interpretación $\mathcal{I}$ tal que 
            $\mathcal{I}(\psi) = 1$ para toda $\psi \in \Gamma$, se da
            $\mathcal{I}(\varphi) = 1$. Esto equivale a decir que no hay 
            interpretación tal que $\mathcal{I}(\psi) = 1$ para toda 
            $\psi \in \Gamma$ y $\mathcal{I}(\varphi) = 0$, lo cual 
            implica que no se puede dar $\Gamma \not \vdash \varphi$, 
            es decir, obtenemos que $\Gamma \vdash \varphi$.

            \item[$\Leftarrow )$] Supongamos que $\Gamma \vdash \varphi$.
            Haremos inducción sobre $\Gamma \vdash \varphi$. Lo cual 
            equivale a probar que todas las reglas del sistema de lógica 
            clásica preservan la noción $\models$. \\ \\
            \texttt{Base de inducción.} Var 
        \end{itemize}
    \end{proof}

    % Sección 4. El problema a resolver.
    \section{El acertijo a resolver.}
    Se ha cometido un asesinato (sólo hay un asesino). Se sospecha del esposo,
    del amante y del mayordomo. Durante los interrogatorios cada sospechoso 
    hizo 2 declaraciones clave:
    \begin{itemize}
        \item Esposo.
        \begin{enumerate}
            \item Yo no lo hice.
            \item El mayordomo tampoco lo hizo.
        \end{enumerate}
        \item Mayordomo.
        \begin{enumerate}
            \item El esposo no lo hizo.
            \item Lo hizo el amante.
        \end{enumerate}
        \item Amante.
        \begin{enumerate}
            \item Yo no lo hice.
            \item Lo hizo el esposo.
        \end{enumerate}
    \end{itemize}

    Al final del juicio pudimos enterarnos de que uno de los sospechosos era un
    lógico que había dicho la verdad en sus dos declaraciones, otro sospechoso
    resultó ser un estafador ya que mintió en ambas declaraciones. El tercer 
    sospechoso resultó ser un loco que dijo la verdad en una declaración, pero 
    mintió en otra.
    El objetivo es determinar quién es el asesino, quién es el lógico, quién es
    el estafador y quién es el loco.

    % Sección 4: Implementación de la solución del acertijo.
    \section{Solución del problema lógico.}
    \subsection{Implementación de la solución.}
    Se crearon dos programas para este proyecto. Explicaremos detalladamente el 
    propósito de cada programa y sus funciones, aunque esto también se 
    encuentra en el programa, incluyendo un ejemplo de entrada y salida de 
    cada función.
    \begin{itemize}
        \item[1)] \textbf{LogicaProp.hs} \\
        Importamos la biblioteca \texttt{Data.List} para poder utilizar 
        la función \texttt{union} más adelante. Una variable proposiconal 
        será del tipo \texttt{Char} y un estado será una lista de tuplas donde 
        el primer componente de la tupla es una variable proposicional y su 
        segundo componente será el valor booleano asociado a dicha variable.
        Para su implementación funcional, creamos los sinónimos de tipo para 
        \texttt{VarP} como sinónimo de \texttt{Char}, y \texttt{Estado} como 
        sinónimo de \texttt{[(VarP, Bool)]}. Creamos el tipo de dato para las 
        fórmulas proposicionales, el cual definimos de la siguiente forma:

        \begin{lstlisting}
            data Prop = Top | Bot | Var VarP | Neg Prop | Conj Prop Prop 
            | Disy Prop Prop | Impl Prop Prop | Syss Prop Prop 
            deriving (Eq, Ord, Show)
        \end{lstlisting}

        donde \texttt{Top = True}, \texttt{Bot = False}, 
        \texttt{Var VarP = Var Char}, y los demás son los conectivos lógicos
        (binarios y unarios) que ya definimos anteriormente. \\
        También se dan ejemplos de variables proposicionales y fórmulas 
        como referencias al lector de cómo se deben escribir y pueda utilizar 
        el programa con mayor claridad en un futuro. \\ 
        En seguida se encuentran las funciones de lógica proposicional, las
        cuales son:

        \begin{itemize}
            \item \textbf{Función interp}. Recibe una fórmula $\varphi$ y un 
            estado e. Regresa la interpretación de $\varphi$ con el estado 
            dado.\\
            La función está implementada de la siguiente forma:
            \begin{lstlisting}
                interp :: Prop -> Estado -> Bool
                interp phi e = case phi of 
                    Var i -> buscaBool i e
                    Neg p -> not (interp p e)
                    Conj p q -> (interp p e) && (interp q e)
                    Disy p q -> (interp p e) || (interp q e)
                    Impl p q -> (not (interp p e)) || (interp q e)
                    Syss p q -> (interp p e) == (interp q e)
            \end{lstlisting}
            la cual es una aplicación directa de nuestra definición de 
            semántica. Aquí podemos notar dos cosas: la primera, que se 
            utilizó una equivalencia lógica para obtener la interpretación 
            de la implicación; y la segunda, que utilizamos una función 
            auxiliar \texttt{buscaBool}. Ésta recibe una variable proposicional
            $p$, y un estado $[(p,b)]$, y regresa la segunda componente del 
            primer par ordenado de la lista de estados I, cuyo primer 
            componente sea igual a la variable p.
            La función auxiliar está implementada de la siguiente forma:
            \begin{lstlisting}
                buscaBool :: (Eq p) => p -> [(p,b)] -> b 
                buscaBool p e = head [b | (x,b) <- e, p == x]
            \end{lstlisting}
            
            Esta función en particular se encuentra hasta abajo del código, en
            el apartado de \textit{Funciones auxiliares}.

            \item \textbf{Función vars}. Recibe una formula $\varphi$. Regresa
            la lista de variables proposicionales que figuran en $\varphi$, 
            sin repetición. \\
            La función está implementada de la siguiente forma:
            \begin{lstlisting}
                vars :: Prop -> [VarP]
                vars phi = case phi of
                    Var x -> [x]
                    Neg p -> vars p
                    Conj p q -> vars p `union` vars q
                    Disy p q -> vars p `union` vars q
                    Impl p q -> vars p `union` vars q
                    Syss p q -> vars p `union` vars q
            \end{lstlisting}

            En un inicio se utilizó \texttt{++} para concatenar las variables,
            pero hacía que fallara la función \texttt{estados} pues al parecer
            no concatenada de la forma que nosotros necesitabamos, y así se 
            consideraban más de $2^{n}$ casos. Buscando una solución al
            problema nos encontramos con que la función \texttt{union} era la 
            ideal para tener todos los elementos que necesitabamos y así no 
            irnos a Disneylandia.

            \item \textbf{Función estados}. Recibe una fórmula $\varphi$ con 
            $n-$variables proposicionales. Regresa la lista con los $2^{n}$
            estados distintos para $\varphi$. \\
            La función está implementada de la siguiente forma:
            \begin{lstlisting}
                estados:: Prop -> [Estado]
                estados phi = subconj (vars phi)
                    where subconj [] = [[]]
                          subconj (x:xs) = 
                            [(x,True):i | i <- subconj xs] 
                            ++ [(x,False):i | i <- subconj xs] 
            \end{lstlisting}

            Notemos que para obtener los $2^{n}$ estados posibles, debemos 
            obtener el subconjunto de listas de la lista de variables 
            proposicionales de $\varphi$, y hacer las combinaciones posibles
            entre los estados y los valores booleanos (que es justo lo que 
            hacemos en la sub-función \texttt{subconj}). Así, estamos
            regresando una lista con las listas de estados posibles para la 
            fórmula $\varphi$.  

            \item \textbf{Función varCN}. Recibe una fórmula $\varphi$. Regresa
            la lista de variables proposicionales que figuran en $\varphi$. La 
            diferencia con la función vars es que si la variable porposicional 
            tiene una negación, la manda a la lista junto con su conectivo 
            unario. \\
            La función está implementada de la siguiente forma:
            \begin{lstlisting}
                varCN :: Prop -> [Prop]
                varCN phi = case phi of
                Var x -> [Var x]
                (Neg (Var i)) -> [Neg (Var i)]
                Neg p -> varCN p
                Conj p q -> varCN p `union` varCN q
                Disy p q -> varCN p `union` varCN q
                Impl p q -> varCN p `union` varCN q
                Syss p q -> varCN p `union` varCN q
            \end{lstlisting}

            La gran utilidad de esta función será explicada más adelante.
        \end{itemize}

        \item[2)] \textbf{Proyecto1.hs} \\
        Importamos el módulo del programa anterior con 
        \texttt{import LogicaProp}, y nuevamente importamos la biblioteca 
        \texttt{Data.List} para poder utilizar la función \texttt{intersect}
        más adelante. \\
        Definimos las variables proposicionales que vamos a utilizar para 
        resolver el problema de la siguiente manera
        % Variables proposicionales del problema.
        \begin{itemize}
            \item \texttt{p} $=$ Lo hizo el esposo.
            \item \texttt{q} $=$ Lo hizo el amante.
            \item \texttt{r} $=$ Lo hizo el mayordomo.
        \end{itemize}

        las cuales corresponden a las proposiciones atómicas de las 
        declaraciones de los sospechosos. El conjunto de declaraciones 
        de los sospechosos queda como: 
        % Conjunto de declaraciones de los sospechosos.
        \begin{itemize}
            \item Declaración del esposo: $\{ \neg p, \neg r \}$
            \item Declaración del mayordomo: $\{ \neg p, q \}$
            \item Declaración del amante: $\{\neg q, p \}$
        \end{itemize}

        Y por la proposición 1 del reporte sabemos que podemos reescrbir
        estos conjuntos de declaraciones como:

        \begin{itemize}
            \item \texttt{desposo = (Conj (Neg (Var p)) (Neg (Var r)))}
            $\equiv \neg p \land \neg r$
            \item \texttt{damante = (Conj (Neg (Var p)) (Var q))}
            $\equiv \neg p \land q$
            \item \texttt{demayordomo = (Conj (Neg (Var q)) (Var p))}
            $\equiv \neg q \land p$
        \end{itemize}

        Finalmente, definimos la conjunción de la declaración de los tres 
        sospechosos
        \begin{center}
            \texttt{argumento = (Conj desposo (Conj dmayordomo damante))}
        \end{center}

        Enseguida están las funciones que son propias del proyecto.

        \begin{itemize}
            % Función unAsesino.
            \item \textbf{Función unAsesino} Recibe una fórmula. Regresa los
            únicos tres estados donde se cumple que sólo hay un asesino. \\
            La función está implementada de la siguiente forma:
            \begin{lstlisting}
                unAsesino :: Prop -> [Estado]
                unAsesino _ = 
                    [e | e <- estados argumento, 
                    (buscaBool p e == True && buscaBool q e == False && buscaBool r e == False)
                    || (buscaBool p e == False && buscaBool q e == True && buscaBool r e == False)
                    || (buscaBool p e == False && buscaBool q e == False && buscaBool r e == True)]
            \end{lstlisting}

            El problema nos dice que sólo hay un asesino, y de acuerdo a cómo 
            definimos nuestras variables proposicionales $p, q$ y $r$, entonces 
            en los estados donde se cumple que sólo hay un asesino pasa que $p$
            es verdadero y los demás son falsos, ó $q$ es verdadero y los demás
            falsos, ó $r$ es verdadero y los demás son falsos. Entonces esta 
            función es muy importante pues así sólo estaremos trabajando en los 
            estados que cumplen esta primera condición del problema. \\
            Para lograr esto, utilizamos una lista de comprensión para obtener
            la lista con los tres estados que estamos buscando. Los estados los
            obtenemos de los $2^{3} = 8$ estados posibles de nuestro argumento. 
            Las propiedades en la lista de comprensión usan la función auxiliar 
            \texttt{buscaBool} para definir cómo son los valores booleanos de 
            cada una de las variables proposicionales de los estados que 
            estamos buscando. Así garantizamos que buscamos los estados 
            correctos. \\
            Notemos que no nos importa qué fórmula recibamos como entrada 
            (debe ser una fórmula válida para que haskell no llore, así que 
            utilizamos argumento por simplicidad), la función siempre nos 
            arrojará el mismo resultado (que es lo que queremos). \\ \\
            \texttt{Salida de la función:}
            \begin{lstlisting}
                *Main> unAsesino argumento
                [[('p',True),('r',False),('q',False)],
                [('p',False),('r',True),('q',False)],
                [('p',False),('r',False),('q',True)]]
            \end{lstlisting}

            % Función modelos.
            \item \textbf{Función modelos} Recibe una fórmula $\varphi$. 
            Regresa la lista con todos los modelos que satisfacen a la 
            fórmula $\varphi$. \\
            La función está implementada de la siguiente forma:
            \begin{lstlisting}
                modelos :: Prop -> [Estado]
                modelos phi = [e | e <- unAsesino argumento, interp phi e == True]
            \end{lstlisting}

            Para obtener los modelos utilizamos una lista de comprensión, la 
            cual tiene como propiedad que la interpretación de $\varphi$ con
            los estados de \texttt{unAsesino} sean igual a \texttt{True} (y 
            como vimos en la parte teórica, esto nos dice que si se cumple 
            la propiedad descrita anteriormente, entonces ese es un modelo 
            de $\varphi$). Notemos que en lugar de tomar los estados de la 
            fórmula $\varphi$, tomamos los estados obtenidos de la función 
            \texttt{unAsesino} con el argumento. ¿Por qué? Bueno, esta 
            función unicamente la usaremos para obtener los modelos de cada
            una de las declaraciones de los sospechosos, y como los únicos 
            estados que nos interesan son los de \texttt{unAsesino}, 
            entonces utilizamos éstos. \\
            Esta función será de mucha utilidad en la función 
            \texttt{unaVerdad} ya que ahí empezamos a descartar estados que 
            no cumplan la propiedad de que sólo una persona dice la verdad. 
            Los estados que queden implican que existe al menos el 
            personaje lógico. \\ \\
            \texttt{Salida de la función:}
            \begin{lstlisting}
                *Main> modelos desposo
                [[('p',False),('r',False),('q',True)]]

                *Main> modelos dmayordomo
                [[('p',False),('r',False),('q',True)]]

                *Main> modelos damante
                [[('p',True),('r',False),('q',False)]]
            \end{lstlisting}

            % Función noModelos.
            \item \textbf{Función noModelos}. Recibe una fórmula $\varphi$.
            Regresa la lista con todos los estados que no satisfacen a la 
            fórmula $\varphi$. \\
            La función está implementada de la siguiente forma:
            \begin{lstlisting}
                noModelos :: Prop -> [Estado]
                noModelos phi = 
                    [e | e <- unAsesino argumento, interp phi e == False]
            \end{lstlisting}

            Para obtener los no modelos utilizamos una lista de comprensión, 
            la cual tiene como propiedad que la interpretación de $\varphi$ con
            los estados de \texttt{unAsesino} sean igual a \texttt{False} (y 
            como vimos en la parte teórica, esto nos dice que si se cumple 
            la propiedad descrita anteriormente, entonces ese no es un modelo 
            de $\varphi$). Notemos que en lugar de tomar los estados de la 
            fórmula $\varphi$, tomamos los estados obtenidos de la función 
            \texttt{unAsesino} con el argumento. ¿Por qué? Bueno, esta 
            función unicamente la usaremos para obtener los no modelos de cada
            una de las declaraciones de los sospechosos, y como los únicos 
            estados que nos interesan son los de \texttt{unAsesino}, 
            entonces utilizamos éstos. \\
            Esta función será de mucha utilidad en la función 
            \texttt{unaVerdad} ya que ahí empezamos a descartar estados que 
            no cumplan la propiedad de que sólo una persona dice la verdad. 
            Los estados que queden implican que existe al menos el 
            personaje lógico. \\ \\
            \texttt{Salida de la función:}
            \begin{lstlisting}
                *Main> noModelos desposo
                [[('p',True),('r',False),('q',False)],
                [('p',False),('r',True),('q',False)]]

                *Main> noModelos dmayordomo
                [[('p',True),('r',False),('q',False)],
                [('p',False),('r',True),('q',False)]]

                *Main> noModelos damante
                [[('p',False),('r',True),('q',False)],
                [('p',False),('r',False),('q',True)]]
            \end{lstlisting}

            % Función unaVerdad.
            \item \textbf{Función unaVerdad}. Recibe una fórmula. Regresa la 
            lista de estados que satisfacen a una de las declaraciones de los 
            sospechosos y al resto no. \\
            La función está implementada de la siguiente forma: 
            \begin{lstlisting}
                unaVerdad :: Prop -> [Estado]
                unaVerdad _ = 
                    [i | i <- interseccion (modelos desposo) 
                    (noModelos dmayordomo) (noModelos damante)] 
                    ++ [i | i <- interseccion (modelos dmayordomo) 
                    (noModelos desposo) (noModelos damante)] 
                    ++ [i | i <- interseccion (modelos damante) 
                    (noModelos desposo) (noModelos dmayordomo)]
            \end{lstlisting}

            El problema nos dice que hay una persona que dice dos verdades, 
            por lo que es buena idea comenzar descartando todos aquellos 
            estados que no cumplen la propiedad de tener sólo una persona 
            que dice la verdad (ya que de lo contrario no se cumple que 
            exista un lógico, un estadafor y un loco). Esto lo logramos 
            mediante la intersección de los modelos de una declaracion y los
            noModelos de las otras dos declaraciones. Así, nuestra lista de 
            comprensión tiene como propiedad todos los posibles casos de la 
            condición descrita anteriormente.\\
            Por ejemplo, sabemos que el estado 
            \texttt{[('p',False),('r',False),('q',True)]} satisface a la 
            declaracion del esposo. Gracias a esto ya tenemos al personaje 
            lógico (ya que si lo satisface, entonces dice dos verdades pues
            el conectivo de las declaraciones es la conjunción). Entonces, 
            este estado no tendría que ser un modelo para la declaración del
            mayordomo y del amante. Si este estado cumple lo anterior, 
            entonces cumple la propiedad de que al menos existe un lógico y
            ya sólo tendríamos que comprobar la existencia del loco y del 
            estafador. \\
            Notemos que no nos importa qué fórmula recibamos como entrada 
            (debe ser una fórmula válida para que haskell no llore, así que 
            utilizamos argumento por simplicidad), la función siempre nos 
            arrojará el mismo resultado (que es lo que queremos). \\ \\
            \texttt{Salida de la función:}
            \begin{lstlisting}
                *Main> unaVerdad argumento
                [[('p',True),('r',False),('q',False)]]
            \end{lstlisting}

            Este resultado se puede corroborar con los resultados dados en 
            las funciones anteriores de modelos y noModelos. \\
            Ahora, veamos el auxiliar que utilizamos aqui. La función 
            \texttt{interseccion} recibe tres listas \texttt{xs, ys, zs}. 
            Regresa una lista con los elementos que tienen en común las tres
            listas. \\
            La implementación de la función es de la siguiente forma:
            \begin{lstlisting}
                interseccion :: Eq a => [a] -> [a] -> [a] -> [a]
                interseccion [] [] [] = [] 
                interseccion xs ys zs = intersect (intersect xs ys) zs
            \end{lstlisting}

            Para nuestro caso particular, la utilizamos para obtener la 
            intersección de tres listas de listas en la función
            \texttt{unaVerdad}. Notemos que aquí utilizamos la función 
            intersect, ya que nos ahorra tener que hacer a pie la función de 
            intersección. Nuestra función simplemente es un caso particular 
            que necesitamos para el proyecto. \\
            Esta función la podemos encontrar hasta abajo del código, en el 
            apartado de \textit{Funciones auxiliares}.
        \end{itemize}

        Ahora que tenemos el estado que cumple que sólo hay un asesino y 
        sólo una persona dice la verdad, entonces sólo nos interesa saber 
        si este estado cumple con la propiedad de tener un loco y un 
        estafador. Para esto tenemos la siguiente función:

        % Función juicio. 
        \item \textbf{Función juicio}. Recibe una fórmula. Regresa la lista 
        con los estados que cumplen la propiedad de que exista un lógico 
        (dos verdades), un estafador (dos mentiras) y un loco (una verdad y
        una mentira). \\
        La implementación de la función es de la siguiente manera:
        \begin{lstlisting}
            juicio:: Prop -> [Estado]
            juicio _ = 
                [e | e <- unaVerdad argumento, 
                (sonDiferentes desposo e == True && sonIguales dmayordomo e == True) 
                && (sonIguales dmayordomo e == True && sonIguales damante e == True)
                || (sonIguales desposo e == True && sonDiferentes dmayordomo e == True) 
                && (sonDiferentes dmayordomo e == True && sonIguales damante e == True)
                || (sonIguales desposo e == True && sonIguales dmayordomo e == True) 
                && (sonIguales dmayordomo e == True && sonDiferentes damante e == True)]
        \end{lstlisting}

        Expliquemos cómo funciona juicio. Tomamos los estado que obtuvimos 
        de la función \texttt{unaVerdad}, el cuál sólo es uno (pero aún así 
        hay que verificar que ese estado cumple con las propriedades del 
        problema). Y como sabemos que en este estado sólo una persona dice 
        la verdad, entonces debemos buscar al estafador y al loco, los 
        cuales tienen una característica muy peculiar: la  interpretación 
        de cada una de las variables proposicionales de la declaración de 
        éstos dos personajes con el estado de la función 'unaVerdad' debería 
        ser diferente (si es el loco) e igual (si es el estafador). Entonces, 
        recordemos que sabemos que hay un lógico, el cual dice dos verdades 
        y por lo tanto la interpretación de cada una de sus variables es 
        igual (por ser conjunción) con el estado de \texttt{unaVerdad}. 
        Sabiendo esto, basta considerar tres casos simples en nuestra lista 
        de comprensión para deducir que existen los tres personajes 
        solicitados. Veremos sólo el primer caso, ya que los demás son 
        análogos. \\
        Supongamos que la interpretación de las variables de la declaracion 
        del esposo con el estado de 'unaVerdad' es diferente y la 
        interpretación de las variables de la declaración del mayordomo con 
        el estado de \texttt{unaVerdad} es igual. Entonces sabemos que el esposo 
        es el loco (ya que dice una verdad y una mentira) y el mayordomo 
        puede ser el estafador o el lógico. Pero si además, la interpretación 
        de las variables proposicionales de la declaración del amante con el 
        estado de \texttt{unaVerdad} son iguales entonces significa que el 
        amante también puede ser el estafador ó el lógico. Y como ese estado 
        de \texttt{unaVerdad}, es el mismo que aplicamos en las tres 
        declaraciones, entonces sabemos que ese estado es el que cumple la
        propiedad que estamos buscando. Bastaría revisar manualmente quién es 
        nuestro lógico y nuestro estafador. \\

        Notemos que no nos importa qué fórmula recibamos como entrada 
        (debe ser una fórmula válida para que haskell no llore, así que 
        utilizamos argumento por simplicidad), la función siempre nos 
        arrojará el mismo resultado (que es lo que queremos). \\ \\
        Salida de la función: 
        \begin{lstlisting}
            *Main> juicio argumento
            [[('p',True),('r',False),('q',False)]]
        \end{lstlisting}
    \end{itemize}

    % 5.2 ¿Quién es el asesino?
    \subsection{¿Quién es el asesino?}
    Abrimos nuestra terminal y nos movemos al directorio donde se encuentren 
    los archivos \textbf{LogicaProp.hs} y \textbf{Proyecto1.hs}. Una vez que 
    estemos en el directorio deseado, ejecutamos el comando \texttt{ghci}, 
    con el cual obtenemos el siguiente resultado:
    \begin{lstlisting}
        GHCi, version 8.6.3: http://www.haskell.org/ghc/  :? for help
        Prelude> 
    \end{lstlisting}

    \justify
    Luego, compilamos el programa principal \textbf{Proyecto1.hs} con el 
    comando \texttt{:l Proyecto1}
    \begin{lstlisting}
        Prelude> :l Proyecto1
        [1 of 2] Compiling LogicaProp       ( LogicaProp.hs, interpreted )
        [2 of 2] Compiling Main             ( Proyecto1.hs, interpreted )
        Ok, two modules loaded.
    \end{lstlisting}

    Para conocer quién es el asesino, ejecutamos la función \textbf{juicio} 
    del programa \textit{Proyecto1.hs} con el parámetro \textbf{argumento}
    \begin{lstlisting}
        *Main> juicio argumento
    \end{lstlisting}

    donde \textit{argumento} es la conjunción de las declaraciones de los 
    tres sospechosos. Dicha ejecución nos arroja el siguiente resultado
    \begin{lstlisting}
        [[('p',True),('r',False),('q',False)]]
    \end{lstlisting}

    \justify
    Este es el estado donde sólo hay un asesino, un lógico, un estafador 
    y un loco. Con el simple estado sabemos que \texttt{p = True}, y 
    recordando nuestra implementación, sabemos que $p$ corresponde a la 
    proposición \textit{Lo hizo el esposo}. Por lo tanto, 
    \textbf{el esposo es el asesino}. \\
    Finalmente, para saber quién es el loco, quién es el estafador y 
    quién es el lógico, basta realizar manualmente la interpretación de 
    cada una de las declaraciones de los sospechosos con el estado que 
    obtuvimos anteriormente. Así, 
    \begin{itemize}
        \item Declaración del esposo: $\{ \neg p, \neg r \}$. \\
        Como $\mathcal{I}(p) = 1$ y $\mathcal{I}(r) = 0$, entonces 
        $\mathcal{I}(\neg p) = 0$ y $\mathcal{I}(\neg r) = 1$. Por lo tanto,
        el esposo es el \textbf{loco}.
        \item Declaración del mayordomo: $\{\neg p, q \}$. \\
        Como $\mathcal{I}(p) = 1$ y $\mathcal{I}(q) = 0$, entonces 
        $\mathcal{I}(\neg p) = 0$. Por lo tanto, el mayordomo es el
        \textbf{estafador}.
        \item Declaración del amante: $\{\neg q, p \}$. \\
        Como $\mathcal{I}(q) = 0$ y $\mathcal{I}(p) = 1$, entonces 
        $\mathcal{I}(\neg q) = 1$. Por lo tanto, el amante es el 
        \textbf{lógico}. 
    \end{itemize}

    \justify
    Con esto, hemos resuelto por completo el acertijo.
    
\end{document}