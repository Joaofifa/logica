\documentclass[letterpaper,12pt]{article}

% Soporte para los acentos.
\usepackage[utf8]{inputenc}
\usepackage[T1]{fontenc}    
% Idioma español.
\usepackage[spanish,mexico, es-tabla]{babel}
% Soporte de símbolos adicionales (matemáticas)
\usepackage{multirow}
\usepackage{amsmath}		
\usepackage{amssymb}		
\usepackage{amsthm}
\usepackage{amsfonts}
\usepackage{latexsym}
\usepackage{enumerate}
\usepackage{ragged2e}
% Soporte para el código.
\usepackage{listings}
% Eliminamos el título del abstract.
\AtBeginDocument{\renewcommand{\abstractname}{}}
% Definiciones, teoremas y lemas.
\newtheorem{define}{Definición}[]
\newtheorem{lema}{Lema}[]
\newtheorem{prop}{Proposición}[]
\newtheorem{teo}{Teorema}[]
\newtheorem{ejem}{Ejemplo}[]
\newtheorem{cor}{Corolario}[]
% Deducción natural.
\usepackage{bussproofs}
% Haskell
\usepackage{color}
\lstset{
  frame=single,
  xleftmargin=2pt,
  belowcaptionskip=\bigskipamount,
  captionpos=b,
  escapeinside={*'}{'*},
  language=prolog,
  tabsize=2,
  emphstyle={\bf},
  commentstyle=\it,
  stringstyle=\mdseries\rmfamily,
  showspaces=false,
  keywordstyle=\bfseries\rmfamily,
  columns=flexible,
  basicstyle=\scriptsize\ttfamily,
  showstringspaces=false,
  morecomment=[l]\%,
}

% Modificamos los márgenes del documento.
\usepackage[lmargin=2cm,rmargin=2cm,top=2cm,bottom=2cm]{geometry}

% Datos del título.
\title{Lógica Computacional \\ 
Proyecto 2: Implementación de un laberinto en PROLOG}
\author{Rubí Rojas Tania Michelle \\
        Universidad Nacional Autónoma de México \\
        taniarubi@ciencias.unam.mx \\
        $\#$ de cuenta: 315121719}
\date{\today}

\begin{document}
    \maketitle
    \tableofcontents

    \newpage
    % Sección: Lógica de Predicados.
    \section{Lógica de Primer Órden}
    % Subsección: Sixtaxis. 
    \subsection{Sintaxis de la Lógica de Predicados}
    En contraposición al lenguaje de la Lógica Proposicional que está 
    determinado de manera única, no es posible hablar de un sólo lenguaje para
    la Lógica de Predicados. Dependiendo de la estructura semántica que tengamos
    en mente será necesario agregar símbolos particulares para denotar objetos
    y relaciones entre objetos. De esta manera, el alfabeto consta de dos partes  
    ajenas entre sí, la parte común a todos los lenguajes determinada por los 
    símbolos lógicos y auxiliares y la parte particular, llamada tipo de 
    semejanza o signatura del lenguaje.
    \begin{itemize}
      \item La parte común a todos los lenguajes consta de:
      \begin{itemize}
        \item Un conjunto finito de variables 
        \texttt{$Var \{x_{1}, ..., x_{n},...\}$}
        \item Constantes lógicas: $\top, \bot$
        \item Conectivos u operadores lógicos: 
        $\neg, \land, \lor, \rightarrow, \leftrightarrow$
        \item Cuantificadores: $\forall, \exists$
        \item Símbolos auxiliares: $(, )$ y $,$ (coma).
        \item Si se agrega el símbolo de igualdad $=$, decimos
        que el lenguaje tiene igualdad.
      \end{itemize}
      
      \item La signatura de un lenguaje en particular está dada por:
      \begin{itemize}
        \item Un conjunto $\mathcal{P}$, posiblemente vacío, de símbolos o 
        letras de predicado:
        \begin{center}
          $P_{1}, ..., P_{n}, ...$
        \end{center}
        A cada símbolo se le asigna un índice o número de argumentos $m$, el 
        cual se hace explícito escribiendo $P_{n}^{(m)}$ lo cual significará 
        que el símbolo $P_{n}$ necesita de $m$ argumentos.
        \item Un conjunto $\mathcal{F}$, posiblemente vacío, de símbolos o 
        letras de función:
        \begin{center}
          $f_{1},..., f_{n},...$
        \end{center}
        Análogamente a los símbolos de predicado cada símbolo de función tiene 
        un índice asignado, $f_{n}^{(m)}$ significará que el símbolo $f_{n}$ 
        necesita de $m$ argumentos.       
        \item Un conjunto $\mathcal{C}$, posiblemente vacío, de símbolos de 
        constante:
        \begin{center}
          $c_{1}, ..., c_{n}, ...$
        \end{center}      
        En algunos libros los símbolos de constante se consideran como parte 
        del conjunto de símbolos de función, puesto que pueden verse como 
        funciones de índice cero, es decir, funciones que no reciben 
        argumentos.
      \end{itemize}
      
      Dado que un lenguaje de primer órden queda determinado de manera única
      por su signatura, abusaremos de la notación y escribiremos
      \begin{center}
        $\mathcal{L} = \mathcal{P} \cup \mathcal{F} \cup \mathcal{C}$
      \end{center}
      para denotar al lenguaje dado por tal signarura.
    \end{itemize}
    
    Los términos del lenguaje son variables, constantes y funciones aplicados a 
    estos.
    \begin{define}[\textbf{Términos}]
      Los términos son definidos como sigue:
      \begin{itemize}
        \item Una variable es un término.
        \item Si $c \in F$ es una función nula, entonces $c$ es un término.
        \item Si $t_{1}, t_{2}, ..., t_{n}$ son términos y $f \in F$ tiene
        índice $n > 0$, entonces $f(t_{1}, t_{2}, ..., t_{n})$ es un término.
        \item Son todos.
      \end{itemize}
    \end{define}
    
    En forma de Backus Naur se escribe:
    \begin{center}
      $t :: x$ $|$ $c$ $|$ $f(t_{1},...,t_{n})$
    \end{center}
    
    donde $x$ son variables, $c$ constantes y $f$ funciones sobre términos con 
    índice $n > 0$.
    
    \begin{define}
      Sean $P$ el conjunto de predicados y $F$ el conjunto de símbolos de
      función. Se define el conjunto de fórmulas sobre $(F, P)$ inductivamente 
      usando la definición de Términos sobre $F$.
      \begin{itemize}
        \item Si $p \in \mathcal{P}$ (símbolo de predicado) con índice 
        $n \geq 1$ y si $t_{1}, t_{2}, ..., t_{n}$ son términos sobre 
        $\mathcal{F}$ entonces $P(t_{1}, t_{2},..., t_{n})$ es una fórmula. 
        \item Si $\varphi$ es una fórmula, entonces $\neg \varphi$ también 
        lo es.
        \item $\varphi, \psi$ son fórmulas, entonces $\varphi \star \psi$ 
        también lo son; donde 
        $\star \in \{ \land, \lor, \rightarrow, \leftrightarrow \}$ 
        \item Si $\varphi$ es una fórmula y $x$ es una variable, entonces 
        $\forall x \varphi$ es una fórmula.
        \item Si $\varphi$ es una fórmula y $x$ es una variable, entonces 
        $\exists x \varphi$ es una fórmula.
        \item No hay más.
      \end{itemize}
    
    \end{define}
    
    Es decir, la sintaxis de la lógica de predicados en forma de Backus Naur se 
    define como:
    \begin{center}
      $\varphi :: P(t_{1}, ..., t_{n})$ $|$ $\neg \varphi$ $|$ 
      $\varphi \land \varphi$ $|$ $\varphi \lor \varphi$ $|$ 
      $\varphi \rightarrow \varphi$ $|$ $\varphi \leftrightarrow 
      \varphi$ $|$ $\forall x \varphi$ $|$ $\exists x \varphi$ $|$
      $t_{1} = t_{2}$ $|$ $\bot$ $|$ $\top$
    \end{center}

    % Ejemplos.
    \begin{ejem}
      Supongamos que el universo del discurso consta de los países del mundo y 
      sus ciudades.
      \begin{itemize}
        \item Las variables $x$ e $y$ denotan países cualesquiera.
        \item La constante $a$ denota a Alemania y la constante $i$ a Italia.
        \item El símbolo funcional $f$ de índice 1 denota a la operación que 
        recibe un país y devuelve su ciudad capital. Es decir, $f(x)$ es la 
        capital de $x$. Esto es posible dado que cada país tiene una única 
        capital, en particular $f(a)$ es Berlín y $f(i)$ es Roma.
      \end{itemize}
    \end{ejem}

    \begin{ejem}
      Si el universo consta de números naturales, entonces:
      \begin{itemize}
        \item La constante $a$ denota al individuo $0$ y la constante $b$ al 
        individuo $1$.
        \item Los términos $f^{2}(x,y)$ y $g^{2}(x,y)$ denotan a los individuos 
        $x + y$ y $x*y$, respectivamente.
        \item En tal caso, los individuos $2$ y $4$ se representan mediante 
        $f(b,b)$ y $g(f(b,b), f(b,b))$, respectivamente.
      \end{itemize}
    \end{ejem}
    
    % Subsección: Especificación Formal.
    \subsection{Especificación Formal}
    El proceso de especificación o traducción del español a la Lógica Formal no 
    es siempre sencillo. Algunas frases del español no se pueden traducir o
    especificar de una manera completamente fiel a la lógica de predicados,
    como veremos en algunos ejemplos.

    \subsubsection{Algunos consejos}
    \begin{itemize}
      \item Únicamente podemos especificar afirmaciones o porposiciones, no es 
      posible traducir preguntas, exclamaciones, órdenes, invitaciones, etc.
      \item La idea básica es extraer predicados a partir de los enunciados
      dados en español de manera que el enunciado completo se construya al 
      combinar dichos predicados mediante conectivos y cuantificadores. Por 
      ejemplo, la frase \textit{me gustan las alitas y el color rosa} debe
      entenderse como \textit{me gustan las alitas} y \textit{me gusta el 
      color rosa}. 
      \item La conjunción $y$ se traduce como $\land$. La palabra $pero$ también 
      corresponde a una conjunción aunque el sentido original del español se 
      pierde. Por ejemplo, \textit{te doy mis audífonos pero me los devuelves}
      sólo puede especificarse como \textit{te doy mis audifonos y me los 
      devuelves}. Lo cual es diferente en español.
      \item La disyunción es incluyente: \textit{iremos al parque o al cine},
      incluye el caso en que se vayan a ambos lados.
      \item Con la implicación hay que ser cautelosos, sobre todo en el caso de 
      frases de la forma $A$ sólo si $B$, lo cual es equivalente a 
      \textit{Si no $B$ entonces no $A$}, que a su vez es equivalente con 
      \textit{Si $A$ entonces $B$}. Es un error común intentar especificar dicha
      frase inicial mediante $B \rightarrow A$.
      \item Si en el español aparecen frases como \textit{para todos, para 
      cualquier, todos, cualesquiera, los, las}, etc; debe usarse el 
      cuantificador universal $\forall$.
      \item Si en el español aparecen frases como \textit{para algún, existe un,
      alguno, alguna, uno, una}, etc; debe usarse el cuantificador existencial
      $\exists$.
      \item Cualquier especificación compuesta que involucre cuantificadores
      puede formarse identificando en ella alguno de los cuatro juicios
      aristotélicos:
      \begin{itemize}
        \item Universal Afirmativo: Todo $S$ es $P$ traducido mediante 
        $\forall x(S(x) \rightarrow P(x))$.
        \item Existencial afirmativo: Algún $S$ es $P$ traducido mediante 
        $\exists x (S(x) \land P(X))$.
        \item Universal negativo: Ningún $S$ es $P$ traducido mediante 
        $\forall x (S(x) \rightarrow \neg P(x))$
        \item Existencial negativo: Algún $S$ no es $P$ traducido mediante 
        $\exists x (S(x) \land \neg P(x))$
      \end{itemize}

      Por supuesto que en especificaciones complicadas, $S$y $P$ no serán 
      predicados sino fórmulas compuestas.

      \item Pronombres como \textit{él, ella, eso} no se refieren a un 
      individuo particular, sino que se usan como referencia a algo o alguien
      mencionado previamente, por lo que obtienen significado del contexto
      particular. Cuando un pronombre aparezca en un enunciado, hay que 
      averiguar a quién o qué se refiere. \\
      Por ejemplo, en el enunciado \textit{Hércules es mascota de Sebastián
      pero él no es mascota de Mía} debe traducirse como \textit{Hércules es
      mascota de Sebastián y Hércules no es mascota de Mía}.
      \item Las variables no se mencionan en español, son sólo un formalismo
      para representar individuos, por ejemplo la fórmula 
      $\forall x(G(x) \rightarrow P(x))$ puede escribirse  en español como 
      \textit{Todo gato es pequeño} y no como \textit{Para todo x, si x es 
      gato entonces x es pequeño}. 
      \item Los esquemas $\forall x (A \rightarrow B)$ y $\exists x (A \land B)$
      son de gran utilidad y bastante comúnes. Menos comúnes, aunque también 
      adecuados, son los esquemas $\forall x (A \land B)$, $\forall x (A \lor B)$
      y $\exists x (A \lor B)$.
      \item El esquema $\exists x (A \rightarrow B)$, si bien es una fórmula 
      sintácticamente correcta, es extremadamente raro que figure en una 
      traducción del español.
      \item El hecho de que se usen dos o más variables distintas no implica que
      éstas representen a elementos distintos del universo, de manera que para 
      especificar dos individuos distintos no es suficiente contar simplemente 
      con variables distintas. \\
      Las fórmulas $\exists x P(x)$ y $\exists x \exists y(P(x) \land P(y))$
      expresan ambas lo mismo, a saber que un objeto del universo cumple $P$.
      Se debe agregar explícitamente que $x$ e $y$ tienen la propiedad de ser 
      distintos, es decir $x \neq y$.
    \end{itemize}

    % Ejemplos.
    \begin{ejem}
      Tenemos los siguientes predicados en el universo de discurso de los 
      habitantes de la Ciudad de México: \\ \\
      $I(x) : x$ es inteligente. \\
      $E(x) : x$ es estudiante de la Facultad de Ciencias. \\
      $M(x) : $a $x$ le gusta la música. \\ \\
      Especificar con cuantificadores los siguientes enunciados:
      \begin{itemize}
        \item Todos los estudiantes de la Facultad de Ciencias son inteligentes.
        \begin{center}
          $\forall x (E(x) \rightarrow I(x))$
        \end{center}
        \item A algunos estudiantes inteligentes les gusta la música.
        \begin{center}
          $\exists x (E(x) \land I(x) \land M(x))$
        \end{center}
        \item Todo aquel a quien le gusta la música es un estudiante que no es 
        inteligente.
        \begin{center}
          $\forall x (M(x) \rightarrow E(x) \land \neg I(x))$
        \end{center}
      \end{itemize}
    \end{ejem}

    \begin{ejem}
      En este ejemplo observamos el significado de las distintas combinaciones
      de dos cuantificadores. Sea $Q(x, y)$ el predicado $x$ quiere a $y$.
      \begin{itemize}
        \item Todos quieren a alguien.
        \begin{center}
          $\forall x \exists y Q(x,y)$
        \end{center}
        \item Alguien quiere a todos.
        \begin{center}
          $\exists x \forall y Q(x,y)$
        \end{center}
        \item Todos se quieren, o bien, todos quieren a todos.
        \begin{center}
          $\forall x \forall y (Q(x,y))$
        \end{center}
        \item Algunos se quieren entre sí, o bien, alguien quiere a alguien.
        \begin{center}
          $\exists x \exists y Q(x,y)$
        \end{center}
        \item Alguno no es querido por nadie.
        \begin{center}
          $\exists x \forall y \neg Q(y,x)$
        \end{center}
        \item Alguien no quiere a nadie.
        \begin{center}
          $\exists x \forall y \neg Q(x,y)$
        \end{center}
        \item Todos no quieren a alguien.
        \begin{center}
          $\forall x \exists y \neg Q(x,y)$
        \end{center}
        \item Nadie quiere a todos.
        \begin{center}
          $\neg \exists x \forall y Q(x,y)$
        \end{center}
        \item Nadie quiere a nadie.
        \begin{center}
          $\forall x \forall y \neg Q(x,y)$
        \end{center}
      \end{itemize}
    \end{ejem}

    % Sebsección: Semántica.
    \subsection{Semántica de la Lógica de Predicados}
    \begin{define}
      Sea $\mathcal{L} = \mathcal{P} \cup \mathcal{F} \cup \mathcal{C}$ un 
      lenguaje de primer órden. Una estructura o interpretación para 
      $\mathcal{L}$ es un par $\mathcal{M} = \langle M, \mathcal{I} \rangle$ 
      donde $M \neq \varnothing$ es un conjunto no vacío llamado el universo 
      de la estructura e $\mathcal{I}$ es una función con dominio $\mathcal{L}$ 
      tal que 
      \begin{itemize}
        \item Si $P^{(n)} \in \mathcal{P}$ entonces $\mathcal{I}(P)$ es una 
        relación de $m$-argumentos sobre $M$, es decir, 
        $\mathcal{I}(\mathcal{P}) \subseteq M^{n}$. Alternativamente podemos 
        definir la interpretación de $\mathcal{P}$ como una función booleana 
        que decide si una tupla está o no en la relación deseada, es decir, 
        $\mathcal{I}(\mathcal{P}): M^{n} \rightarrow Bool$.
        
        \item Si $f^{(n)} \in \mathcal{F}$ entonces $\mathcal{I}(f)$ es una 
        función con dominio $M^{n}$ y contradominio $M$, es decir, 
        $\mathcal{I}(f): M^{n} \rightarrow M$.
        
        \item Si $c \in \mathcal{C}$ entonces $\mathcal{I}(c)$ es un elemento 
        de $M$, es decir, $\mathcal{I}(c) \in M$
      
      \end{itemize}
      
      Dada una interpretación $\mathcal{M} = \langle M,  \mathcal{I} \rangle$, 
      la siguiente notación es de utilidad:
      \begin{center}
        $|\mathcal{M}| =_{def} M$ \\
        $P^{\mathcal{I}} =_{def} \mathcal{I}(P)$ \\
        $f^{\mathcal{I}} =_{def} \mathcal{I}(f)$ \\
        $c^{\mathcal{I}} =_{def} \mathcal{I}(c)$
      \end{center}
    \end{define}

    % Subsección: Interpretación de términos.
    \subsection{Interpretación de términos}
    \begin{define}[\textbf{Estado o Asignación}]
      Un estado, asignación o evaluación de las variables es una función 
      $\sigma: Var \rightarrow M$
    \end{define}

    \begin{define}[\textbf{Interpretación de Términos}]
      Sea $\sigma$ un estado de las variables. Definimos la función de 
      interpretación o significado de los términos con respecto a
      $\sigma, \mathcal{I_{\sigma}}: TERM \rightarrow |\mathcal{M}|$
      como sigue:
      \begin{center}
          $\mathcal{I_{\sigma}}(x) = \sigma(x)$ \\
          $\mathcal{I_{\sigma}}(c) = \mathcal{I}(c)$ \\
          $\mathcal{I_{\sigma}}(f(t_{1},...,t_{n})) =
          f^{\mathcal{I}}(\mathcal{I_{\sigma}}(t_{1}),...,
          \mathcal{I_{\sigma}}(t_{n}))$                
      \end{center}
      
    \end{define}

    \begin{lema}[\textbf{Lema de coincidencia para términos}]
      Sean $t \in TERM$ y $\sigma_{1}, \sigma_{2}$ dos estados de las variables
      tales que $\sigma_{1}(x) = \sigma_{2}(x)$ para toda variable $x$ que 
      figura en $t$. Entonces $\mathcal{I}_{\sigma_{1}}(t) =$
      $\mathcal{I}_{\sigma_{2}}(t)$.
    \end{lema}

    % Subsección: Interpretación de fórmulas.
    \subsection{Interpretación de fórmulas}
    \begin{define}[\textbf{Interpretación de fórmulas}]
      Sea $\sigma$ un estado de las variables. Definimos la función de 
      interpretación o significado de las fórmulas con respecto a 
      $\sigma, \mathcal{I_{\sigma}}: FORM \rightarrow \{ 0,1 \}$ como 
      sigue:
      \begin{center}
        $\mathcal{I_{\sigma}}(\bot) = 0$ \\
        $\mathcal{I_{\sigma}}(\top) = 1$ \\
        $\mathcal{I_{\sigma}}(P(t_{1},...,t_{m})) = 1$ si y sólo si 
        $(\mathcal{I_{\sigma}}(t_{1}), ..., \mathcal{I_{\sigma}}(t_{m})) 
        \in P^{\mathcal{I}}$ \\
        $\mathcal{I_{\sigma}}(t_{1}=t_{2}) = 1$ si y sólo si 
        $\mathcal{I_{\sigma}}(t_{1}) = \mathcal{I_{\sigma}}(t_{2})$ \\
        $\mathcal{I_{\sigma}}(\neg \varphi) = 1$ si y sólo si 
        $\mathcal{I_{\sigma}}(\varphi) = 0$ \\
        $\mathcal{I_{\sigma}}(\varphi \land \psi) = 1$ si y sólo si 
        $\mathcal{I_{\sigma}}(\varphi) = \mathcal{I_{\sigma}}(\psi) = 1$ \\
        $\mathcal{I_{\sigma}}(\varphi \lor \psi) = 0$ si y sólo si 
        $\mathcal{I_{\sigma}}(\varphi) = \mathcal{I_{\sigma}}(\psi) = 0$ \\
        $\mathcal{I_{\sigma}}(\varphi \rightarrow \psi) = 0$ si y sólo si 
        $\mathcal{I_{\sigma}}(\varphi) = 1$ e $\mathcal{I_{\sigma}}(\psi) = 0$ \\
        $\mathcal{I_{\sigma}}(\varphi \leftrightarrow \psi) = 1$ si y sólo si 
        $\mathcal{I_{\sigma}}(\varphi) = \mathcal{I_{\sigma}}(\psi)$ \\
        $\mathcal{I_{\sigma}}(\forall x \varphi) = 1$ si y sólo si 
        $\mathcal{I_{\sigma}}_{[x/m]}(\varphi) = 1$ para todo $m \in M$ \\
        $\mathcal{I_{\sigma}}(\exists x \varphi) = 1$ si y sólo si
        $\mathcal{I_{\sigma}}_{[x/m]}(\varphi) = 1$ para algún $m \in M$.
      \end{center}

    \end{define}

    \begin{lema} [\textbf{Lema de coincidencia para fórmulas}]
      Sean $\varphi$ y $\sigma_{1}, \sigma_{2}$ dos estados de las variables 
      tales que $\sigma_{1}(x) = \sigma_{2}(x)$ para toda variable 
      $x \in FV(\varphi)$. Entonces $\mathcal{I}_{\sigma_{1}}(\varphi) = 
      \mathcal{I}_{\sigma_{2}}(\varphi)$
    \end{lema}

    % Subsección: Verdad.
    \subsubsection{Verdad}
    \begin{define}[\textbf{Satisfacibilidad y Verdad}]
      Sean $\varphi$ una fórmula y 
      $\mathcal{M} = \langle M, \mathcal{I} \rangle$ una interpretación. 
      Entonces
      \begin{itemize}
        \item $\varphi$ es satisfacible en $\mathcal{M}$ si existe un estado de 
        las variables $\sigma$ tal que $\mathcal{I}_{\sigma}(\varphi) = 1$, lo 
        cual suele denotarse como $\mathcal{M} \models_{\sigma} \varphi$.
        \item $\varphi$ es verdadera en $\mathcal{M}$ si para todo estado de
        las variables $\sigma$ se tiene que $\mathcal{I}_{\sigma}(\varphi) = 1$,
        es decir, si $\varphi$ es satisfacible en $\mathcal{M}$ en todos los 
        estados posibles.
        En tal caso también decimos que $M$ es un modelo de $\varphi$, lo cual 
        se denotará con $\mathcal{M} \models \varphi$.
      \end{itemize}
    \end{define}

    Las definiciones anteriores  aplican también para conjuntos de fórmulas, 
    por ejemplo, $\mathcal{M} \models \Gamma$ denota el hecho de que $\Gamma$
    tiene un modelo, y $\mathcal{M} \models_{\sigma} \Gamma$ que todas las fórmulas 
    de $\Gamma$ se satisfacen en $\mathcal{M}$ en el estado $\sigma$.

    % Subsección: Falsedad.
    \subsubsection{Falsedad}
    \begin{define}
      Sean $\mathcal{M} = \langle M, \mathcal{I} \rangle$ una interpretación y
      $\varphi$ una fórmula. Decimos que $\varphi$ es falsa en $\mathcal{M}$ si
      y sólo si $\mathcal{M} \models \neg \varphi$. Es decir, $\varphi$ es 
      falsa sy y sólo si su negación $\neg \varphi$ es verdadera.
    \end{define}

    Así, una fórmula no verdadera es aquella tal que es insatisfacible en 
    algún estado de sus variables, o bien tal que su negación es 
    satisfacible en algún estado de sus variables. Sin embargo, para poder 
    afirmar que $\varphi$ es falsa, por definición tendríamos que mostrar
    que $\mathcal{M} \models \neg \varphi$, es decir que $\neg \varphi$ es 
    satisfacible en todos los estados posibles.

    % Ejemplo.
    \begin{ejem}
      Sean $\mathcal{L} = \{ P^{(1)}, Q^{(1)}\}$, 
      $\mathcal{M} = \langle \mathbb{N}, \mathcal{I} \rangle$ donde 
      $P^{\mathcal{I}}$ es la propiedad \textit{ser par} y $Q^{\mathcal{I}}$ es 
      la propiedad \textit{ser impar}. \\
      Entonces, $\mathcal{M} \models P(x) \lor Q(x)$ puesto que cualquier 
      número natural es par o impar. Sin embargo, no se cumple que 
      $\mathcal{M} \models P(x)$ ni que $\mathcal{M} \models Q(x)$. Puesto que 
      el valor de $x$ no puede ser siempre par o siempre impar, todo depende 
      del estado de las variables.
    \end{ejem}

    % Subsección: Propiedades de satisfacibilidad.
    \subsubsection{Propiedades de satisfacibilidad}
    \begin{prop}
      Sean $\mathcal{M}$ una interpretación y $\varphi, \psi$ fórmulas.
      Entonces 
      \begin{itemize}
        \item Si $\mathcal{M} \models \varphi$ entonces 
        $\mathcal{M} \not \models \neg \varphi$.
        \item Si $\mathcal{M} \models \varphi \rightarrow \psi$ y 
        $\mathcal{M} \models \varphi$ entonces $\mathcal{M} \models \psi$
        \item Si $\mathcal{M} \models \varphi$ entonces 
        $\mathcal{M} \models \varphi \lor \psi$.
        \item $\mathcal{M} \models \varphi \land \psi$ si y sólo si 
        $\mathcal{M} \models \varphi$ y $\mathcal{M} \models \psi$.
        \item $\mathcal{M} \models \varphi$ si y sólo si 
        $\mathcal{M} \models \forall \varphi$, donde $\forall \varphi$ denota 
        a la cerradura universal de $\varphi$, es decir, a la fórmula obtenida
        al cuantificar universalmente todas las variables libres de $\varphi$.
      \end{itemize}
    \end{prop}

    \begin{prop}
      Sean $\varphi$ un enunciado y 
      $\mathcal{M} = \langle M, \mathcal{I} \rangle$ una interpretación. 
      Entonces se cumple una y sólo una de las siguientes condiciones:
      \begin{enumerate}
        \item $\mathcal{M} \models \varphi$, es decir, $\varphi$ es verdadero 
        en $\mathcal{M}$.
        \item $\mathcal{M} \models \neg \varphi$, es decir, $\varphi$ es falso 
        en $\mathcal{M}$.
      \end{enumerate} 
    \end{prop}

    \begin{cor}
      Si $\varphi$ es un enunciado, entonces $\varphi$ es falso si y sólo si 
      $\varphi$ es no verdadero en $\mathcal{M}$.
    \end{cor}

    % Subsección: Modelos.
    \subsection{Modelos}
    Sabemos que un modelo para una fórmula $\varphi$ es una interpretación 
    $\mathcal{M} = \langle M, \mathcal{I} \rangle$ tal que 
    $\mathcal{M} \models \varphi$. Entonces, veamos algunos ejemplos:
    \begin{ejem}
      Sea $\Gamma = \{ P(b), Q(b), R(b), \exists(P(x) \land 
      \neg (Q(x) \lor R(x))), \forall x (R(x) \rightarrow P(x))\}$. Queremos 
      ver si $\Gamma$ tiene un modelo. \\
      En este caso, lo más fácil es ir construyendo un modelo para cada 
      fórmula de $\Gamma$. Si logramos que todas las fórmulas de $\Gamma$ sean 
      verdaderas al mismo tiempo entonces habremos construido un modelos para 
      $\Gamma$. Analicemos cada fórmula de $\Gamma$.
    \end{ejem}

    % Sección: El laberinto a resolver (puede variar).
    \section{El laberinto a resolver}
    % Sección: Solución del problema.
    \section{Solución del problema}
    % Subsección: Bracktracking.
    \subsubsection{Técnica de Backtracking}
    % Subsección: Implememtación.
    \subsubsection{Implementación de la solución}
    % Sección: Bibliografía.
    \section{Bibliografía}
\end{document}