\documentclass[letterpaper,12pt]{article}

% Soporte para los acentos.
\usepackage[utf8]{inputenc}
\usepackage[T1]{fontenc}    
% Idioma español.
\usepackage[spanish,mexico, es-tabla]{babel}
% Soporte de símbolos adicionales (matemáticas)
\usepackage{multirow}
\usepackage{amsmath}		
\usepackage{amssymb}		
\usepackage{amsthm}
\usepackage{amsfonts}
\usepackage{latexsym}
\usepackage{enumerate}
\usepackage{ragged2e}
% Modificamos los márgenes del documento.
\usepackage[lmargin=2cm,rmargin=2cm,top=2cm,bottom=2cm]{geometry}

\title{Lógica Computacional \\ Tarea Semanal 7}
\author{Rubí Rojas Tania Michelle}
\date{\today}

\begin{document}
\maketitle

Encuentra un programa $t$ que tenga el tipo indicado:

\begin{enumerate}
    % Ejercicio 1.
    \item[a)] $\vdash t: (A \rightarrow B \rightarrow C) \rightarrow$ 
    $(A \rightarrow B) \rightarrow (A \rightarrow C)$ \\
    \textsc{Solución:}
    \begin{enumerate}
        \item[1.] $f: A \rightarrow B \rightarrow C, x: A \rightarrow B, y: A
        \vdash f: A \rightarrow B \rightarrow C$ \; \; \; $(Hip)$
        \item[2.] $f: A \rightarrow B \rightarrow C, x: A \rightarrow B, y: A
        \vdash x: A \rightarrow B$ \; \; \;  \; \; \; \; $(Hip)$
        \item[3.] $f: A \rightarrow B \rightarrow C, x: A \rightarrow B, y: A
        \vdash y: A$ \; \; \; \; \; \; \; \; \; \;  \; $(Hip)$ 
        \item[4.] $f: A \rightarrow B \rightarrow C, x: A \rightarrow B, y: A
        \vdash fx: C$ \; \; \; \; \; \; \; \; \; \; $(\rightarrow E) \; 1,2$
        \item[5.] $f: A \rightarrow B \rightarrow C, x: A \rightarrow B
        \vdash fun(y: A.fx) : A \rightarrow C$ \; \; \;$(\rightarrow I) \; 4$
        \item[6.] $f: A \rightarrow B \rightarrow C \vdash fun(x: A \rightarrow B.fun(y: A.fx)) : 
        (A \rightarrow B) \rightarrow (A \rightarrow C)$ \; \; $(\rightarrow I) \; 5$
        \item[9.] $\vdash fun(f: A \rightarrow B \rightarrow C.
        fun(x: A \rightarrow B.fun(y: A.fx)) : \\
        (A \rightarrow B \rightarrow C) \rightarrow (A \rightarrow B) \rightarrow (A \rightarrow C)$ 
        \; \; \; \; \; \; \; \; \; \; \; \; \; $(\rightarrow I) \; 6$ 
    \end{enumerate}

    % Ejercicio 2.
    \item[b)] $x: (A \rightarrow C) \land (B \rightarrow C) \vdash$
    $t: A \lor B \rightarrow C$ \\
    \textsc{Solución:}
    \begin{enumerate}
        \item[1.] $x: (A \rightarrow C) \land (B \rightarrow C), 
        y: A \lor B \vdash x: (A \rightarrow C) \land (B \rightarrow C)$
        \; \; \; \; \; $(Hip)$
        \item[2.] $x: (A \rightarrow C) \land (B \rightarrow C), 
        y: A \lor B \vdash y: A \lor B$ \; \; \; \; \; \; \; \; \; \; \; \; \; \; \; \;$(Hip)$
        \item[3.] $x: (A \rightarrow C) \land (B \rightarrow C), 
        y: A \lor B \vdash snd \; x: B \rightarrow C$ \; \; \; \; \; \; \; \; \; \; \; \; $(\land E) \; 1$
        \item[4.] $x: (A \rightarrow C) \land (B \rightarrow C), 
        y: A \lor B \vdash fst \; x: A \rightarrow C$ \; \; \; \; \; \; \; \; \; \; \; \; \;($\land E) \; 1$
        \item[5.] $x: (A \rightarrow C) \land (B \rightarrow C), 
        y: A \lor B, r: A \vdash r: A$ \; \; \; \; \; \; \; \; \; \; \; \; \; \; \; $(Hip)$
        \item[6.] $x: (A \rightarrow C) \land (B \rightarrow C), 
        y: A \lor B, r: A \vdash fstxr: C$ \; \; \; \; \; \; \; \; \; \; \; \; $(\rightarrow E) \; 4,5$ 
        \item[7.] $x: (A \rightarrow C) \land (B \rightarrow C), 
        y: A \lor B, s: B \vdash s: B$ \; \; \; \; \; \; \; \; \; \; \; \; \; \; \; \;$(Hip)$ 
        \item[8.] $x: (A \rightarrow C) \land (B \rightarrow C), 
        y: A \lor B, s: B \vdash sndxs: C$ \; \; \; \; \; \; \; \; \; \; \; \; $(\rightarrow E) \; 3,7$
        \item[9.] $x: (A \rightarrow C) \land (B \rightarrow C), 
        y: A \lor B \vdash \\ 
        case \; y \; of \; inl r \Rightarrow fstx r\; | \; inr s \Rightarrow snd xs : C$ \; \; \; \;\; \; \; \; \; \; \; \; \; \; \; \; \; \; \; \;$(\lor E) \; 2, 6, 8$
        \item[10.] $x: (A \rightarrow C) \land (B \rightarrow C) \vdash \\ 
        fun(y: A \lor B.(case \; y \; of \; inl r \Rightarrow fstxr\; | \; inrs \Rightarrow sndxs)) : A \lor B \rightarrow C$ \; \;$(\rightarrow I)\; 9$
    \end{enumerate}

    % Ejercicio 3.
    \item[c)] $x: P \rightarrow Q \land R \vdash t: (P \rightarrow Q) \land$
    $(P \rightarrow R)$ \\
    \textsc{Solución:} Sabemos que $\Gamma \vdash A \land B \Leftrightarrow 
    \Gamma \vdash A$ y $\Gamma \vdash B$. Así, basta probar cada uno de los
    lados de la conjunción por separado. Entonces 
    \begin{enumerate}
        \item PD. $x: P \rightarrow Q \land R \vdash P \rightarrow Q$
        \begin{enumerate}
            \item[1.] $x: P \rightarrow Q \land R, y: P \vdash x: P \rightarrow Q \land R$ \; \; \; \; \; $(Hip)$
            \item[2.] $x: P \rightarrow Q \land R, y: P \vdash y: P$ \; \; \; \; \; \; \;  \; \; \; \; \; $(Hip)$
            \item[3.] $x: P \rightarrow Q \land R, y: P \vdash xy: Q \land R$ \; \; \; \; \; \; \; \; $(\rightarrow E) \; 1,2$ 
            \item[4.] $x: P \rightarrow Q \land R, y: P \vdash fst xy: Q$ \; \; \; \; \; \; \; \; \; $(\land E) \; 3$ 
            \item[5.] $x: P \rightarrow Q \land R \vdash fun(y: P.fst xy): P \rightarrow Q$ \; \;$(\rightarrow I) \; 4$   
        \end{enumerate}

        \item PD. $x: P \rightarrow Q \land R \vdash P \rightarrow R$
        \begin{enumerate}
            \item[6.] $x: P \rightarrow Q \land R, y: P \vdash x: P \rightarrow Q \land R$ \; \; \; \; \; $(Hip)$
            \item[7.] $x: P \rightarrow Q \land R, y: P \vdash y: P$ \; \; \; \; \; \; \;  \; \; \; \; \; $(Hip)$
            \item[8.] $x: P \rightarrow Q \land R, y: P \vdash xy: Q \land R$ \; \; \; \; \; \; \; \; $(\rightarrow E) \; 1,2$
            \item[9.] $x: P \rightarrow Q \land R, y: P \vdash snd xy: R$  \; \; \; \; \; \; \; \; \; $(\land E) \; 3$ 
            \item[10.] $x: P \rightarrow Q \land R \vdash fun(y: P.snd xy): P \rightarrow R$ \; \;$(\rightarrow I) \; 4$
        \end{enumerate}
    \end{enumerate}
    Por lo tanto, $x: P \rightarrow Q \land R \vdash \langle fun(y: P.fst xy): P \rightarrow Q, \; 
    fun(y: P.snd xy): P \rightarrow R \rangle : (P \rightarrow Q) \land (P \rightarrow R)$ por $(\land I) \; 5, 10$
\end{enumerate}

\end{document}
