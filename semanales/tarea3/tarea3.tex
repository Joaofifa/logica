\documentclass[letterpaper,11pt]{article}

% Soporte para los acentos.
\usepackage[utf8]{inputenc}
\usepackage[T1]{fontenc}  
% Idioma español.
\usepackage[spanish,mexico, es-tabla]{babel}
% Soporte de símbolos adicionales (matemáticas)
\usepackage{multirow}
\usepackage{amsmath}		
\usepackage{amssymb}		
\usepackage{amsthm}
\usepackage{amsfonts}
\usepackage{latexsym}
\usepackage{enumerate}
\usepackage{ragged2e}
% Código
\usepackage{listings}
% Modificamos los márgenes del documento.
\usepackage[lmargin=2cm,rmargin=2cm,top=2cm,bottom=2cm]{geometry}

\title{Lógica Computacional \\ Tarea Semanal 3}
\author{Rubí Rojas Tania Michelle}
\date{\today}

\begin{document}
    \maketitle

    \begin{enumerate}

        % Ejercicio 1.
        \item Da la especificación formal del siguiente argumento, definiendo
        previamente un glosario adecuado. \\
        \textit{Los alumnos de la Facultad de Ciencias y los programadores 
        sólo alcanzan su máximo nivel cuando la luna es azul.} \\  \\
        \textsc{Solución:} El universo de discurso serán todas las personas,
        los programadores y los satélites naturales. Los predicados que 
        utilizaremos serán los siguientes:
        \begin{itemize}
            \item $E(x):$ $x$ es alumno de la Facultad de Ciencias.
            \item $P(x):$ $x$ es programador.
            \item $M(x):$ $x$ alcanza su máximo nivel.
            \item $A(x):$ $x$ es azul.
        \end{itemize}

        Además, utilizaremos un término:
        \begin{itemize}
            \item l: Luna
        \end{itemize}

        Por lo que el argumento queda de la siguiente forma:
        \begin{center}
            $\forall x(A(l) \rightarrow (E(x) \land P(x) \land 
            M(x)))$
        \end{center}

        % Ejercicio 2.
        \item Considera las siguientes expresiones:
        \begin{itemize}
            \item[(a)] $\forall x \forall u$ $(M(a,x) \rightarrow M(a,u))$
            \item[(b)] $\forall xu$ $M(a,x) \rightarrow M(a,u)$
            \item[(c)] $\forall xu$ $(M(a,u) \rightarrow M(a,x))$  

            \begin{enumerate}
                % Ejercicio 2.1
                \item[2.1] Indica qué relaciones se cumplen de $\alpha-$
                equivalencia entre los argumentos. \\
                $((a) \sim_{\alpha} (b), (a) \sim_{\alpha} (c)$ ó 
                $(b) \sim_{\alpha} (c))$. Justifica adecuadamente. \\ \\
                \textit{Solución:} Notemos que $(b)$ es equivalente a escribir 
                $(\forall x \forall u M(a,x)) \rightarrow M(a,u)$ y que $(c)$
                es equivalente a escribir $\forall x \forall u (M(a,u) 
                \rightarrow M(a,x))$. Entonces
                \begin{itemize}
                    % Caso 1.
                    \item $(a) \sim_{\alpha} (b)$ \\
                    Este caso no se cumple. Como el alcance de los 
                    cuantificadores de $(a)$ es diferente al alcance de los
                    cuantificadores de $(b)$ entonces no tienen la misma 
                    'estructura', por lo que $(a)$ y $(b)$ no difieren a lo
                    más en los nombres de sus variables ligadas. Así, 
                    $(a) \not \sim_{\alpha} (b)$.

                    % Caso 2.
                    \item $(a) \sim_{\alpha} (c)$ \\
                    Este caso sí se cumple. Como $(a)$ y $(b)$ difieren a lo 
                    más en los nombres de sus variables ligadas (es decir,
                    tienen la misma 'estructura' y sólo sus variables 
                    ligadas cambian) entonces $(a) \sim_{\alpha} (c)$.

                    % Caso 3.
                    \item $(b) \sim_{\alpha} (c)$ \\
                    Este caso no se cumple.  Como el alcance de los 
                    cuantificadores de $(b)$ es diferente al alcance de los
                    cuantificadores de $(c)$ entonces no tienen la misma 
                    'estructura', por lo que $(b)$ y $(c)$ no difieren a lo
                    más en los nombres de sus variables ligadas. Así, 
                    $(b) \not \sim_{\alpha} (c)$.
                \end{itemize}

                % Ejercicio 2.2
                \item[2.2] Aplica ls siguiente sustitución 
                $\sigma = [x, u := L(a,n,d,r,o),$ $C(i,z,o)]$ a cada una 
                de las expresiones. \\ \\
                \textit{Solución:}
                \begin{itemize}
                    % Primera sustitución.
                    \item[(a)] $\forall x \forall u (M(a,x)$ 
                    $\rightarrow M(a,u))$ \\
                    Aplicando la sustitución obtenemos 
                    \begin{align*}
                        (\forall x \forall u ((M(a,x) 
                        \rightarrow M(a,u)))) \sigma
                        &= \forall x \forall u ((M(a,x) 
                        \rightarrow M(a,u)) \sigma_{xu}) \\
                        &= \forall x \forall u (M(a,x) \sigma_{xu} 
                        \rightarrow M(a,u) \sigma_{xu}) \\
                        &= \forall x \forall u (M(a,x) \rightarrow M(a,u))
                    \end{align*}

                    Notemos que en este ejercicio no cambió nada con la 
                    sustitución ya que todas las variables a sustituir son 
                    ligadas. 

                    % Segunda sustitución.
                    \item[(b)] $\forall xu$ $M(a,x) \rightarrow M(a,u)$ \\
                    Notemos que esta expresión es equivalente a escribir 
                    $(\forall x \forall u \; M(a,x)) \rightarrow M(a,u)$. Así,
                    aplicando la sustitución obtenemos 
                    \begin{align*}
                        (\forall x \forall u \; M(a,x)) 
                        \rightarrow (M(a,u)) \sigma
                        &= (\forall x \forall u \; M(a,x))
                        \rightarrow M(a, \; C(i,z,o)) 
                    \end{align*}

                    % Tercera sustitución.
                    \item[(c)] $\forall xu$ $(M(a,u) \rightarrow M(a,x))$ \\
                    Notemos que esta expresión es equivalente a escribir 
                    $\forall x \forall u (M(a,u) \rightarrow M(a,x))$. Así, 
                    aplicando la sustitución obtenemos   
                    \begin{align*}
                        (\forall x \forall u (M(a,u) 
                        \rightarrow M(a,x))) \sigma
                        &= \forall x \forall u ((M(a,u) 
                        \rightarrow M(a,x)) \sigma_{xu}) \\
                        &= \forall x \forall u (M(a,u) \sigma_{xu} 
                        \rightarrow M(a,x) \sigma_{xu}) \\
                        &= \forall x \forall u (M(a,u) \rightarrow M(a,x))
                    \end{align*}

                    Notemos que en este ejercicio no cambió nada con la 
                    sustitución ya que todas las variables a sustituir son 
                    ligadas. 

                \end{itemize}

            \end{enumerate}

        \end{itemize}
        
    \end{enumerate}

\end{document}