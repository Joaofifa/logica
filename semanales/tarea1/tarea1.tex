\documentclass[letterpaper,12pt]{article}

% Soporte para los acentos.
\usepackage[utf8]{inputenc}
\usepackage[T1]{fontenc}    
% Idioma español.
\usepackage[spanish,mexico, es-tabla]{babel}
% Soporte de símbolos adicionales (matemáticas)
\usepackage{multirow}
\usepackage{amsmath}		
\usepackage{amssymb}		
\usepackage{amsthm}
\usepackage{amsfonts}
\usepackage{latexsym}
\usepackage{enumerate}
\usepackage{ragged2e}
% Modificamos los márgenes del documento.
\usepackage[lmargin=2cm,rmargin=2cm,top=2cm,bottom=2cm]{geometry}

\title{Lógica Computacional \\ Tarea Semanal 1}
\author{Rubí Rojas Tania Michelle}
\date{21 de febrero de 2019}

\begin{document}
\maketitle

\begin{enumerate}
    
    % Ejercicio 1.
    \item Define recursivamente la función $icd$ que se especifica como sigue:
    $icd$ toma como entrada una fórmula $\phi$ y devuelve la fórmula resultante
    al intercambiar en $\phi$ todas las conjunciones por disyunciones, y las 
    disyunciones por conjunciones, respectivamente. Por ejemplo, se debe 
    cumplir que:
    \begin{center}
        $icd (p \land (q \lor \neg r) \rightarrow \neg (r \lor s) \land t) = $
        $p \lor (q \land \neg r) \rightarrow \neg (r \land s) \lor t$
    \end{center}

    % Solución del ejercicio 1.
    \textit{Solución:} Definimos recursivamente la función 
    $icd :: PROP \rightarrow PROP$ de la siguiente forma:
    \begin{itemize}
        \item $icd (\top) = \top$.
        \item $icd (\bot) = \bot$.
        \item $icd (Var P) = Var P$. 
        \item $icd (\neg \phi) = \neg icd(\phi)$. 
        \item $icd (\phi \land \psi) = icd(\phi) \lor icd(\psi)$. 
        \item $icd (\phi \lor \psi) = icd(\phi) \land icd(\psi)$. 
        \item $icd (\phi \rightarrow \psi) = icd(\phi) \rightarrow icd(\psi)$. 
    \end{itemize}

    % Ejercicio 2.
    \item Verifica tu definición mostrando paso a paso el cálculo del ejemplo 
    de arriba.

    % Solución del ejercicio 2.
    \textit{Solución: }
    \begin{align*}
        icd(p \land (q \lor \neg r) \rightarrow \neg (r \lor s) \land t)
        &= icd(p \land (q\lor\neg r)) \rightarrow icd(\neg (r\lor s) \land t)\\
        &= icd(p) \lor icd(q\lor\neg r) \rightarrow icd( \neg(r\lor s)) \lor icd(t)\\
        &= p \lor (icd(q) \land icd(\neg r)) \rightarrow \neg icd(r \lor s) \lor t\\
        &= p \lor (q \land \neg icd(r)) \rightarrow \neg (icd(r) \land icd(s)) \lor t\\
        &= p \lor (q  \land \neg r) \rightarrow \neg (r \land s) \lor t
    \end{align*}

    \newpage
    % Ejercicio 3.
    \item Define la función $atom(\phi)$ que, para $\phi \in PL$, devuelve el 
    número de fórmulas atómicas $(\top, \bot$ o variables$)$ en $\phi$.
    
    % Solución del ejercicio 3.
    \textit{Solución:} Definimos la función $atom :: PL \rightarrow \mathbb{N}$
    de la siguiente forma:
    \begin{itemize}
        \item $atom(\top) = 1$.
        \item $atom(\bot) = 1$.
        \item $atom(Var P) = 1$.
        \item $atom(\neg \phi) = atom(\phi)$. 
        \item $atom(\phi \star \psi) = atom(\phi) + atom(\psi)$.
    \end{itemize}

    % Ejercicio 4.
    \item Demuestra que para cualquier fórmula $\phi \in PL$ se cumple que:
    \begin{center}
        $atom(\phi) \leq con(\phi) + 1$
    \end{center}
    donde $con(\phi$) es la función que devuelve el número de conectivos de 
    $\phi$.

    % Demostración del ejercicio 4.
    \begin{proof}
        Inducción sobre la fórmula $\phi$. \\ \\
        \textbf{Base de inducción:} $\phi$ es atómica (no tiene operadores). 
        \begin{itemize}
            \item $\phi = \top$. Entonces 
            $atom(\top)= 1 = 0 + 1 = con(\top) + 1$
            \item $\phi = \bot$. Entonces 
            $atom(\bot)= 1 = 0 + 1 = con(\bot) + 1$
            \item $\phi = Var P$. Entonces 
            $atom(Var P) = 1 = 0 + 1 = con(Var P) + 1$
        \end{itemize}

        \textbf{Hipótesis de inducción:} Supongamos que 
        \begin{center}
            $atom(\phi') \leq con(\phi') + 1$ y $atom(\phi'') \leq con(\phi'') + 1$
        \end{center}

        \textbf{Paso inductivo:} Probamos la propiedad para dos casos:
        \begin{itemize}
            \item[i)] $\phi = \neg \phi'$. Sabemos que  
            $atom(\phi) = atom(\phi')$ y $con(\phi) = con(\phi')$. Entonces, 
            \begin{align*}
                atom(\phi) &= atom(\neg \phi')
                           && \text{def. de $\phi$ en el caso $i$)} \\
                           &= atom(\phi')
                           && \text{def. recursiva de $\phi$} \\
                           &\leq con(\phi') + 1
                           && \text{hipótesis de inducción} \\
                           &\leq con(\phi) + 1
                           && \text{ya que $con(\phi) = con(\phi')$}
            \end{align*} 

            \item[ii)] $\phi = (\varphi$ $\star$ $\psi)$. Entonces tenemos que 
            \begin{align*}
                atom(\phi) &= atom(\varphi \star \psi) 
                           && \text{def. de $\phi$ en el caso $ii)$} \\ 
                           &= atom(\varphi) + atom(\psi)
                           && \text{def. recursiva de $atom$} \\
                           &\leq (con(\varphi) + 1) + (con(\psi) + 1)
                           && \text{hipótesis de inducción} \\
                           &\leq (con(\varphi) + con(\psi) + 1) + 1
                           && \text{reagrupando} \\
                           &\leq con (\varphi \star \psi) + 1
                           && \text{def. recursiva de $con$} \\
                           &\leq con(\phi) + 1
                           && \text{def. de $\phi$ en el caso $ii)$}
            \end{align*}

        \end{itemize} 
        
    \end{proof}

\end{enumerate}

\end{document}
