\documentclass[letterpaper,12pt]{article}

% Soporte para los acentos.
\usepackage[utf8]{inputenc}
\usepackage[T1]{fontenc}    
% Idioma español.
\usepackage[spanish,mexico, es-tabla]{babel}
% Soporte de símbolos adicionales (matemáticas)
\usepackage{multirow}
\usepackage{amsmath}		
\usepackage{amssymb}		
\usepackage{amsthm}
\usepackage{amsfonts}
\usepackage{latexsym}
\usepackage{enumerate}
\usepackage{ragged2e}
% Código
\usepackage{listings}
%Tableaux
\usepackage{prooftrees}
% Modificamos los márgenes del documento.
\usepackage[lmargin=2cm,rmargin=2cm,top=2cm,bottom=2cm]{geometry}
% Datos del documento.
\title{Reposiciones de ejercicios semanales}
\author{Rubí Rojas Tania Michelle \\
        Universidad Nacional Autónoma de México \\
        taniarubi@ciencias.unam.mx \\
        $\#$ de cuenta: 315121719}
\date{29 de mayo de 2019}

\begin{document}
\maketitle

\begin{enumerate}
    % Semanal 1.
    \item \textbf{Semanal 1}
    
    \begin{enumerate}
        % Ejercicio 1.1
        \item Defina recursivamente la función \textbf{nn} especificada como
        sigue: \\
        Dada una fórmula $\varphi$, \textbf{nn}$(\varphi)$ devuelve el número de
        símbolos de negación en la fórmula. 

        \text{Solución:} Definimos la función $nn :: PROP \rightarrow \mathbb{N}$
        de la siguiente forma:
        
        \begin{align*}
            nn(\top) &= 0 \\
            nn(\bot) &= 0 \\
            nn(Var P) &= 0 \\
            nn(\neg \varphi) &= nn(\varphi) + 1\\
            nn(\varphi \star \psi) &= nn(\varphi) + nn(\psi) + 0
        \end{align*}
        
        % Ejercicio 1.2
        \item Demuestre utilizando inducción estructural que para cualquier 
        fórmula $\varphi$ se cumple 

        \begin{center}
            \textbf{nn($\varphi$) $\leq$ nn(qi($\varphi)$)}. 
        \end{center}

        Donde la función $qi$ devuelve una fórmula lógicamente equivalente en 
        la que no figura el símbolo de implicación.

        \begin{proof}
            Inducción sobre la fórmula $\varphi$.
            
            \begin{itemize}
                % Base de inducción.
                \item \textit{Base de inducción.}
                
                \begin{itemize}
                    \item $\varphi = \top$. Entonces 
                    $nn(\top) = 0 = nn(\top) = nn(qi(\top))$
                    \item $\varphi = \bot$. Entonces 
                    $nn(\bot) = 0 = nn(\bot) = nn(qi(\bot))$
                    \item $\varphi = Var P$. Entonces
                    $nn(Var P) = 0 = nn(Var P) = nn(qi(Var P))$  
                \end{itemize}

                % Hipótesis de inducción.
                \item \textit{Hipótesis de inducción.} \\
                Supongamos que se cumple para $nn(\varphi ') \leq nn(qi(\varphi '))$
                y $nn(\varphi '') \leq nn(qi(\varphi ''))$.

                \newpage
                % Paso inductivo.
                \item \textit{Paso inductivo.} \\
                Tenemos dos casos:
                
                \begin{enumerate}
                    \item $\varphi = \neg \varphi '$. Entonces 
                    
                    \begin{align*}
                        nn(\varphi) &= nn(\neg \varphi ')
                        && \text{def. de $\varphi$} \\
                        &= nn(\varphi ') + 1
                        && \text{def. recursiva de nn} \\
                        &\leq nn(qi(\varphi ')) + 1
                        && \text{H.I.} \\
                        &\leq nn(\neg (qi(\varphi '))) 
                        && \text{def. recursiva de nn} \\
                        &\leq nn(qi(\neg \varphi '))
                        && \text{def. recursiva de qi} \\
                        &\leq nn(qi(\varphi))
                        && \text{def. de $\varphi$}
                    \end{align*}
                    
                    \item $\varphi = \phi \star \psi$. Tenemos tres subcasos:
                    
                    \begin{itemize}
                        \item[i)] $\varphi = \phi \star \psi$ con 
                        $\star \in \{\land, \lor \}$. Entonces 
                        
                        \begin{align*}
                            nn(\varphi) 
                            &= nn(\phi \star \psi)
                            && \text{def. de $\varphi$} \\
                            &= nn(\phi) + nn(\psi) + 0
                            && \text{def. recursiva de nn} \\
                            &\leq nn(qi(\phi)) + nn(qi(\psi)) + 0
                            && \text{H.I.} \\
                            &\leq nn(qi(\phi) \star qi(\psi))
                            && \text{def. recursiva de nn} \\
                            &\leq nn(qi(\phi \star \psi))
                            && \text{def. recursiva de qi} \\
                            &\leq nn(qi(\varphi)) 
                            && \text{def. de $\varphi$}
                        \end{align*}

                        \item[ii)] $\varphi = \phi \star \psi$ con 
                        $\star \in \{ \rightarrow\}$. Entonces 
                        
                        \begin{align*}
                            nn(\varphi) 
                            &= nn(\phi \rightarrow \psi)
                            && \text{def. de $\varphi$} \\
                            &= nn(\phi) + nn(\psi) + 0
                            && \text{def. recursiva de nn} \\
                            &\leq nn(qi(\phi)) + nn(qi(\psi)) + 0
                            && \text{H.I.} \\
                            &\leq nn(qi(\phi) \rightarrow qi(\psi))
                            && \text{def. recursiva de nn} \\
                            &\leq nn(\neg qi(\phi) \lor qi(\psi))
                            && \text{equivalencia lógica} \\
                            &\leq nn(qi(\phi \rightarrow \psi))
                            && \text{def. recursiva de qi} \\
                            &\leq nn(qi(\varphi))
                            && \text{def. de $\varphi$}
                        \end{align*}

                        \item[iii)] $\varphi = \phi \star \psi$ con 
                        $\star \in \{ \leftrightarrow\}$. Entonces 
                        
                        \begin{align*}
                            nn(\varphi) 
                            &= nn(\phi \leftrightarrow \psi)
                            && \text{def. de $\varphi$} \\
                            &= nn(\phi) + nn(\psi) + 0
                            && \text{def. recursiva de nn} \\
                            &\leq nn(qi(\phi)) + nn(qi(\psi)) + 0
                            && \text{H.I.} \\
                            &\leq nn(qi(\phi) \leftrightarrow qi(\psi))
                            && \text{def. recursiva de nn} \\
                            &\leq nn((\neg qi(\phi) \lor qi(\psi)) \land 
                            (qi(\phi) \lor \neg qi(\psi)))
                            && \text{equivalencia lógica} \\
                            &\leq nn(qi(\phi \leftrightarrow \psi))
                            && \text{def. recursiva de qi} \\
                            &\leq nn(qi(\varphi))
                            && \text{def. de $\varphi$}
                        \end{align*}
                    \end{itemize}
                \end{enumerate}
            \end{itemize}
        \end{proof}
    \end{enumerate}

    \newpage
    % Semanal 2.
    \item \textbf{Semanal 2}
    
    \begin{enumerate}
        % Ejercicio 2.1
        \item Sea 
        $\varphi = \neg(q \land ((r \rightarrow \neg s \lor r) \rightarrow p))$.
        Convierta a $\varphi$ en una fórmula lógicamente equivalente $\varphi '$
        que se encuentre en Forma Normal Negativa. \\
        \textsc{Solución:}
        
        \begin{align*}
            fnn(\varphi) 
            &= \neg(q \land ((\neg r \lor \neg s \lor r) \rightarrow p)) \\
            &= \neg(q \land ((r \land s \land \neg r) \lor p)) \\
            &= \neg q \lor ((\neg r \lor \neg s \lor r) \land \neg p)
        \end{align*}

        % Ejercicio 2.2
        \item Sea 
        $\Gamma = \{ (a \lor b) \land c, \neg b \lor \neg c\}$ y $\varphi = a$.
        Determine mediante el método de tableaux si $\Gamma \models \varphi$.
        
        \begin{proof}
            Para mostrar que $\Gamma \models \varphi$ entonces hay que 
            trabajar nuestro Tableaux con $\Gamma \cup \{\neg \varphi \}$. 
            Así, \\ 

            \centering
            \begin{prooftree}{}
            [(a \lor b) \land c, checked, just={Hip}
                [\neg b \lor \neg c, checked, just={Hip}
                    [\neg a, checked, just={Hip}
                        [c, checked, just={ext. de $\alpha$ en 1}
                            [a \lor b, checked, just={ext. de $\alpha$ en 1}
                                [a, checked, close={3,6}, just={ext. de $\beta$ en 5}]
                                    [b, checked
                                        [\neg b, checked, close={6,7}, just={ext. de $\beta$ en 2}]
                                            [\neg c, checked, close={4,7}]]]]]]]
            \end{prooftree}

            \justify
            Como todas las ramas se cerraron entonces podemos concluir que 
            $\Gamma \models \varphi$.
        \end{proof}
    \end{enumerate}

    % Semanal 3.
    \item \textbf{Semanal 3}
    
    \begin{enumerate}
        % Ejercicio 3.1
        \item Da la especificación formal del siguiente argumento, definiendo
        previamente un glosario adecuado.
        
        \begin{center}
            \textit{Todos los estudiantes cursan al menos una materia}
        \end{center}

        \textsc{Solución:} 
        
        \begin{itemize}
            \item Universo del discurso: Todas las personas.
            \item Predicados: $E(x) :$ $x$ es estudiante y 
            $C(x,y) :$ $x$ cursa la materia $y$.
            \item Especificación Formal: 
            $\forall x (E(x) \rightarrow \exists y C(x,y))$
        \end{itemize}

        \newpage
        % Ejercicio 3.2
        \item Considere la siguiente expresión.

        \begin{center}
            $\forall x \exists y (A(y,x) \rightarrow M(x,y) \land 
            (\exists z A(x,z) \land M(z,x)))$
        \end{center}

        Aplique la siguiente sustitución: $\sigma = [u := a][z := x][x := n]$.\\ \\
        \text{Solución:} Primero aplicamos $\alpha-$equivalencia, donde 
        obtenemos 

        \begin{center}
            $\forall w \exists s (A(s,w) \rightarrow M(w,s) \land 
            (\exists r A(w,r) \land M(z,w)))$
        \end{center}

        Así, al aplicar la sustitución $\sigma$ tenemos que \\
        $(\forall w \exists s (A(s,w) \rightarrow M(w,s) \land 
        (\exists r A(w,r) \land M(z,w))))[u := a][z := x][x := n]$ \\  
        $= (\forall w \exists s (A(s,w) \rightarrow M(w,s) \land 
        (\exists r A(w,r) \land M(z,w))))[z := x][x := n]$ \\
        $ = (\forall w \exists s (A(s,w) \rightarrow M(w,s) \land 
        (\exists r A(w,r) \land M(x,w))))[x := n]$ \\
        $ = (\forall w \exists s (A(s,w) \rightarrow M(w,s) \land 
        (\exists r A(w,r) \land M(n,w))))$ 
    \end{enumerate}

    % Semanal 4.
    \item \textbf{Semanal 4}
    
    \begin{enumerate}
        % Ejercicio 4.1
        \item $\Gamma = \{ \forall x (Qy \rightarrow Px)\}$. utilizando Tableaux
        demuestre lo siguiente: 

        \begin{center}
            $\Gamma \models Qy \rightarrow \forall x Px$
        \end{center}

        \begin{proof}
            Para mostrar que $\Gamma \models Qy \rightarrow \forall x Px$ 
            entonces hay que trabajar nuestro Tableaux con 
            $\Gamma \cup \{Qy \land \exists x \neg Px\}$. Así, \\ 

            \centering
            \begin{prooftree}{}
            [\forall x (Qy \rightarrow Px), just={Hip}
                [Qy \land \exists x \neg Px, checked, just={Hip}
                    [Qy, checked, just={ext. de $\alpha$ en 2}
                        [\exists x \neg Px, checked, just={ext. de $\alpha$ en 2}
                            [\neg Pa, checked, just={ext. de $\delta$ en 4}
                                [Qy \rightarrow Pa, checked, just={ext. de $\gamma$ en 1}
                                    [\neg Qy, checked, close={3,7}, just={ext. de $\beta$ en 6}]
                                        [Pa, checked, close={5,7}]]]]]]]
            \end{prooftree}

            \justify
            Como todas las ramas se cerraron entonces podemos concluir que 
            $\Gamma \models Qy \rightarrow \forall x Px$.

        \end{proof}
    \end{enumerate}

    % Semanal 5.
    \item \textbf{Semanal 5}
    
    \begin{enumerate}
        % Ejercicio 5.1
        \item Sea 
        $\varphi = \forall x \exists y (Pxyz \rightarrow (Qz \lor Ryx)) 
        \rightarrow Qy \land (\exists x \forall z Rxz \lor \exists w Sx)$.
        Obten la Forma Normal Clausular de $\varphi$. \\
        \textsc{Solución:} Primero, rectificamos a $\varphi$.
        
        \begin{align*}
            rect(\varphi) &=
            \forall m \exists n (Pmnz \rightarrow (Qz \lor Rnm)) 
            \rightarrow Qy \land (\exists s \forall u Rsu \lor Sx)
        \end{align*}

        \newpage
        Así, 
        
        \begin{align*}
            fnp(\varphi) &=
            \exists s \forall u (\forall m \exists n (Pmnz \rightarrow (Qz \lor Rnm)) 
            \rightarrow Qy \land (Rsu \lor Sx)) 
            && \text{eq. lógica} \\
            &= 
            \exists s \forall u \exists m \forall n ((Pmnz \rightarrow 
            (Qz \lor Rnm)) \rightarrow Qy \land (Rsu \lor Sx)) 
            && \text{eq. lógica}
        \end{align*}

        Entonces,
        
        \begin{align*}
            fns(\varphi) &=
            \forall u \forall n ((Pfunz \rightarrow 
            (Qz \lor Rnfu)) \rightarrow Qy \land (Rau \lor Sx)) 
        \end{align*}
        
        Luego, 
        
        \begin{align*}
            fnn(\varphi) &=
            \forall u \forall n ((Pfunz \land (\neg Qz \land \neg Rnfu)) 
            \lor Qy \land (Rau \lor Sx)) 
        \end{align*}

        Por lo tanto, 
        
        \begin{center}
            $Cl(\varphi) =
            Pfunz \land \neg Qz \land (\neg Rnfu \lor Qy) \land (Rau \lor Sx)$
        \end{center}    
    \end{enumerate}

    % Semanal 6.
    \item \textbf{Semanal 6}
    
    \begin{enumerate}
        % Ejercicio 6.1
        \item Transforme a Forma Normal Clausular y decida mediante resolución
        binaria si se cumple la siguiente consecuencia lógica.

        \begin{center}
            $\{\forall x (Pxy \rightarrow \exists y Qy), 
            \exists x \forall y (Qy \rightarrow Pyz \lor Rx),
            \forall y (Ry \rightarrow \exists x \neg Qa) \} 
            \models \forall x (Qfa \rightarrow Qa)$
        \end{center}

        \textsc{Solución:} Ya que trabajaremos con resolución binaria entonces 
        debemos trabajar con el conjunto

        \begin{center}
            $\varphi = \{\forall x (Pxy \rightarrow \exists y Qy), 
            \exists x \forall y (Qy \rightarrow Pyz \lor Rx),
            \forall y (Ry \rightarrow \exists x \neg Qa),
            \exists x (Qfa \land \neg Qa)\}$
        \end{center}

        Para obtener la Forma Normal Clausular, primero
        rectificamos $\varphi$.
        
        \begin{align*}
            rect(\varphi) &= 
            \forall m (Pmy \rightarrow \exists n Qn) \land 
            \exists s \forall r (Qr \rightarrow Prz \lor Rs) \land 
            \forall u (Ru \rightarrow \neg Qa) \land Qfa \land \neg Qa
        \end{align*}

        Así,
        
        \begin{align*}
            fnn(\varphi) &=
            \forall m (\neg Pmy \lor \exists n Qn) \land 
            \exists s \forall r (\neg Qr \lor Prz \lor Rs) \land 
            \forall u (\neg Ru \lor \neg Qa) \land Qfa \land \neg Qa 
        \end{align*}

        Después, 
        
        \begin{align*}
            fnp(\varphi) &=
            \forall m \exists n \exists s \forall r \forall u
            ((\neg Pmy \lor Qn) \land (\neg Qr \lor Prz \lor Rs) \land 
            (\neg Ru \lor \neg Qa) \land Qfa \land \neg Qa)
        \end{align*}

        \newpage
        Luego, 
        
        \begin{align*}
            fns(\varphi) &= 
            \forall m \exists s \forall r \forall u
            ((\neg Pmy \lor Qfm) \land (\neg Qr \lor Prz \lor Rs) \land 
            (\neg Ru \lor \neg Qa) \land Qfa \land \neg Qa) \\
            &= 
            \forall m \forall r \forall u
            ((\neg Pmy \lor Qfm) \land (\neg Qr \lor Prz \lor Rgm) \land 
            (\neg Ru \lor \neg Qa) \land Qfa \land \neg Qa)
        \end{align*}

        Finalmente, 
        
        \begin{align*}
            Cl(\varphi) &=
            (\neg Pmy \lor Qfm) \land (\neg Qr \lor Prz \lor Rgm) \land 
            (\neg Ru \lor \neg Qa) \land Qfa \land \neg Qa
        \end{align*}

        Haciendo resolución binaria tenemos que 
        
        \begin{align*}
            1 &. (\neg Pmy \lor Qfm) 
            && \text{Hip} \\
            2 &. (\neg Qr \lor Prz \lor Rgm)
            && \text{Hip} \\
            3 &. (\neg Ru \lor \neg Qa)
            && \text{Hip} \\ 
            4 &. \; Qfa 
            && \text{Hip} \\
            5 &. \; \neg Qa
            && \text{Hip} \\
            6 &. \; \square
            && \text{$Res(4,5) [a := b][b := fa]$}
        \end{align*}

        Por lo tanto, podemos concluir que 
        
        \begin{center}
            $\varphi = \{\forall x (Pxy \rightarrow \exists y Qy), 
            \exists x \forall y (Qy \rightarrow Pyz \lor Rx),
            \forall y (Ry \rightarrow \exists x \neg Qa),
            \exists x (Qfa \land \neg Qa)\}$
        \end{center}
    \end{enumerate}

    % Semanal 7.
    \item \textbf{Semanal 7}
    
    \begin{enumerate}
        % Ejercicio 7.1
        \item Demuestre lo siguiente mediante deducción natural.
        
        \begin{center}
            $\exists x Fx \rightarrow \forall y (Gy \rightarrow Hy),
            \exists z Jz \rightarrow \exists w Gw 
            \vdash \exists z (Fz \land Jz) \rightarrow \exists v Hv$
        \end{center}
        % Demostración.
        \begin{proof}
            Por el Teo. de DN basta mostrar

            \begin{center}
                $\Gamma = \{ \exists x Fx \rightarrow 
                \forall y(Gy \rightarrow Hy), \exists z Jz \rightarrow 
                \exists w Gw, \exists z(Fz \land Jz) \} \vdash\exists v Hv$
            \end{center}
            
            Entonces
            
            \begin{align*}
                1. \; \; &\Gamma \vdash \exists x Fx \rightarrow 
                \forall y(Gy \rightarrow Hy)
                && \text{Hip} \\
                2. \; \; &\Gamma \vdash \exists z Jz \rightarrow 
                \exists w Gw
                && \text{Hip} \\
                3. \; \; &\Gamma \vdash \exists z(Fz \land Jz)
                && \text{Hip} \\
                4. \; \; &\Gamma, Fz \land Jz \vdash Fz \land Jz
                && \text{Hip} \\
                5. \; \; &\Gamma, Fz \land Jz \vdash Fz 
                && \text{$(\land E) 4$} \\
                6. \; \; &\Gamma, Fz \land Jz \vdash Jz
                && \text{$(\land E) 4$} \\
                7. \; \; &\Gamma, Fz \land Jz \vdash \exists x Fx
                && \text{$(\exists I) 4$} \\
                8. \; \; &\Gamma, Fz \land Jz \vdash \forall y(Gy 
                \rightarrow Hy)
                && \text{$(\rightarrow E) 6,1$} \\
                9. \; \; &\Gamma, Fz \land Jz \vdash \exists z Jz
                && \text{$(\exists I) 5$} \\
                10. \; \; &\Gamma, Fz \land Jz \vdash \exists Gw
                && \text{$(\rightarrow E) 8, 2$} \\
                11. \; \;  &\Gamma, Fz \land Jz, Gw \vdash Gw 
                && \text{Hip} \\
                12. \; \; &\Gamma, Fz \land Jz, Gw \vdash Gw \rightarrow Hw
                && \text{$(\forall E) 7$} \\
                13. \; \; &\Gamma Fz \land Jz, Gw \vdash Hw 
                && \text{$(\rightarrow E) 11,10$} \\
                14. \; \; &\Gamma, Fz \land Jz, Gw \vdash \exists u Hu
                && \text{$(\exists I) 12$} \\
                15. \; \; &\Gamma, Fz \land Jz \vdash \exists u Hu
                && \text{$(\exists E) 9, 12$} \\
                16. \; \; &\Gamma \vdash \exists u Hu 
                && \text{$(\exists E) 3, 15$}
            \end{align*}

            Por lo tanto, podemos concluir que 

            \begin{center}
                $\exists x Fx \rightarrow \forall y (Gy \rightarrow Hy),
                \exists z Jz \rightarrow \exists w Gw 
                \vdash \exists z (Fz \land Jz) \rightarrow \exists v Hv$
            \end{center}
        \end{proof}
    \end{enumerate}
\end{enumerate}
\end{document}
