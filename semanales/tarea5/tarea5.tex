\documentclass[letterpaper,12pt]{article}

% Soporte para los acentos.
\usepackage[utf8]{inputenc}
\usepackage[T1]{fontenc}    
% Idioma español.
\usepackage[spanish,mexico, es-tabla]{babel}
% Soporte de símbolos adicionales (matemáticas)
\usepackage{multirow}
\usepackage{amsmath}		
\usepackage{amssymb}		
\usepackage{amsthm}
\usepackage{amsfonts}
\usepackage{latexsym}
\usepackage{enumerate}
\usepackage{ragged2e}
% Modificamos los márgenes del documento.
\usepackage[lmargin=2cm,rmargin=2cm,top=2cm,bottom=2cm]{geometry}

\title{Lógica Computacional \\ Tarea Semanal 5}
\author{Rubí Rojas Tania Michelle}
\date{\today}

\begin{document}
\maketitle

Sea $\varphi = \{ \forall x \forall y \forall z(Pxy \land Pyz
\rightarrow Rxz), $ $\forall x \exists y Pxy, $ $\neg \forall x Pxx\}$
\begin{enumerate}
    % Ejercicio 1.
    \item Obtener la Forma Normal Prenex de $\varphi$. \\
    \textit{Solución:} \\
    Reescribimos a $\varphi$ de la siguiente forma:
    \begin{center}
        $\varphi = $
        $\forall x \forall y \forall z(Pxy \land Pyz \rightarrow Rxz) \land 
        \forall x \exists y Pxy \land \neg \forall x Pxx$
    \end{center}

    Primero, rectificamos a $\varphi$. 
    \begin{align*}
        rec(\varphi) &= 
        \forall x \forall y \forall z(Pxy \land Pyz \rightarrow Rxz) \land 
        \forall u \exists v Puv \land \neg \forall w Pww
        && \text{$\alpha$-equivalencia}
    \end{align*}
    
    Ahora, usando $rec(\varphi)$ procedemos a encontrar $fnn(\varphi)$.
    \begin{align*}
        fnn(\varphi) 
        &= \forall x \forall y \forall z(\neg (Pxy \land Pyz) \lor Rxz) \land
        \forall u \exists v Puv \land \exists w \neg Pww
        && \text{eqv. lógicas} \\
        &= \forall x \forall y \forall z((\neg Pxy \lor \neg Pyz) \lor Rxz)
        \land \forall u \exists v Puv \land \exists w \neg Pww
        && \text{De Morgan}
    \end{align*}

    Finalmente, usando $fnn(\varphi)$,  procedemos a encontrar $fnp(\varphi)$.
    \begin{align*}
        fnp(\varphi)
        &= \forall x \forall y \forall z \forall u \exists v \exists w
        (\neg (Pxy \land Pyz) \lor Rxz \land Puv \land \neg Pww)
        && \text{eqv. lógicas}
    \end{align*}

    Por lo tanto, la Forma Normal Prenex de $\varphi$ es 
    \begin{center}
        $fnp(\varphi) = 
        \forall x \forall y \forall z \forall u \exists v \exists w
        (\neg (Pxy \land Pyz) \lor Rxz \land Puv \land \neg Pww)$
    \end{center}

    % Ejercicio 2.
    \item Obtener la Forma Normal de Skolem de $\varphi$. \\
    \textit{Solución:} \\
    Usando $fnp(\varphi)$, procedemos a encontrar $fns(\varphi)$.
    \begin{align*}
        fns(\varphi) 
        &= \forall x \forall y \forall z \forall u \exists w
        (\neg (Pxy \land Pyz) \lor Rxz \land Pufxyzu \land \neg Pww) \\
        &= \forall x \forall y \forall z \forall u
        (\neg (Pxy \land Pyz) \lor Rxz \land Pufxyzu \land \neg Pgxyzugxyzu)
    \end{align*}

    Por lo tanto, la Forma Normal de Skolem de $\varphi$ es 
    \begin{center}
        $¿\forall x \forall y \forall z \forall u
        (\neg (Pxy \land Pyz) \lor Rxz \land Pufxyzu \land \neg Pgxyzugxyzu)$
    \end{center}

\end{enumerate}

\end{document}
