\documentclass[letterpaper,11pt]{article}

% Soporte para los acentos.
\usepackage[utf8]{inputenc}
\usepackage[T1]{fontenc}    
% Idioma español.
\usepackage[spanish,mexico, es-tabla]{babel}
% Soporte de símbolos adicionales (matemáticas)
\usepackage{multirow}
\usepackage{amsmath}		
\usepackage{amssymb}		
\usepackage{amsthm}
\usepackage{amsfonts}
\usepackage{latexsym}
\usepackage{enumerate}
\usepackage{ragged2e}
% Código
\usepackage{listings}
%Tableaux
\usepackage{prooftrees}
% Modificamos los márgenes del documento.
\usepackage[lmargin=2cm,rmargin=2cm,top=2cm,bottom=2cm]{geometry}

\title{Lógica Computacional \\ Tarea Semanal 4}
\author{Rubí Rojas Tania Michelle}
\date{\today}

\begin{document}
\maketitle

\begin{enumerate}
    \item Sea $\Gamma = \{\forall x(P xy \rightarrow \exists y Qy), $
    $\exists x \forall y(Qy \rightarrow Pyx \lor Rx), $
    $\forall y(Ry \rightarrow \exists x \neg Qa)\}$.
    Utilizando Tableaux demuestra lo siguiente.
    \begin{center}
        $\Gamma \models \forall x (Qfa \rightarrow Qa)$
    \end{center}

    \begin{proof}
        Construimos el Tableaux para 

    \end{proof}
\end{enumerate}
    
\end{document}
