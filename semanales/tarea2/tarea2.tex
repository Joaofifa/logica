\documentclass[letterpaper,11pt]{article}

% Soporte para los acentos.
\usepackage[utf8]{inputenc}
\usepackage[T1]{fontenc}    
% Idioma español.
\usepackage[spanish,mexico, es-tabla]{babel}
% Soporte de símbolos adicionales (matemáticas)
\usepackage{multirow}
\usepackage{amsmath}		
\usepackage{amssymb}		
\usepackage{amsthm}
\usepackage{amsfonts}
\usepackage{latexsym}
\usepackage{enumerate}
\usepackage{ragged2e}
% Modificamos los márgenes del documento.
\usepackage[lmargin=2cm,rmargin=2cm,top=2cm,bottom=2cm]{geometry}

\title{Lógica Computacional \\ Tarea Semanal 2}
\author{Rubí Rojas Tania Michelle}
\date{28 de febrero de 2019}

\begin{document}
\maketitle

\begin{enumerate}
    
    % Pregunta 1.
    \item Sea $\varphi = ((p \rightarrow r) \land (q \rightarrow r)) 
    \rightarrow ((p \land q) \rightarrow r)$. Convierte a $\varphi$ en una
    fórmula equivalente $\varphi'$ que se encuentre en forma normal negativa.\\
    \textit{Solución:}
    \begin{align*}
        \varphi &= ((p \rightarrow r) \land (q \rightarrow r)) 
                   \rightarrow ((p \land q) \rightarrow r) 
                && \text{definición de $\varphi$} \\
                &\equiv \neg ((\neg p \lor r) \land (\neg q \lor r))
                   \lor (\neg (p \land q) \lor r)
                && \text{ya que $P \rightarrow Q \equiv \neg P \lor Q$} \\
                &\equiv (\neg (\neg p \lor r) \lor \neg (\neg q \lor r))
                   \lor ((\neg p \lor \neg q) \lor r)
                && \text{De Morgan} \\
                &\equiv ((\neg \neg p \land \neg r) \lor 
                   (\neg \neg q \land \neg r)) \lor ((\neg p \lor \neg q) \lor r)
                && \text{De Morgan} \\
                &\equiv ((p \land \neg r) \lor (q \land \neg r))
                   \lor ((\neg p \lor \neg q) \lor r)
                && \text{ya que $\neg \neg P \equiv P$}
    \end{align*}

    Como $\varphi' = ((p \land \neg r) \lor (q \land \neg r)) \lor 
    ((\neg p \lor \neg q) \lor r)$ no contiene equivalencias ni implicaciones y
    las negaciones que figuran en $\varphi$ afectan sólo a fórmulas atómicas, 
    entonces $\varphi'$ es una fórmula equivalente a $\varphi$ que se encuentra 
    en forma normal negativa.
    
    % Pregunta 2.
    \item Define recursivamente la función \textbf{isPermutation} que, dadas
    dos listas nos dice si una es permutación de la otra.\\
    \textit{Solución:} Definimos recursivamente la función 
    $isPermutation :: [a] -> [a] -> Bool$ de la siguiente manera:

    \begin{itemize}
        \item isPermutation $[] [] = True$
        \item isPermutation $(x:xs) (y:ys) = $
    \end{itemize}

    % Pregunta 3.
    \item Verifica tu definición aplicándola a las siguientes listas (debes
    mostrar paso a paso las llamadas recursivas) $A = [1,2,3,4],$ 
    $B = [1,2,3,4]$, osea, hacer la ejecución de \textbf{isPermutation A B}.\\
    \textit{Solución:}
    
    % Pregunta 4.
    \item Determina mediante el método de Tableaux si $\Gamma \models \varphi$
    donde $\Gamma = \{ \neg p \lor q$, $\neg (q \land \neg r)$, 
    $r \rightarrow s \}$ y $\varphi = \neg p \lor s$.\\
    \textit{Solución:}

\end{enumerate}
    
\end{document}
