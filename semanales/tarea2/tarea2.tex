\documentclass[letterpaper,11pt]{article}

% Soporte para los acentos.
\usepackage[utf8]{inputenc}
\usepackage[T1]{fontenc}    
% Idioma español.
\usepackage[spanish,mexico, es-tabla]{babel}
% Soporte de símbolos adicionales (matemáticas)
\usepackage{multirow}
\usepackage{amsmath}		
\usepackage{amssymb}		
\usepackage{amsthm}
\usepackage{amsfonts}
\usepackage{latexsym}
\usepackage{enumerate}
\usepackage{ragged2e}
% Código
\usepackage{listings}
%Tableaux
\usepackage{prooftrees}
% Modificamos los márgenes del documento.
\usepackage[lmargin=2cm,rmargin=2cm,top=2cm,bottom=2cm]{geometry}

\title{Lógica Computacional \\ Tarea Semanal 2}
\author{Rubí Rojas Tania Michelle}
\date{28 de febrero de 2019}

\begin{document}
\maketitle

\begin{enumerate}
   
   % Pregunta 1.
   \item Sea $\varphi = ((p \rightarrow r) \land (q \rightarrow r)) 
   \rightarrow ((p \land q) \rightarrow r)$. Convierte a $\varphi$ en una
   fórmula equivalente $\varphi'$ que se encuentre en forma normal negativa.\\
   \textit{Solución:}
   \begin{align*}
      \varphi &= ((p \rightarrow r) \land (q \rightarrow r)) 
                 \rightarrow ((p \land q) \rightarrow r) 
              && \text{definición de $\varphi$} \\
              &\equiv \neg ((\neg p \lor r) \land (\neg q \lor r))
                      \lor (\neg (p \land q) \lor r)
              && \text{ya que $P \rightarrow Q \equiv \neg P \lor Q$} \\
              &\equiv (\neg (\neg p \lor r) \lor \neg (\neg q \lor r))
                      \lor ((\neg p \lor \neg q) \lor r)
              && \text{De Morgan} \\
              &\equiv ((\neg \neg p \land \neg r) \lor 
                      (\neg \neg q \land \neg r)) \lor 
                      ((\neg p \lor \neg q) \lor r)
              && \text{De Morgan} \\
              &\equiv ((p \land \neg r) \lor (q \land \neg r))
                      \lor ((\neg p \lor \neg q) \lor r)
              && \text{ya que $\neg \neg P \equiv P$}
   \end{align*}

   Sea $\varphi' = ((p \land \neg r) \lor (q \land \neg r)) \lor 
   ((\neg p \lor \neg q) \lor r)$. Como $\varphi'$ no contiene equivalencias ni 
   implicaciones y las negaciones que figuran en $\varphi$ afectan sólo a 
   fórmulas atómicas, entonces $\varphi'$ es una fórmula equivalente a 
   $\varphi$ que se encuentra en forma normal negativa.
    
   % Pregunta 2.
   \item Define recursivamente la función \textbf{isPermutation} que, dadas
   dos listas nos dice si una es permutación de la otra.\\
   \textit{Solución:} 
   \begin{lstlisting}[language=Haskell]
      import Data.List

      isPermutation :: (Eq a) => [a] -> [a] -> Bool
      isPermutation [] [] = True
      isPermutation xs ys 
          | (length xs /= length ys) = False
          | elem (head xs) ys = 
              isPermutation (tail xs) (delete (head xs) ys)
          | otherwise = False
   \end{lstlisting}

   % Pregunta 3.
   \item Verifica tu definición aplicándola a las siguientes listas (debes
   mostrar paso a paso las llamadas recursivas) $A = [1,2,3,4],$ 
   $B = [1,2,3,4]$, osea, hacer la ejecución de \textbf{isPermutation A B}.\\
   \textit{Solución:}
   \begin{lstlisting}
      isPermutation A B = isPermutation [1,2,3,4] [1,2,3,4]
                        = isPermutation [2,3,4] [2,3,4]
                        = isPermutation [3,4] [3,4]
                        = isPermutation [4] [4]
                        = isPermutation [] []
                        = True
   \end{lstlisting}
    
   % Pregunta 4.
   \item Determina mediante el método de Tableaux si $\Gamma \models \varphi$
   donde $\Gamma = \{ \neg p \lor q$, $\neg (q \land \neg r)$, 
   $r \rightarrow s \}$ y $\varphi = \neg p \lor s$.\\
   \textit{Solución:} Primero, para $\Gamma$ eliminamos las implicaciones y
   simplificamos las fórmulas. 

   \begin{center}
      $\Gamma = \{ \neg p \lor q, \neg q \lor r, \neg r \lor s \}$
   \end{center}
   
   Ahora, para mostrar que $\Gamma \models \varphi$ entonces hay que trabajar
   nuestro Tableaux con $\Gamma \cup \{\neg \varphi \}$. Así,

   \begin{center}
      $\Gamma \cup \{\neg \varphi \} = 
      \{ \neg p \lor q, \neg q \lor r, \neg r \lor s \} 
      \cup \{p \land \neg s \}$
   \end{center}

   Finalmente, construimos el Tableaux para el conjunto
   $\Gamma \cup \{ \neg \varphi \}$: \\ 

   \centering
   \begin{prooftree}{}
      [\neg p \lor q, checked
         [\neg q \lor r, checked
            [\neg r \lor s, checked
               [p \land \neg s, checked
                  [p, checked, just={ext. de $\alpha$ en 4}
                     [\neg s, checked, just={ext. de $\alpha$ en 4}
                        [\neg r, checked, just={ext. de $\beta$ en 3}
                           [\neg q, checked, just={ext. de $\beta$ en 2}
                              [\neg p, checked, close={5,9}, just={ext. de $\beta$ en 1}]
                                 [q, checked, close={8,9}, just={ext. de $\beta$ en 1}]]
                                    [r, checked, close={7,8}, just={ext. de $\beta$ en 2}]]
                                       [s, checked, close={6,7}, just={ext. de $\beta$ en 3}]]]]]]]
   \end{prooftree}

   \justify
   Como el Tableaux se cierra, entonces la consecuencia lógica 
   $\Gamma \models \varphi$ se da y el argumento que representa es correcto.

\end{enumerate}
    
\end{document}
