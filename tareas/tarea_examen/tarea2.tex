\documentclass[letterpaper,11pt]{article}

% Soporte para los acentos.
\usepackage[utf8]{inputenc}
\usepackage[T1]{fontenc}    
% Idioma español.
\usepackage[spanish,mexico, es-tabla]{babel}
% Soporte de símbolos adicionales (matemáticas)
\usepackage{multirow}
\usepackage{amsmath}		
\usepackage{amssymb}		
\usepackage{amsthm}
\usepackage{amsfonts}
\usepackage{latexsym}
\usepackage{enumerate}
\usepackage{ragged2e}
\newtheorem{teo}{Definición}[]
% Código
\usepackage{listings}
%Tableaux
\usepackage{prooftrees}
% Modificamos los márgenes del documento.
\usepackage[lmargin=2cm,rmargin=2cm,top=2cm,bottom=2cm]{geometry}

\title{Lógica Computacional \\ Tarea-Examen Semanal}
\author{Rubí Rojas Tania Michelle \\
        Universidad Nacional Autónoma de México \\
        taniarubi@ciencias.unam.mx \\
        \# cuenta: 315121719
        }
\date{30 de abril de 2019}

\begin{document}
    \maketitle

    \begin{enumerate}
        % Ejercicio 1.
        \item Enuncia la sintaxis y semnántica de la lógica de predicados. \\
        \textsc{Solución:}

        \begin{itemize}
            % Sixtaxis.
            \item[a)] Sintaxis de la lógica de predicados. \\
            Dependiendo de la estructura semnántica que tengamos en mente será
            necesario agregar símbolos particulares para denotar objetos y
            relaciones entre objetos. De esta manera, el alfabeto consta de 
            dos partes ajenas entre sí, la parte común a todos los lenguajes
            determinada por los símbolos lógicos y auxiliares y la parte 
            particular, llamada tipo de semejanza o signatura del lenguaje.

            \begin{itemize}
                \item La parte común a todos los lenguajes consta de:
                
                \begin{itemize}
                    \item Un conjunto finito de variables 
                    \texttt{$Var \{x_{1}, ..., x_{n},...\}$}
                    \item Constantes lógicas: $\top, \bot$
                    \item Conectivos u operadores lógicos: 
                    $\neg, \land, \lor, \rightarrow, \leftrightarrow$
                    \item Cuantificadores: $\forall, \exists$
                    \item Símbolos auxiliares: $(, )$ y $,$ (coma).
                    \item Si se agrega el símbolo de igualdad $=$, decimos
                    que el lenguaje tiene igualdad.
                \end{itemize}

                \item La signatura de un lenguaje en particular está dada por:
                
                \begin{itemize}
                    \item Un conjunto $\mathcal{P}$, posiblemente vacío, de
                    símbolos o letras de predicado:
                    
                    \begin{center}
                        $P_{1}, ..., P_{n}, ...$
                    \end{center}

                    A cada símbolo se le asigna un índice o número de argumentos 
                    $m$, el cual se hace explícito escribiendo $P_{n}^{(m)}$ lo 
                    cual significará que el símbolo $P_{n}$ necesita de $m$ 
                    argumentos.
                    \item Un conjunto $\mathcal{F}$, posiblemente vacío, de
                    símbolos o letras de función:
                    
                    \begin{center}
                        $f_{1},..., f_{n},...$
                    \end{center}

                    Análogamente a los símbolos de predicado cada símbolo de 
                    función tiene un índice asignado, $f_{n}^{(m)}$ significará 
                    que el símbolo $f_{n}$ necesita de $m$ argumentos.       
                    \item Un conjunto $\mathcal{C}$, posiblemente vacío, de 
                    símbolos de constante:
                    
                    \begin{center}
                        $c_{1}, ..., c_{n}, ...$
                    \end{center}

                    En algunos libros los símbolos de constante se consideran 
                    como parte del conjunto de símbolos de función, puesto que 
                    pueden verse como funciones de índice cero, es decir, 
                    funciones que no reciben argumentos.
                \end{itemize}

                \newpage
                Dado que un lenguaje de primer órden queda determinado de manera única
                por su signatura, abusaremos de la notación y escribiremos
                
                \begin{center}
                    $\mathcal{L} = \mathcal{P} \cup \mathcal{F} \cup \mathcal{C}$
                \end{center}

                para denotar al lenguaje dado por tal signarura.
            \end{itemize}

            Los términos del lenguaje son variables, constantes y funciones aplicados
            a estos.
            
            \begin{teo}[\textbf{Términos}]
                Los términos son definidos como sigue:
                
                \begin{itemize}
                    \item Una variable es un término.
                    \item Si $c \in F$ es una función nula, entonces $c$ es un término.
                    \item Si $t_{1}, t_{2}, ..., t_{n}$ son términos y $f \in F$ tiene
                    índice $n > 0$, entonces $f(t_{1}, t_{2}, ..., t_{n})$ es un término.
                    \item Son todos.
                \end{itemize}
            \end{teo}

            En forma de Backus Naur se escribe:
            
            \begin{center}
                $t :: x$ $|$ $c$ $|$ $f(t_{1},...,t_{n})$
            \end{center}

            \begin{teo}
                Sean $P$ el conjunto de predicados y $F$ el conjunto de símbolos de 
                función. Se define el conjunto de fórmulas sobre $(F, P)$
                inductivamente usando la definición de Términos sobre F.
                
                \begin{itemize}
                    \item Si $p \in \mathcal{P}$ (símbolo de predicado) con 
                    índice $n \geq 1$ y si $t_{1}, t_{2}, ..., t_{n}$ son términos
                    sobre $\mathcal{F}$ entonces $P(t_{1}, t_{2},..., t_{n})$ es 
                    una fórmula. 
                    \item Si $\varphi$ es una fórmula, entonces $\neg \varphi$ 
                    $\varphi$ también lo es.
                    \item $\varphi, \psi$ son fórmulas, entonces 
                    $\varphi \star \psi$ también lo son; donde 
                    $\star \in \{ \land, \lor, \rightarrow, \leftrightarrow \}$ 
                    \item Si $\varphi$ es una fórmula y $x$ es una variable, 
                    entonces $\forall x \varphi$ es una fórmula.
                    \item Si $\varphi$ es una fórmula y $x$ es una variable,
                    entonces $\exists x \varphi$ es una fórmula.
                    \item No hay más.
                \end{itemize}
            \end{teo}

            Es decir, la sintaxis de la lógica de predicados en forma de Backus 
            Naur se define como:
            
            \begin{center}
                $\varphi :: P(t_{1}, ..., t_{n})$ $|$ $\neg \varphi$ $|$ 
                $\varphi \land \varphi$ $|$ $\varphi \lor \varphi$ $|$ 
                $\varphi \rightarrow \varphi$ $|$ $\varphi \leftrightarrow 
                \varphi$ $|$ $\forall x \varphi$ $|$ $\exists x \varphi$ $|$
                $t_{1} = t_{2}$ $|$ $\bot$ $|$ $\top$
            \end{center}

            % Semántica.
            \item[b)] Semántica de la lógica de predicados.
            
            \begin{teo}
                Sea $\mathcal{L} = \mathcal{P} \cup \mathcal{F} \cup 
                \mathcal{C}$ un lenguaje de primer órden. Una estructura o 
                interpretación para $\mathcal{L}$ es un par 
                $\mathcal{M} = \langle M, \mathcal{I} \rangle$ donde 
                $M \neq \varnothing$ es un conjunto no vacío llamado el 
                universo de la estructura e $\mathcal{I}$ es una función con
                dominio $\mathcal{L}$ tal que 
                
                \begin{itemize}
                    \item Si $P^{(n)} \in \mathcal{P}$ entonces 
                    $\mathcal{I}(P)$ es una relación de $m$-argumentos sobre 
                    $M$, es decir, $\mathcal{I}(\mathcal{P}) \subseteq M^{n}$.
                    Alternativamente podemos definir la interpretación de 
                    $\mathcal{P}$ como una función booleana que decide si una 
                    tupla está o no en la relación deseada, es decir, 
                    $\mathcal{I}(\mathcal{P}): M^{n} \rightarrow Bool$.

                    \item Si $f^{(n)} \in \mathcal{F}$ entonces 
                    $\mathcal{I}(f)$ es una función con dominio 
                    $M^{n}$ y contradominio $M$, es decir, 
                    $\mathcal{I}(f): M^{n} \rightarrow M$.
                    \item Si $c \in \mathcal{C}$ entonces $\mathcal{I}(c)$
                    es un elemento de $M$, es decir, $\mathcal{I}(c) \in M$
                \end{itemize}
                
                Dada una interpretación $\mathcal{M} = \langle M, 
                \mathcal{I} \rangle$, la siguiente notación es de utilidad:
                
                \begin{center}
                    $|\mathcal{M}| =_{def} M$ \\
                    $P^{\mathcal{I}} =_{def} \mathcal{I}(P)$ \\
                    $f^{\mathcal{I}} =_{def} \mathcal{I}(f)$ \\
                    $c^{\mathcal{I}} =_{def} \mathcal{I}(c)$
                \end{center}
            \end{teo}

            \newpage
            \begin{teo}[\textbf{Estado o Asignación}]
                Un estado, asignación o evaluación de las variables es una
                función $\sigma: Var \rightarrow M$
            \end{teo}

            \begin{teo}[\textbf{Interpretación de Términos}]
                Sea $\sigma$ un estado de las variables. Definimos la función
                de interpretación o significado de los términos con respecto a
                $\sigma, \mathcal{I_{\sigma}}: TERM \rightarrow |\mathcal{M}|$
                como sigue:
                
                \begin{center}
                    $\mathcal{I_{\sigma}}(x) = \sigma(x)$ \\
                    $\mathcal{I_{\sigma}}(c) = \mathcal{I}(c)$ \\
                    $\mathcal{I_{\sigma}}(f(t_{1},...,t_{n})) =
                    f^{\mathcal{I}}(\mathcal{I_{\sigma}}(t_{1}),...,
                    \mathcal{I_{\sigma}}(t_{n}))$                
                \end{center}
            \end{teo}

            \begin{teo}[\textbf{Interpretación de fórmulas}]
                Sea $\sigma$ un estado de las variables. Definimos la
                función de interpretación o significado de las fórmulas con 
                respecto a $\sigma, \mathcal{I_{\sigma}}: FORM \rightarrow
                \{ 0,1 \}$ como sigue:
                
                \begin{center}
                    $\mathcal{I_{\sigma}}(\bot) = 0$ \\
                    $\mathcal{I_{\sigma}}(\top) = 1$ \\
                    $\mathcal{I_{\sigma}}(P(t_{1},...,t_{m})) = 1$ si y 
                    sólo si $(\mathcal{I_{\sigma}}(t_{1}), ...,
                    \mathcal{I_{\sigma}}(t_{m})) \in P^{\mathcal{I}}$ \\
                    $\mathcal{I_{\sigma}}(t_{1}=t_{2}) = 1$ si y sólo si 
                    $\mathcal{I_{\sigma}}(t_{1}) = 
                    \mathcal{I_{\sigma}}(t_{2})$ \\
                    $\mathcal{I_{\sigma}}(\neg \varphi) = 1$ si y sólo si 
                    $\mathcal{I_{\sigma}}(\varphi) = 0$ \\
                    $\mathcal{I_{\sigma}}(\varphi \land \psi) = 1$ si y sólo 
                    si $\mathcal{I_{\sigma}}(\varphi) = 
                    \mathcal{I_{\sigma}}(\psi) = 1$ \\
                    $\mathcal{I_{\sigma}}(\varphi \lor \psi) = 0$ si y sólo 
                    si $\mathcal{I_{\sigma}}(\varphi) = 
                    \mathcal{I_{\sigma}}(\psi) = 0$ \\
                    $\mathcal{I_{\sigma}}(\varphi \rightarrow \psi) = 0$ si 
                    y sólo si $\mathcal{I_{\sigma}}(\varphi) = 1$ e 
                    $\mathcal{I_{\sigma}}(\psi) = 0$ \\
                    $\mathcal{I_{\sigma}}(\varphi \leftrightarrow \psi) = 1$
                    si y sólo si $\mathcal{I_{\sigma}}(\varphi) = 
                    \mathcal{I_{\sigma}}(\psi)$ \\
                    $\mathcal{I_{\sigma}}(\forall x \varphi) = 1$ si y sólo si 
                    $\mathcal{I_{\sigma}}_{[x/m]}(\varphi) = 1$ para todo 
                    $m \in M$ \\
                    $\mathcal{I_{\sigma}}(\exists x \varphi) = 1$ si y sólo si
                    $\mathcal{I_{\sigma}}_{[x/m]}(\varphi) = 1$ para algún 
                    $m \in M$.
                \end{center}
            \end{teo}
        
        \end{itemize}
        
        % Ejercicio 2.
        \item Obtén la Forma Normal Prenex de la siguiente fórmula:
        
        \begin{center}
            $\forall x \forall y[\exists z (A(x,z) \land B(y,z)) 
            \rightarrow \exists u C(x,y,z)]$
        \end{center}

        \textsc{Solución:}\\
        Sea $\varphi = \forall x \forall y[\exists z (A(x,z) \land B(y,z)) 
        \rightarrow \exists u C(x,y,z)]$. Primero, rectificamos a $\varphi$:
        
        \begin{align*}
            rec(\varphi) 
            &= \forall x \forall y[\exists z (A(x,z) \land B(y,z)) \rightarrow 
            C(x,y,z)]
            && \text{eliminamos cuantificadores vacuos} \\
            &= \forall x \forall y[\exists w (A(x,w) \land B(y,w)) \rightarrow 
            C(x,y,z)]
            && \text{$\alpha - equivalencia$}
        \end{align*}

        Ahora, obtenemos $fnn(\varphi)$:
        
        \begin{align*}
            fnn(\varphi) 
            =& \forall x \forall y[\neg \exists w (A(x,w) \land B(y,w)) \lor
            C(x,y,z)]
            && \text{eq. lógica} \\
            =& \forall x \forall y[\forall w \neg (A(x,w) \land B(y,w)) \lor
            C(x,y,z)]
            && \text{eq. lógica} \\
            =& \forall x \forall y[\forall w (\neg A(x,w) \lor \neg B(y,w)) \
            lor C(x,y,z)]
            && \text{eq. lógica}
        \end{align*}
        
        Finalmente, obtenemos $fnp(\varphi)$:
        
        \begin{align*}
            fnp(\varphi) 
            =& \forall x \forall y \forall w [(\neg A(x,w) \lor \neg B(y,w)) 
            \lor C(x,y,z)] 
        \end{align*}
        
        Por lo tanto, la Forma Normal Prenex de $\varphi$ es $fnp(\varphi)$, 
        definido anteriormente.

        % Ejercicio 3.
        \item Obtén la Forma Normal de Skolem de la siguiente fórmula:
        
        \begin{center}
            $\forall x[(A(x,y) \rightarrow \exists y P(x,y,z)) 
            \rightarrow \neg \forall z Q(x,z)]$
        \end{center}
        
        \textsc{Solución:} \\
        Sea $\varphi = \forall x[(A(x,y) \rightarrow \exists y P(x,y,z)) 
        \rightarrow \neg \forall z Q(x,z)]$. Primero, rectificamos $\varphi$:
        
        \begin{align*}
            rec(\varphi)
            =& \forall x[(A(x,y) \rightarrow \exists u P(x,u,z)) 
            \rightarrow \neg \forall w Q(x,w)]
            && \text{$\alpha$ - equivalencia}
        \end{align*}
        
        Ahora, obtenemos $fnn(\varphi)$:
        
        \begin{align*}
            fnn(\varphi) 
            =& \forall x[\neg (\neg A(x,y) \lor \exists u P(x,u,z)) \lor \neg 
            \forall w Q(x,w)]
            && \text{eq. lógica} \\
            =& \forall x[(A(x,y) \land \neg \exists u P(x,u,z)) \lor \neg 
            \forall w Q(x,w)] 
            && \text{eq. lógica} \\
            =& \forall x[(A(x,y) \land \forall u \neg P(x,u,z)) \lor 
            \exists w \neg Q(x,w)]
            && \text{eq. lógica} 
        \end{align*}
        
        Luego, obtenemos $fnp(\varphi)$:
        
        \begin{align*}
            fnp(\varphi)
            =& \forall x \forall u \exists w[(A(x,y) \land \neg P(x,u,z)) \lor 
            \neg Q(x,w)] 
        \end{align*}
        
        Finalmente, obtenemos $fns(\varphi)$:
        
        \begin{align*}
            fns(\varphi) 
            =& \forall x \forall u[(A(x,y) \land \neg P(x,u,z)) \lor 
            \neg Q(x,f(x,u))] 
        \end{align*}

        % Ejercicio 4.
        \item Obtén la Forma Normal Clausular de la siguiente fórmula:
        
        \begin{center}
            $\{ \forall x (P(x,y) \rightarrow \exists y Q(y)), \exists x 
            \forall y (Q(y) \rightarrow P(y,x) \lor R(x)), \forall y(R(y) 
            \rightarrow \exists x \neg Q(a))\}$ \\
            $\models \forall x (Q(fa) \rightarrow Q(a))$
        \end{center}
        
        \textsc{Solución:} \\
        Primero, reorganizamos nuestro conjunto de fórmulas:
        
        \begin{center}
            $\{ \forall x (P(x,y) \rightarrow \exists y Q(y)), \exists x 
            \forall y (Q(y) \rightarrow P(y,x) \lor R(x)), \forall y(R(y) 
            \rightarrow \exists x \neg Q(a)), \neg \forall x (Q(fa) 
            \rightarrow Q(a)) \}$
        \end{center}
        
        de donde
        
        \begin{align*}
            \varphi =& \forall x (P(x,y) \rightarrow \exists y Q(y)) \land  
            \exists x \forall y (Q(y) \rightarrow P(y,x) \lor R(x)) \land 
            \forall y(R(y) \rightarrow \exists x \neg Q(a)) \land 
            \neg \forall x (Q(fa) \rightarrow Q(a))
        \end{align*}

        Así, primero rectificamos a $\varphi$:
        
        \begin{align*}
            rec(\varphi) 
            =& \forall x (P(x,y) \rightarrow \exists y Q(y)) \land  
            \exists x \forall y (Q(y) \rightarrow P(y,x) \lor R(x)) \land 
            \forall y(R(y) \rightarrow \neg Q(a)) \land \neg (Q(fa) 
            \rightarrow Q(a)) \\
            =& \forall x (P(x,y) \rightarrow \exists s Q(s)) \land  
            \exists w \forall u (Q(u) \rightarrow P(u,w) \lor R(w)) \land 
            \forall z(R(z) \rightarrow \neg Q(a)) \land \neg (Q(fa) 
            \rightarrow Q(a))
        \end{align*}
        
        donde eliminamos los cuantificadores vacuos y aplicamos $\alpha$-
        equivalencia. \\
        Ahora, obtenemos $fnn(\varphi)$:
        
        \begin{align*}
            fnn(\varphi) 
            =& \forall x (\neg P(x,y) \lor \exists s Q(s)) \land  
            \exists w \forall u (\neg Q(u) \lor P(u,w) \lor R(w)) \land 
            \forall z(\neg R(z) \lor \neg Q(a)) \land \neg 
            (\neg Q(fa) \lor Q(a)) \\
            =& \forall x (\neg P(x,y) \lor \exists s Q(s)) \land  
            \exists w \forall u (\neg Q(u) \lor P(u,w) \lor R(w)) \land 
            \forall z(\neg R(z) \lor \neg Q(a)) \land (Q(fa) \land \neg Q(a)) 
        \end{align*}

        Luego, obtenemos $fnp(\varphi)$:
        
        \begin{align*}
            fnp(\varphi) 
            =& \forall x \exists s(\neg P(x,y) \lor Q(s)) \land  
            \exists w \forall u (\neg Q(u) \lor P(u,w) \lor R(w)) \land 
            \forall z(\neg R(z) \lor \neg Q(a)) \land (Q(fa) \land \neg Q(a))\\
            =& \forall x \exists s \exists w \forall u \forall z[(\neg P(x,y) 
            \lor Q(s)) \land  (\neg Q(u) \lor P(u,w) \lor R(w)) \land 
            (\neg R(z) \lor \neg Q(a)) \land (Q(fa) \land \neg Q(a))]
        \end{align*}

        Después, obtenemos $fns(\varphi)$:
        
        \begin{align*}
            fns(\varphi)
            =& \forall x \exists w \forall u \forall z[(\neg P(x,y) \lor 
            Q(f(x))) \land  (\neg Q(u) \lor P(u,w) \lor R(w)) \land 
            (\neg R(z) \lor \neg Q(a)) \land (Q(fa) \land \neg Q(a))] \\
            =& \forall x \forall u \forall z[(\neg P(x,y) \lor Q(f(x))) 
            \land  (\neg Q(u) \lor P(u, h(x)) \lor R(h(x))) \land 
            (\neg R(z) \lor \neg Q(a)) \land (Q(fa) \land \neg Q(a))]
        \end{align*}

        Finalmente, obtenemos $Cl(\varphi)$:
        
        \begin{align*}
            Cl(\varphi)
            =& (\neg P(x,y) \lor Q(f(x))) \land  (\neg Q(u) \lor P(u, h(x)) 
            \lor R(h(x))) \land (\neg R(z) \lor \neg Q(a)) \land Q(fa) 
            \land \neg Q(a)
        \end{align*}

        % Ejercicio 5.
        \item Demuestra mediante deducción natural lo siguiente:
        
        \begin{itemize}
            % Ejercicio 5.1
            \item $\forall x (Hx \rightarrow Gx \land Kx), 
            \neg \exists z(Fz \land Gz) \vdash \forall w \neg (Fw \land Hw)$
            
            \begin{proof}
                Utilizando equivalencias lógicas, basta mostrar que 
                
                \begin{center}
                    $\Gamma = \{\forall x (Hx \rightarrow Gx \land Kx), 
                    \forall z (\neg Fz \lor \neg Gz)\} \vdash \forall w 
                    \neg (Fw \land Hw)$
                \end{center}
                
                Entonces 
                
                \begin{align*}
                    1.\; \; &\Gamma \vdash \forall x (Hx \rightarrow Gx \land Kx)
                    && \text{Hip} \\
                    2. \; \; &\Gamma \vdash \forall z (\neg Fz \lor \neg Gz)
                    && \text{Hip} \\
                    3. \; \; &\Gamma \vdash Hw \rightarrow Gw \land Kw
                    && \text{$(\forall E) 1$} \\
                    4. \; \; &\Gamma \vdash \neg Fw \lor \neg Gw
                    && \text{$(\forall E) 2$}  
                \end{align*}

                de donde, tenemos dos casos:
                
                \begin{align*}
                    5. \; \; &\Gamma, \neg Fw \vdash \neg Fw
                    && \text{Hip} \\
                    6. \; \; &\Gamma, \neg Fw \vdash \neg Fw \lor \neg Hw
                    && \text{$(\lor I) 5$} \\ 
                    7. \; \; &\Gamma, \neg Fw \vdash \neg (Fw \land Hw)
                    && \text{De Morgan, 6} \\
                    8. \; \; &\Gamma, \neg Fw \vdash \forall w \neg (Fw \land Hw)
                    && \text{$(\forall I) 7$} \\
                    9. \; \; &\Gamma, \neg Gw \vdash \neg Gw
                    && \text{Hip} \\
                    10. \; \; &\Gamma, \neg Gw \vdash \neg Gw \lor \neg Kw
                    && \text{$(\lor I) 9$} \\ 
                    11. \; \; &\Gamma, \neg Gw \vdash \neg (Gw \land Kw)
                    && \text{De Morgan, 10} \\
                    12. \; \; &\Gamma, \neg Gw \vdash \neg Hw
                    && \text{MT 11, 3} \\
                    13.\; \; &\Gamma, \neg Gw \vdash \neg Fw \lor \neg Hw
                    && \text{$(\lor I) 12$} \\
                    14. \; \; &\Gamma, \neg Gw \vdash \neg (Fw \land Hw)
                    && \text{De Morgan, 13} \\
                    15. \; \; &\Gamma, \neg Fw \vdash \forall w \neg (Fw \land Hw)
                    && \text{$(\forall I) 14$} 
                \end{align*}

                Por $(\lor E)$ podemos concluir que $\Gamma \vdash \forall w 
                \neg (Fw \land Hw)$.
            \end{proof}

            % Ejercicio 5.2
            \item $\exists x Fx \rightarrow \forall y(Gy \rightarrow Hy),
            \exists z Jz \rightarrow \exists w Gw \vdash \exists z(Fz \land Jz)
            \rightarrow \exists v Hv$
            
            \begin{proof}
                Por el Teo. de DN basta mostrar
                
                \begin{center}
                    $\Gamma = \{ \exists x Fx \rightarrow 
                    \forall y(Gy \rightarrow Hy), \exists z Jz \rightarrow 
                    \exists w Gw, \exists z(Fz \land Jz) \} \vdash\exists v Hv$
                \end{center}
                
                Entonces
                
                \begin{align*}
                    1. \; \; &\Gamma \vdash \exists x Fx \rightarrow 
                    \forall y(Gy \rightarrow Hy)
                    && \text{Hip} \\
                    2. \; \; &\Gamma \vdash \exists z Jz \rightarrow 
                    \exists w Gw
                    && \text{Hip} \\
                    3. \; \; &\Gamma \vdash \exists z(Fz \land Jz)
                    && \text{Hip} \\
                    4. \; \; &\Gamma, Fz \land Jz \vdash Fz \land Jz
                    && \text{Hip} \\
                    5. \; \; &\Gamma, Fz \land Jz \vdash Fz 
                    && \text{$(\land E) 4$} \\
                    6. \; \; &\Gamma, Fz \land Jz \vdash Jz
                    && \text{$(\land E) 4$} \\
                    7. \; \; &\Gamma, Fz \land Jz \vdash \exists x Fx
                    && \text{$(\exists I) 4$} \\
                    8. \; \; &\Gamma, Fz \land Jz \vdash \forall y(Gy 
                    \rightarrow Hy)
                    && \text{$(\rightarrow E) 6,1$} \\
                    9. \; \; &\Gamma, Fz \land Jz \vdash \exists z Jz
                    && \text{$(\exists I) 5$} \\
                    10. \; \; &\Gamma, Fz \land Jz \vdash \exists Gw
                    && \text{$(\rightarrow E) 8, 2$} \\
                    11. \; \;  &\Gamma, Fz \land Jz, Gw \vdash Gw 
                    && \text{Hip} \\
                    12. \; \; &\Gamma, Fz \land Jz, Gw \vdash Gw \rightarrow Hw
                    && \text{$(\forall E) 7$} \\
                    13. \; \; &\Gamma Fz \land Jz, Gw \vdash Hw 
                    && \text{$(\rightarrow E) 11,10$} \\
                    14. \; \; &\Gamma, Fz \land Jz, Gw \vdash \exists u Hu
                    && \text{$(\exists I) 12$} \\
                    15. \; \; &\Gamma, Fz \land Jz \vdash \exists u Hu
                    && \text{$(\exists E) 9, 12$} \\
                    16. \; \; &\Gamma \vdash \exists u Hu 
                    && \text{$(\exists E) 3, 15$}
                \end{align*}
                
                Por lo tanto, podemos concluir que $\Gamma \vdash \exists u Hu$.
            \end{proof}
        \end{itemize}
    \end{enumerate}
\end{document}