\documentclass[letterpaper,10pt]{article}

% Soporte para los acentos.
\usepackage[utf8]{inputenc}
\usepackage[T1]{fontenc}    
% Idioma español.
\usepackage[spanish,mexico, es-tabla]{babel}
% Soporte de símbolos adicionales (matemáticas)
\usepackage{multirow}
\usepackage{amsmath}		
\usepackage{amssymb}		
\usepackage{amsthm}
\usepackage{amsfonts}
\usepackage{latexsym}
\usepackage{enumerate}
\usepackage{ragged2e}
% Tablas
\usepackage{multirow}
% Código
\usepackage{listings}
%Tableaux
\usepackage{prooftrees}
% Modificamos los márgenes del documento.
\usepackage[lmargin=2cm,rmargin=2cm,top=2cm,bottom=2cm]{geometry}

\title{Lógica Computacional \\ Tarea 1}
\author{Rubí Rojas Tania Michelle}
\date{28 de febrero de 2019}

\begin{document}
    \maketitle

    \begin{enumerate}
        
        % Ejercicio 1.
        \item Enuncia formalmente lo siguiente:
        \begin{itemize}

            % Definición 1.
            \item[a)] Sintaxis de la lógica proposicional. \\
            \textit{Solución:} Definamos un lenguaje para la lógica de
            proposiciones.  \\
            El alfabeto consta de:

            \begin{itemize}
                \item Símbolos o variables proposicionales (un número
                infinito) : $p_{1}, ... , p_{n}, ...$
                \item Constantes lógicas: $\bot, \top$
                \item Conectivos u operadores lógicos: $\neg, \land, \lor 
                \rightarrow, \leftrightarrow$
                \item Símbolos auxiliares: $(,)$
            \end{itemize}

            El conjunto de expresiones o fórmulas atómicas, denotado
            $ATOM$ consta de:

            \begin{itemize}
                \item Las variables proposicionales: $p_{1}, ..., p_{n}, ...$
                \item Las constantes $\bot, \top$
            \end{itemize}

            Las expresiones que formarán nuestro lenguaje $PROP$, llamadas
            usualmente fórmulas, se definen recursivamente como sigue:

            \begin{itemize}
                \item Si $\varphi \in ATOM$ entonces $\varphi \in PROP$. Es 
                decir, toda fórmula atómica es una fórmula.
                \item Si $\varphi \in PROP$ entonces 
                $(\neg \varphi) \in PROP$.
                \item $\varphi, \psi$ entonces $(\varphi \land \psi), 
                (\varphi \lor \psi), (\varphi \rightarrow \psi), 
                (\varphi \leftrightarrow \psi) \in PROP$.
                \item Son todas.
            \end{itemize}

            % Definición 2. 
            \item[b)] Semántica de la lógica proposicional. \\ 
            \textit{Solución:}
            \newtheorem{teo}{Definición}[]
            \begin{teo} 
                El tipo de valores booleanos denotado \textbf{Bool} se define
                como $Bool = \{0, 1 \}$
            \end{teo}

            \begin{teo}
                Un estado o asignación de las variables (proposicionales) es 
                una función 
                \begin{center}
                    $\mathcal{I} : (Var P) \rightarrow Bool$
                \end{center}
                Dadas $n$ variables proposicionales existen $2^{n}$ estados
                distintos para estas variables.
            \end{teo}

            \begin{teo}
                Dado un estado de las variables 
                $\mathcal{I} : (Var P) \rightarrow Bool$, definamos la
                interpretación de las fórmulas con respecto a $\mathcal{I}$ 
                como la función $\mathcal{I^{\star}} : PROP \rightarrow Bool$
                tal que: 
                \begin{itemize}
                    \item $\mathcal{I^{\star}}(p) = \mathcal{I}(p)$ para 
                    $p \in Var P$, es decir, 
                    $\mathcal{I^{\star}}|_{Var P} = \mathcal{I}$
                    \item $\mathcal{I^{\star}}(\top) = 1$
                    \item $\mathcal{I^{\star}}(\bot) = 0$
                    \item $\mathcal{I^{\star}}(\neg \varphi) = 1$ sii 
                    $\mathcal{I^{\star}}(\varphi) = 0$
                    \item $\mathcal{I^{\star}}(\varphi \land \psi) = 1$
                    sii $\mathcal{I^{\star}}(\varphi) = 
                    \mathcal{I^{\star}}(\psi) = 1$
                    \item $\mathcal{I^{\star}}(\varphi \lor \psi) = 0$
                    sii $\mathcal{I^{\star}}(\varphi) = 
                    \mathcal{I^{\star}}(\psi) = 0$
                    \item $\mathcal{I^{\star}}(\varphi \rightarrow \psi) = 0$
                    sii $\mathcal{I^{\star}}(\varphi) = 1$ e
                    $\mathcal{I^{\star}}(\psi) = 0$
                    \item $\mathcal{I^{\star}}(\varphi \leftrightarrow \psi) = 1$
                    sii $\mathcal{I^{\star}}(\varphi) = 
                    \mathcal{I^{\star}}(\psi)$

                \end{itemize}
            \end{teo}

        \end{itemize}

        % Ejercicio 2.
        \item Dado el conjunto de proposiciones 
        $\Gamma = \{ \neg (p \land q), (t \leftrightarrow r), q, (\neg r) \}$.
        Verifica si el conjunto $\Gamma$ es tautología, satisfacible o
        insatisfacible. \\
        \textit{Solución:} Veamos si $\Gamma$ es satisfacible: \\
        Inicialmente podemos asignar los estados $\mathcal{I}(q) = 1$ y
        $(\mathcal{I}(\neg r) = 1$ sii $\mathcal{I}(r) = 0)$ ya que $q$ y $r$
        se encuentran solitas. Ahora, sabemos que 
        $\mathcal{I}(t \leftrightarrow r) = 1$ sii 
        $\mathcal{I}(t) = \mathcal{I}(r)$, por lo que 
        $\mathcal{I}(r) = 0 = \mathcal{I}(t)$. Finalmente, como 
        $\mathcal{I}(q) = 1$ entonces $\mathcal{I}(p) = 0$ para que 
        $\mathcal{I}(\neg (p \land q)) = 1$. \\
        Por lo tanto, como $\mathcal{I}(q) = 1$ e 
        $\mathcal{I}(p) = \mathcal{I}(r) = \mathcal{I}(t) = 0$ hacen que 
        $\mathcal{I}(\Gamma) = 1$, entonces $\Gamma$ es satisfacible. \\

        Más aún, podemos afirmar que $\Gamma$ no es una tautología, ya que se 
        puede dar una asignación de estado que no hace a $\Gamma$ verdadera.
        En particular, 
        $\mathcal{I}(p) = \mathcal{I}(q) = \mathcal{I}(t) =\mathcal{I}(r)= 1$ 
        hacen que $\mathcal{I}(\Gamma) = 0$.

        % Ejercicio 3.
        \item Utilizando interpretaciones verifica si el siguiente argumento
        es verdadero o falso.
        \begin{center}
            $\{ (p \land q), (q \lor r), (\neg s) \} \models p \land s$
        \end{center}
        \textit{Solución:} Debemos mostrar que $\mathcal{I}(p \land s) = 1$.
        Entonces 
        \begin{align*}
            \mathcal{I}(p \land q) &= 1
            && \text{Premisa (1)} \\
            \mathcal{I}(q \lor r) &= 1
            && \text{Premisa (2)} \\
            \mathcal{I}(\neg s) &= 1 
            && \text{Premisa (3)} \\
            \mathcal{I}(p) &= 1
            && \text{por (1)} \\
            \mathcal{I}(q) &= 1
            && \text{por (1)} \\
            \mathcal{I}(s) &= 0
            && \text{por (3)}
        \end{align*}

        De manera que la interpretación dada por 
        $\mathcal{I}(p) = \mathcal{I}(q) = 1$ e $\mathcal{I}(s) = 0$ e 
        $\mathcal{I}(r) =$ arbitrario, es un contraejemplo al argumento, pues 
        con esta interpretación tenemos que $\mathcal{I}(p \land s) = 0$.
        Por lo tanto, el argumento es falso.

        % Ejercicio 4.
        \item Demuestra que los siguientes secuentes son válidos usando
        deducción natural.
        \begin{itemize}

            % Demostración 1.
            \item[a)] $\{ p \rightarrow (q \lor r) \} \vdash
            (p \rightarrow q) \lor (p \rightarrow r)$
            \begin{proof}
                Utilizando una estrategia de derivación que está en las notas 
                de clase, basta derivar 
                $\{ p \rightarrow (q \lor r) \} \vdash (p \rightarrow q)$ o
                bien, $\{ p \rightarrow (q \lor r) \} \vdash 
                (p \rightarrow r)$. Así, 
                \begin{align*}
                    p \rightarrow (q \lor r)
                    && \text{Premisa (1)} \\
                    \neg p \lor (q \lor r)
                    && \text{Eliminación de equivalencia en 1, (2)} \\
                    (\neg p \lor \neg p) \lor (q \lor r)
                    && \text{idempotencia en 2, (3)} \\
                    (\neg p \lor q) \lor (\neg p \lor r)
                    && \text{conmuta. y asocia. en 3, (4)} \\
                    (p \rightarrow q) \lor (p \rightarrow r)
                    && \text{$P \rightarrow Q \equiv \neg P \lor Q$, 
                    en 4, (5)}
                \end{align*}
                Por lo tanto, podemos concluir que es válido 
                $\{ p \rightarrow (q \lor r) \} \vdash
                (p \rightarrow q) \lor (p \rightarrow r)$

            \end{proof}

            % Demostración 2.
            \item[b)] $\{ \} \vdash p \lor (q \land r) 
            \rightarrow (p \land r) \lor q$
            \begin{proof}
                Utilizando una estrategia de derivación que está en las notas
                de clase, basta derivar 
                $\{p\lor (q \land r) \} \vdash (p \land r) \lor q$. Así, 
                \begin{align*}
                    p \lor (q \land r) 
                    && \text{Premisa (1)} \\
                    q \lor (p \land r) 
                    && \text{Conmuta. y asocia. en 1, (2)} \\
                    (p \land r) \lor q
                    && \text{Conmutatividad en 2, (3)}
                \end{align*}
                Por lo tanto, podemos concluir que es válido
                $\{p\lor (q \land r) \} \vdash (p \land r) \lor q$
            \end{proof}
        
        \end{itemize}

        \newpage
        % Ejercicio 5.
        \item Realiza las siguientes sustituciones eliminando los paréntesis
        innecesarios en el resultado:
        \begin{itemize}
            
            % Sustitución 1.
            \item[a)] $((q \lor r) [q, p := \neg p, s] 
            \rightarrow (r \land \neg (r \leftrightarrow p)))
            [p, r, q := r \lor q, q \land p, s]$ \\
            \textit{Solución: \\ \\}
            $((q \lor r) [q, p := \neg p, s] \rightarrow (r \land \neg (r \leftrightarrow p)))
            [p, r, q := r \lor q, q \land p, s] $ \\ \\
            $= (((\neg p) \lor r) \rightarrow (r \land \neg (r \leftrightarrow p)))
            [p, r, q := r \lor q, q \land p, s] $ \\
            $= ((\neg (r \lor q) \lor (q \land p)) \rightarrow 
            ((q \land p) \land \neg ((q \land p) \leftrightarrow (r \lor q))))$ \\
            $= 7((\neg (r \lor q) \lor q \land p) \rightarrow
            ((q \land p) \land \neg (q \land p \leftrightarrow r \lor q)))$  
            
            % Sustitución 2.
            \item[b)] $(u \lor t) \rightarrow 
            (\neg r \leftrightarrow (u \leftrightarrow s))[r, u, t := u, t, r]$ \\
            \textit{Solución:}
            \begin{align*}
                (u \lor t) \rightarrow (\neg r \leftrightarrow (u \leftrightarrow s))
                [r, u, t := u, t, r]
                &= (u \lor t) \rightarrow (\neg (u) \leftrightarrow ((t) \leftrightarrow s)) \\
                &= (u \lor t) \rightarrow (\neg u \leftrightarrow (t \leftrightarrow s))
            \end{align*}

        \end{itemize}

        % Ejercicio 6. 
        \item Realizar el Tableaux de la siguiente fórmula en $PL$ y da el
        modelo que satisfaga la fórmula en caso de que el Tableaux sea 
        abierto.
        \begin{center}
            $\neg ((q \lor \neg (p \rightarrow r)) 
            \rightarrow (p \land (q \rightarrow r)))$
        \end{center}
        \textit{Solución:} Primero, eliminamos las implicaciones y simplificamos la 
        expresión para facilitar la elaboración de nuestro Tableaux:
        \begin{align*}
            \neg ((q \lor \neg (p \rightarrow r)) 
            \rightarrow (p \land (q \rightarrow r)))
            &\equiv \neg (\neg (q \lor \neg (\neg p \lor r)) 
            \lor (p \land (\neg q \lor r)))
            && \text{ya que $P \rightarrow Q \equiv \neg P \lor Q$} \\
            &\equiv \neg (\neg (q \lor (p \land \neg r)) 
            \lor (p \land (\neg q \lor r)))
            && \text{De Morgan} \\
            &\equiv \neg ((\neg q \land (\neg p \lor r)) 
            \lor (p \land (\neg q \lor r)))
            && \text{De Morgan} \\
            &\equiv \neg (\neg q \land (\neg p \lor r)) 
            \land \neg (p \land (\neg q \lor r)))
            && \text{De Morgan} \\
            &\equiv (q \lor (p \land \neg r)) 
            \land (\neg p \lor (q \land \neg r)))
            && \text{De Morgan} \\
            &\equiv ((q \lor p) \land (q \lor \neg r))
            \land ((\neg p \lor q) \land (\neg p \lor \neg r)))
            && \text{distributividad} \\
            &\equiv (q \lor p) \land (q \lor \neg r)
            \land (\neg p \lor q) \land (\neg p \lor \neg r)
            && \text{eliminando paréntesis} \\
            &\equiv (q \lor q) \land (q \lor \neg r) \land (\neg p \lor \neg r)
            && \text{resolución binaria} \\
            &\equiv (q \land q) \lor (\neg r \land \neg p \lor \neg r)
            && \text{idempotencia y asocia.} \\
            &\equiv q \lor (\neg r \land \neg p \lor \neg r)
            && \text{idempotencia} \\
            &\equiv q \lor (\neg p \land (\neg r \lor \neg r))
            && \text{asociatividad y conmuta.} \\
            &\equiv q \lor (\neg p \land \neg r)
            && \text{idempotencia} \\
            &\equiv (\neg p \lor \neg r) \land q
            && \text{asociatividad y conmuta.}
        \end{align*}

        Ahora que tenemos una expresión equivalente más simple a la original,
        procedemos a realizar su Tableaux:

        \begin{center}
        \begin{prooftree}{}
            [(\neg p \lor \neg r) \land q, checked
               [q, just={ext. de $\alpha$ en 1}
                  [\neg p \lor \neg r, checked, just={ext. de $\alpha$ en 1}
                     [p, just={ext. de $\beta$ en 3}]
                        [\neg r]]]]
        \end{prooftree}
        \end{center}

        
        Como el Tableaux es abierto, entonces podemos dar un modelo que
        satisfaga la fórmula. En particular, $\mathcal{I}(q) = 1$ e 
        $\mathcal{I}(p) = \mathcal{I}(r) = 0$ satisface la fórmula. 

        Nota: Una de las justificaciones para obtener una expresión equivalente
        es la resolución binaria. Por falta de tiempo ya no pude transcribirla
        a latex, pero la incluyo anexada junto con la tarea.

        \newpage
        % Ejercicio 7.
        \item Obtener la forma normal conjuntiva de las siguientes fórmulas
        (mencionando la operación realizada en cada paso):
        \begin{itemize}
            
            % Ejercicio 7.1
            \item[a)] $((q \rightarrow r) \rightarrow q) 
            \land (r \rightarrow q)$ \\
            \textit{Solución:}
            \begin{align*}
                ((q \rightarrow r) \rightarrow q) \land (r \rightarrow q)
                &\equiv (\neg (\neg q \lor r) \lor q) \land (\neg r \lor q)
                && \text{ya que $P \rightarrow Q \equiv \neg P \lor Q$} \\
                &\equiv ((q \land \neg r) \lor q) \land (\neg r \lor q)
                && \text{De Morgan} \\
                &\equiv (\neg r \land (q \lor q)) \land (\neg r \lor q)
                && \text{asociatividad y conmutatividad} \\
                &\equiv (\neg r \land q) \land (\neg r \lor q)
                && \text{idempotencia} \\
                &\equiv (\neg r \land \neg r) \land (q \lor q)
                && \text{asociatividad y conmutatividad}\\
                &\equiv \neg r \land q
                && \text{idempotencia}
            \end{align*}

            Como $\varphi = \neg r \land q$ es una conjunción de literales, 
            entonces $\varphi$ es de la Forma Normal Conjuntiva.

            \item[b)] $\neg p \land q \rightarrow p \land (r \rightarrow q)$ \\
            \textit{Solución:} 
            \begin{align*}
                \neg p \land q \rightarrow p \land (r \rightarrow q)
                &\equiv (\neg p \land q) \rightarrow (p \land (r \rightarrow q))
                && \text{precedencia y asociatividad de conectivos} \\
                &\equiv \neg (\neg p \land q) \lor (p \land (\neg r \lor q))
                && \text{{ya que $P \rightarrow Q \equiv \neg P \lor Q$}} \\
                &\equiv (p \lor \neg q) \lor (p \land (\neg r \lor q))
                && \text{De Morgan} \\
                &\equiv (p \lor p) \lor (\neg q \land (\neg r \lor q))
                && \text{asociatividad y conmutatividad} \\
                &\equiv p \lor (\neg q \land (\neg r \lor q))
                && \text{idempotencia} \\
                &\equiv (p \lor \neg q) \land (\neg r \lor q)
                && \text{asociatividad}
            \end{align*}

            Como $\varphi =  (p \lor \neg q) \land (\neg r \lor q)$ es una 
            conjunción de disyunciones, entonces $\varphi$ es de la Forma 
            Normal Conjuntiva.
             
        \end{itemize}

        % Ejercicio 8.
        \item Obtener la Forma Normal Disyuntiva de 
        $\neg (w \rightarrow \neg p) \lor \neg ((\neg s \leftrightarrow w) 
        \lor (p \land s))$. \\
        \textit{Solución:} 
        \begin{align*}
            \neg (w \rightarrow \neg p) \lor \neg ((\neg s \leftrightarrow w) 
            \lor (p \land s))
            &\equiv \neg (\neg w \lor \neg p) \lor 
            \neg ((\neg s \leftrightarrow w) \lor (p \land s)) \\
            &\equiv \neg (\neg w \lor \neg p) \lor 
            \neg (((s \lor w) \land (\neg s \lor \neg w)) \lor (p \land s)) \\
            &\equiv (w \land p) \lor (((\neg s \land \neg w) \lor 
            (s \land w)) \land (\neg p \lor \neg s)) \\
            &\equiv (w \land p) \lor (\neg s \land \neg w) \lor (s \land w)
            \land (\neg p \lor \neg s) \\
            &\equiv (w \land p) \lor (\neg s \land \neg w) \lor (s \land w
            \land \neg p) \lor \neg s
        \end{align*}

        Por falta de espacio, no pude colocar arriba la justificación de cada 
        paso, pero lo explico en seguida:
        \begin{itemize}
            \item Eliminamos la implicación, ya que 
            $P \rightarrow Q \equiv \neg P \lor Q$.
            \item Elimamos la doble implicación, ya que 
            $P \leftrightarrow Q \equiv (\neg P \lor Q) \land (P \lor \neg Q)$
            \item Hacemos que las negaciones figuren únicamente en las variables
            proposicionales.
            \item Eliminamos los paréntesis innecesarios, pues todos los
            operadores tienen la misma precedencia.
            \item Aplicamos asociatividad.
        \end{itemize}

        Como $\varphi = (w \land p) \lor (\neg s \land \neg w) \lor (s \land w
        \land \neg p) \lor \neg s$ es una disyunción de conjunciones o
        literales, entonces $\varphi$ es de la Forma Normal Disyuntiva.
        
        % Ejercicio 9.
        \item Obtener la Forma Normal Negativa de 
        $(p \land (q \rightarrow r)) \rightarrow s$ \\
        \textit{Solución:}
        \begin{align*}
            (p \land (q \rightarrow r)) \rightarrow s 
            &\equiv \neg (p \land (\neg q \lor r)) \lor s
            && \text{ya que $P \rightarrow Q \equiv \neg P \lor Q$} \\
            &\equiv (\neg p \lor \neg (\neg q \lor r)) \lor s
            && \text{prop. de $\neg$} \\
            &\equiv (\neg p \lor (\neg \neg q \land \neg r)) \lor s
            && \text{{prop. de $\neg$}} \\
            &\equiv (\neg p \lor (q \land \neg r)) \lor s
            && \text{ya que $\neg \neg P \equiv P$}
        \end{align*}

        Como $\varphi = (\neg p \lor (q \land \neg r)) \lor s$ no contiene 
        implicaciones ni equivalencias, y las negaciones que figuran en 
        $\varphi$ sólo afectan a fórmulas atómicas, entonces $\varphi$
        es de la Forma Normal Negativa.
            
        \newpage
        % Ejercicio 10.
        \item 
        \begin{itemize}

            % Ejercicio 10.1
            \item [a)] Define una función recursiva \textbf{pa} que, dada una 
            fórmula $\phi$, devuelve el número de paréntesis abiertos $"("$ que
            tiene $\phi$. \\
            \textit{Solución:} Definimos recursivamente la función 
            $\textbf{pa} :: PROP \rightarrow \mathbb{N}$ de la siguiente forma:

            \begin{itemize}
                \item $pa (\top) = 0$
                \item $pa (\bot) = 0$
                \item $pa (Var P) = 0$
                \item $pa (\neg \varphi) = pa (\varphi)$
                \item $pa ((\varphi \star \psi)) = pa(\varphi) + pa(\psi) + 1$
            \end{itemize}

            % Ejercicio 10.2
            \item[b)] Define una función recursiva \textbf{pc} que dada una 
            fórmula $\phi$, devuelve el número de paréntesis cerrados $")"$ que
            tiene $\phi$. \\
            \textit{Solución:} Definimos recursivamente la función 
            $\textbf{pc} :: PROP \rightarrow \mathbb{N}$ de la siguiente forma:

            \begin{itemize}
                \item $pc (\top) = 0$
                \item $pc (\bot) = 0$
                \item $pc (Var P) = 0$
                \item $pc (\neg \varphi) = pc (\varphi)$
                \item $pc ((\varphi \star \psi)) = pc (\varphi) + pc (\psi) + 1$
            \end{itemize}
            
            % Ejercicio 10.3
            \item[c)] Sea $\phi = ((( \neg p \land q) \lor \neg r) 
            \rightarrow r)$. Prueba que 
            \textbf{pa}$(\phi)$ - \textbf{pc}$(\phi) = 0$. 
            \begin{proof}
                Aplicamos la definición de \textbf{pa} a $\phi$:

                \begin{align*}
                    pa(\phi) 
                    &= pa(((( \neg p \land q) \lor \neg r) \rightarrow r))
                    && \text{def. de $\phi$.} \\
                    &= pa(((\neg p \land q) \lor \neg r)) + pa(r) + 1
                    && \text{def. recursiva de \textbf{pa}} \\
                    &= pa((\neg p \land q)) + pa(\neg r) + 1 + 0 + 1
                    && \text{def. recursiva de \textbf{pa}} \\
                    &= pa(\neg p) + pa(q) + 1 + pa(r) + 1 + 0 + 1
                    && \text{def. recursiva de \textbf{pa}} \\
                    &= pa(p) + 0 + 1 + 0 + 1 + 0 + 1
                    && \text{def. recursiva de \textbf{pa}} \\
                    &= 0 + 0 + 1 + 0 + 1 + 0 + 1
                    && \text{def. recursiva de \textbf{pa}} \\
                    &= 3
                    && \text{aritmética}
                \end{align*}

                Análogamente, aplicamos la definición de \textbf{pc} a $\phi$:
                \begin{align*}
                    pc(\phi) 
                    &= pc(((( \neg p \land q) \lor \neg r) \rightarrow r))
                    && \text{def. de $\phi$} \\
                    &= pc(((\neg p \land q) \lor \neg r)) + pc(r) + 1
                    && \text{def. recursiva de \textbf{pc}} \\
                    &= pc((\neg p \land q)) + pc(\neg r) + 1 + 0 + 1
                    && \text{def. recursiva de \textbf{pc}} \\
                    &= pc(\neg p) + pc(q) + 1 + pa(r) + 1 + 0 + 1
                    && \text{def. recursiva de \textbf{pc}} \\
                    &= pc(p) + 0 + 1 + 0 + 1 + 0 + 1
                    && \text{def. recursiva de \textbf{pc}} \\
                    &= 0 + 0 + 1 + 0 + 1 + 0 + 1
                    && \text{def. recursiva de \textbf{pc}} \\
                    &= 3
                    && \text{aritmética}
                \end{align*}

                Por lo tanto, 
                \textbf{pa}$(\phi)$ - \textbf{pc}$(\phi) = 3 - 3 = 0$.

            \end{proof}

        \end{itemize}

        \newpage
        % Ejercicio 11.
        \item Define recursivamente una función \textbf{compress} que comprime 
        los elementos consecutivos repetidos de una lista. Ejemplo:
        \textit{> compress ''mooloolaba'' = ''mololaba''}. Prueba, usando tu
        definición, que:
        \begin{center}
            \textit{compress [1,2,2,3,3,3] = [1,2,3]}
        \end{center}

        \textit{Solución:}
        \begin{lstlisting}
            compress :: (Eq a) => [a] -> [a]
            compress [] = []
            compress [a] = [a]
            compress xs =
                if (head xs) == (head (tail xs)) then compress (drop 1 xs)
                else (head xs) : compress (tail xs)
         \end{lstlisting}

         Finalmente, probemos que \textit{compress [1,2,2,3,3,3] = [1,2,3]}
         \begin{proof}
            \begin{align*}
                compress [1,2,3,3,3] 
                &= 1 : compress [2,3,3,3] \\
                &= 1 : 2 : compress [3,3,3] \\
                &= 1 : 2 : compress [3,3] \\
                &= 1 : 2 : compress [3] \\
                &= 1 : 2 : [3] \\
                &= [1, 2, 3]
            \end{align*} 

         \end{proof}

    \end{enumerate}

\end{document}